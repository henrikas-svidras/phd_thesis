\chapter{Experimental overview of rare radiative decays}\label{ch:exp_overview}

There are two main types of analyses which retain the meaning 
from the theoretical description presented in \Cref{ch:theory}: \textit{inclusive} analyses and \textit{exclusive} analyses.
Another method, which is a mixture of the two methods, is known as the \textit{sum-of-exclusive} method.
Theoretical motivations for inclusive measurements of \BtoXsgamma have already been discussed in \Cref{sec:btosgamma_totalrate_theory,sec:btosgamma_spectrum_theory}.
Similar motivations hold for other electroweak decay channels, such as $B\to X_s\ell\bar{\ell}$ ($\ell\in\{\mu^-,e^-,\nu\}$).
However, compared to the latter, $B\rightarrow X_s\gamma$ contains one fewer interaction vertex and is therefore enhanced by two orders of magnitude, which makes it experimentally accessible with smaller datasets.
In this chapter, the main methods of performing an inclusive measurement are introduced while focusing predominantly on \BtoXsdgamma, although, generally, these methods are applicable to other decay channels as well.

\section{Past measurements of \safeBtoXsdgamma decays}

Inclusive measurements target a wide selection of decay products; in the case of $X_{sd}$, the sum of all states that originate in $b\to s$ or $b\to d$ transitions.
Such states include resonant and non-resonant particles.
A notable resonant state is $B\rightarrow \Kstar(892)\gamma$.
Experimentally, it is highly accessible due to its narrow and isolated peak near the $b\to s\g$ two-body decay kinematic limit.
A similar state for $b\rightarrow d\g$ is $B\rightarrow \rho(770)\gamma$.
Non-resonant states include combinations of one or more pions and, in the case of $b\rightarrow s$, kaons.
While the goal of an inclusive measurement is to measure the energy of photons from all these decays simultaneously, an exclusive measurement attempts to select one or several particular states from the spectrum.

The world-average values of inclusive radiative $B$ decay measurements are~\cite{Amhis:2022mac,Workman:2022ynf}:
\begin{align}\label{eq:btosgamma_experimental}
    \begin{split}
    \mathcal{B}(\BtoXsgamma)=(3.49\pm 0.19)\times10^{-4}\\
    \mathcal{B}(\BtoXdgamma)=(0.09\pm 0.03)\times10^{-4}.
    \end{split}
\end{align}
The values of \Cref{eq:btosgamma_experimental} can be compared to the theoretical prediction in \Cref{eq:btosgamma_theoretical,eq:btodgamma_theoretical}.
\BtoXsgamma results show an excellent agreement between theory and experiment.
For \BtoXdgamma, values agree within $2\sigma$, however, the dominant source of experimental uncertainty is statistical. 

\Cref{tab:btosgamma_bfs} provides the experimental status of observed the most prominent observed \BtoXsdgamma decay channels as of December 2022.
Due to the final state similarity and overlap between various resonant and non-resonant decay modes, only the most prominent resonant decays have been measured.
However, even in the case of a relatively isolated decay channel, such as \BtoKstargamma, the overall precision of inclusive measurements is higher (compare to \Cref{eq:btosgamma_experimental}).

{\renewcommand{\arraystretch}{1.2}
\begin{table}[!htbp]
    \centering
    \caption{\label{tab:btosgamma_bfs} 
    Branching fractions of \BtoXsgamma modes for charged and neutral modes.
    The table only includes decay modes that have been observed and (for $\BtoXsgamma$ only) have a branching fraction $\gtrsim10^{-5}$.
    The \Bp decays are ordered in terms of the experimental precision $\mathcal{B}/\Delta\mathcal{B}$, whereas \Bz are ordered in relation to \Bp, where applicable.
    The values correspond to the averages of experimental measurements given in Refs. \cite{Amhis:2022mac,Workman:2022ynf}.
    }
    \begin{minipage}[c]{0.5\textwidth}
    \centering
    \BptoXsgamma exclusive modes
    \resizebox{1\textwidth}{!}{
    \begin{tabular}{l|c}
        Decay mode & Branching fraction ($\times 10^{-4}$) \\
        \hline
        \multicolumn{2}{c}{\textbf{Two decay products}}\\
        $\Bp \to \Kstar(892)^+ \gamma$  & $0.392 \pm 0.022$ \\
        $\Bp \to K_1(1270)^+  \gamma$   & $0.438 \pm ^{0.071}_{0.063}$ \\
        $\Bp \to K_2^*(1430)^+  \gamma$ & $0.138 \pm 0.040$ \\            
        $\Bp \to K^*(1410)^+  \gamma$   & $0.271 \pm ^{0.080}_{0.061}$ \\
        $\Bp \to K^*(1680)^+  \gamma$   & $0.670 \pm ^{0.170}_{0.140}$ \\
        $\Bp \to K_1(1400)^+  \gamma$   & $0.097 \pm ^{0.054}_{0.038}$ \\
        \multicolumn{2}{c}{\textbf{Three or more decay products}}\\
        $\Bp \to \Kstar(892)^0 \pi^+ \gamma$ & $0.233 \pm 0.012$ \\
        $\Bp \to K^+ \pi^+\pi^- \gamma$ & $0.258 \pm 0.015$ \\
        $\Bp \to K^0 \pi^+\pi^0 \gamma$ & $0.456 \pm 0.052$ \\
        $\Bp \to K^+ \pi^+\pi^- \gamma$ (non-resonant) & $0.099 \pm ^{0.017}_{0.020}$ \\
    \end{tabular}
    }
\end{minipage}
\begin{minipage}[c]{0.395\textwidth}
    \centering
    \BztoXsgamma exclusive modes
    \resizebox{1\textwidth}{!}{
    \begin{tabular}{l|c}
        Decay mode & Branching fraction ($\times 10^{-4}$) \\
        \hline
        \multicolumn{2}{c}{\textbf{Two decay products}}\\
        $\Bz \to \Kstar(892)^0 \gamma$ & $0.418 \pm 0.025$ \\
        - & -\\
        $\Bz \to K_2^*(1430)^0 \gamma$ & $0.124 \pm 0.024$ \\ 
        - & -\\
        - & -\\
        - & -\\
        \multicolumn{2}{c}{\textbf{Three or more decay products}}\\
        - & -\\
        $\Bz \to K^+ \pi^-\pi^0 \gamma$ & $0.407 \pm 0.038$\\
        $\Bz \to K^0 \pi^+\pi^- \gamma$ & $0.199 \pm 0.018$\\
        - & -\\
    \end{tabular}
    }
\end{minipage}

\vspace{10pt}

\begin{minipage}[c]{0.395\textwidth}
    \centering
    $B\to X_d\gamma$ exclusive modes
    \resizebox{1\textwidth}{!}{
    \begin{tabular}{l|c}
        Decay mode & Branching fraction ($\times 10^{-4}$) \\
        \hline
        $\Bp \to \rho^+(770)\gamma$ & $0.0098 \pm ^{0.0025}_{0.0024}$\\
        $\Bz \to \rho^0(770)\gamma$ & $0.0086 \pm 0.0015$\\
        $\Bz \to \omega(782)\gamma$ & $0.0044 \pm ^{0.0018}_{0.0016}$\\
    \end{tabular}
    }
\end{minipage}

\end{table}
}

Currently, inclusive measurements are only attainable at $B$ factories.
The relatively low-background environment offered by $\epem$ collisions enables treating the $X_{s/d}$ system as a `missing-momentum' system with few explicit requirements.
Although the process of event reconstruction may introduce biases to the inclusive system, this is expected to be a much smaller effect than other experimental factors, such as finite resolution or statistical fluctuations.
Conversely, at hadron colliders,
large proton-proton collision background makes it complicated to select an unbiased and model-independent inclusive sample.
This is further complicated by multiple proton pairs interacting in a collision event and creating large multijet backgrounds, usually referred to as \textit{pileup}.
At the time of writing this, no inclusive $B$ measurement has been performed outside \epem collision experiments.

Even in the case of exclusive radiative measurements, such as $B\rightarrow \Kstar\gamma$, $B$ factories have historically outperformed hadron collider experiments (such as LHCb) due to the cleaner \epem collision environment.
Final states that include neutral particles and photons are problematic to measure accurately in hadron colliders.
Therefore, with several exceptions (e.g. \cite{Bellee:2019qbt}), the field of rare radiative $B$ decay measurements is dominantly probed by the $B$ factory experiments.



\section{Techniques for inclusive \safeBtoXsgamma measurements}\label{sec:btosgamma_techniques}
Historically, three different techniques were applied for inclusive \BtoXsgamma analyses at $B$ factories: 
sum-of-exclusive measurements,
untagged inclusive measurements, and
tagged inclusive measurements.
These experiment techniques are explained in more detail in this Section.
The primary focus will be given to the hadronic-tagged technique, which is applied for the measurement described in this thesis.
The summary of the most precise measurements that are used in the experimental average of \Cref{eq:btosgamma_experimental} is given in \Cref{tab:btosgamma_inclusive_summary}.

{\renewcommand{\arraystretch}{1.2}
 \begin{table}[!htbp]
     \centering
     \caption{\label{tab:btosgamma_inclusive_summary}
     The table shows different experiments and their most precise results using various techniques of measuring \BtoXsgamma.
     These results are included in the total \BtoXsgamma world average (\Cref{eq:btosgamma_experimental}) \cite{Amhis:2022mac,Workman:2022ynf}.
     The thresholds of the photon energy in the decaying $B$ meson rest frame (\EB), quoted in the corresponding papers, are also provided.
     The branching fractions are extrapolated to 1.6~\gev, using extrapolation factors calculated in Ref.~\cite{Buchmuller:2005zv}.
     The Belle$^{\dagger}$ measurement was not published or used in the averages but is included here as the lepton-tagged measurement with the largest data sample.
     }
     \resizebox{1.\textwidth}{!}{
\begin{tabular}{llllll}

Year                       & Experiment                 & Technique          & Data used   & $\mathcal{B}(B\rightarrow X_s\gamma) \times 10^{-4}$ & Energy threshold\\ 
\hline
2007 & BaBar \cite{BaBar:2007yhb} & Hadronic-tagged           & 210~\invfb & $3.66 \pm 0.85 (\mathrm{stat.}) \pm 0.60 (\mathrm{syst.})$ & $\EB>1.9~\gev$ \\ 
2009 & Belle \cite{Belle:2009nth} & Untagged/lepton-tagged & 605~\invfb & $3.45 \pm 0.15  (\mathrm{stat.}) \pm 0.40 (\mathrm{syst.})$  & $E^B_{\gamma}>1.7~\gev$ \\ 
2012 & BaBar \cite{BaBar:2012fqh} & lepton-tagged                 & 347~\invfb & $3.21 \pm 0.15 (\mathrm{stat.})\pm 0.29 (\mathrm{syst.})$    & $E^B_{\gamma}>1.7~\gev$                            \\ 
2012 & BaBar \cite{BaBar:2012eja} & Sum-of-exclusive       & 429~\invfb & $3.29 \pm 0.19 (\mathrm{stat.}) \pm 0.48 (\mathrm{syst.})$ &  $E^B_{\gamma}>1.7~\gev$                            \\ 
2014 & Belle \cite{Belle:2014nmp} & Sum-of-exclusive       & 711~\invfb & & \\
2016 & Belle \cite{Belle:2016ufb} & lepton-tagged               & 711 fb$^{-1}$ & $3.12 \pm0.10 (\mathrm{stat.})\pm0.19(\mathrm{syst.})$ &              $E^B_{\gamma}>1.6~\gev$                 \\
%\hline
%2022 & {\bf Belle II}& Hadronic               & 189 fb$^{-1}$ & $3.54 \pm0.78 (\mathrm{stat.})\pm0.83(\mathrm{syst.})$ &              $E^B_{\gamma}>1.8~\gev$                 \\
\end{tabular}
}
 \end{table}
 }

\subsection{Sum-of-exclusive technique}\label{sec:sum_of_exclusive}

The sum-of-exclusive measurement technique embodies the idea of reconstructing all $X_s$ states separately and summing them up into an inclusive spectrum.
In practice, this is, of course, impossible; previous BaBar and Belle analyses (\cite{BaBar:2012eja,Belle:2014nmp}) reconstruct a sum of 38 exclusive channels that amount to roughly 70\% of the total \BtoXsgamma decay width.
The selected final states include various combinations of one or multiple $K^{\pm}$, $K_S^0$, $\pi^{\pm}$, $\piz$, $\eta$.
Modes with up to three kaons, four pions and one $\eta$-meson are considered.


A significant challenge of the method is the proper treatment of \BtoXsgamma events that have been incorrectly reconstructed in one of the 38 final states.
Photons originating in non-\BtoXsgamma decay chains, particularly decays of the type $B\to D^{(*)}\rho^+$ and non-$B$ events, also contribute significantly as background.
Much more than in the case of inclusive measurements, this technique strongly depends on the $X_s$ fragmentation modelling, which has to be carefully tuned and calibrated to represent the experimental data.
Finally, a measurement performed this way is only `pseudo-inclusive', meaning that additional uncertainties for unaccounted decay phase space are incurred.


The main advantages of this method are due to the precise knowledge of the $X_s$ system, which gives additional information about the $B$ meson, and a higher degree of control of background.
The knowledge of the charge and flavour of the decaying $B$ meson enables the measurement of the $\mathcal{CP}$ and isospin asymmetries \cite{BaBar:2014czi}.
Furthermore, it is the only inclusive measurement technique that has been able to distinguish $X_s$ and $X_d$ states experimentally \cite{BaBar:2010vgu}.
Direct reconstruction of the mass of the $X_s$ system, $M_{X_s}$, and decaying $B$ meson mass, $M_B$, allows expressing the photon energy directly in the signal $B$ meson rest frame:
\begin{equation}\label{eq:mx_egamma_relation}
    \EB = \frac{M_B^2 - M_{X_s}^2}{2M_B},
\end{equation}
which is otherwise only directly obtainable by hadronic-tagged inclusive measurements (see \Cref{sec:had_tagged_overview}).
Furthermore, the full reconstruction of the candidate $B$ meson allows using the well-defined initial state of the \epem collision for additional background suppression.
Two observables can be defined:
\begin{equation}\label{eq:deltae_inclusive}
    \Delta E \equiv E^*_B - \sqrt{s}/2,
\end{equation}
known as energy difference, expressed in terms of the energy of the $B$ meson in the collision centre-of-mass frame, $E^*_B$, and
\begin{equation}\label{eq:mbc_exclusive}
    M_{bc} \equiv \sqrt{(\sqrt{s}/2)^2 - (p^*_B)^2},
\end{equation}
known as the beam-constrained mass, expressed in terms of the momentum of the $B$ meson in the collision centre-of-mass frame, $p^*_B$.
From \Cref{eq:deltae_inclusive,eq:mbc_exclusive} it is clear that $B$ candidates which are reconstructed correctly tend to have a resonant behaviour in \Mbc and \DeltaE, with their distributions peaking at the nominal $B$ mass $\approx5.28~\gevcc$ and $0$, respectively.
The backgrounds tend to have broader or even non-peaking shapes. 
An \Mbc distribution is shown in \Cref{fig:mbc_sum_of_exclusive}, as measured by the Belle sum-of-exclusive measurement \cite{Belle:2014nmp}.
In the past, these analysis techniques achieved an average signal reconstruction efficiency of 3.5\%~(larger for greater values of \EB), after correcting for missing $X_s$ modes \cite{Belle:2014nmp}.

\begin{figure}[htbp!]
    \centering
    \includegraphics[width=0.4\textwidth]{figures/experiment_overview/Mbc_sum_exclusive_Belle.png}
    \caption{\label{fig:mbc_sum_of_exclusive} 
    The distribution of \Mbc, as seen in the $1.9>M_{X_s}>1.8~\gevcc$ interval by Ref.~\cite{Belle:2014nmp} in the sum-of-exclusive state analysis.
    The data points are fitted in an unbinned maximum likelihood fit with a combination of fit functions for 
    signal events (red, thick and short dashed), 
    cross-feed (red, thin and short dashed), 
    peaking $\BB$ (green, thick and long dashed), 
    non-peaking $\BB$ (green, thin and long dashed)
    and $\qqbar$ background events (blue, dash-dotted).
    }
\end{figure}

\subsection{Untagged technique}\label{sec:untagged}
The untagged inclusive measurements are the simplest conceptually as they only require a high-energy photon in the final state, without any explicit assumptions about the $X_s$ or partner $B$ meson decay.
Such a simple requirement guarantees that photons from all $\BtoXsgamma$ decays are included in the selected data sample.
Because any partner $B$ meson state is accepted, they are sometimes also called \textit{inclusive-tagged} or \textit{fully inclusive}.
For the rest of the thesis \textit{untagged} is used to stress the difference between the signal $B$ decay and the partner $B$.

Although the concept of this approach is simple, the measurement is highly challenging experimentally.
In particle collisions and subsequent decay processes, high-energy photons can originate through numerous ways, such as initial \epem state radiation, \epem\ra\qqbar processes, $B$ decays etc.
In the previously introduced sum-of-exclusive technique this problem is solved by using the information of decays of $X_s$ and using observables such as \Mbc and \DeltaE to suppress incorrect photon candidates. 
Conversely, in the inclusive case, $X_s$ is treated as unreconstructed, and therefore a background suppression procedure is performed such that no selection bias is introduced to the $X_s$ system.
An example spectrum based on Belle~II simulation is shown in \Cref{fig:untagged_btosgamma_background}, which highlights the signal-to-background difference.

The predominant background originates from events where no $B$ mesons are created.
In $\epem\to\qqbar$ events ($q\in\{u,d,s,c\}$), where a \piz is created through hadronisation or decays of other hadrons.
The \piz then decays asymmetrically into two photons, which mimics the isolated high-energy photon of \BtoXsgamma decays.
In particular, $q={u,c}$ processes contribute strongly, since they tend to produce energetic \piz or charm mesons (whose decay chains include \piz).
Semileptonic and hadronic $B$ decays that produce \piz, $\eta$ and $\rho$ mesons, are also among the main sources of background contribution.

The usual way to perform this analysis involves using boosted decision trees or other multivariate methods (see \Cref{sec:classification}) to suppress highly prominent backgrounds.
For example, the technique applied in Refs.~\cite{CLEO:2001gsa,Belle:2009nth} combines high-energy photons with all the other photons in the event and vetoes those that are compatible with \piz or similar decays.
The contributions from \epem\ra\qqbar are suppressed by parametrising the different decay topologies that are observed for \BB and \qqbar events (see \Cref{sec:continuum_suppression}).
Data samples collected below the \FourS resonance, containing only $\epem\ra\qqbar$ events, are used to subtract the non-\BB contributions remaining after the background treatment.
The \BB background contribution is usually removed using simulation.
An example result of the extracted \BtoXsgamma spectrum from Belle~II data using $63.1~\invfb$, which is a work that I was involved in during the doctoral research~\cite{Collaboration:2302}, is shown in \Cref{fig:untagged_btosgamma_measured}.

The strength of the untagged technique is the `truly' inclusive approach, which ensures that all $X_s$ states are selected, as well as a large signal reconstruction efficiency.
Previous analyses (e.g. Ref.~\cite{Belle:2009nth}) report an average selection efficiency in their final sample of $\sim10\%$ (increasing with photon energy), which is several times higher than the sum-of-exclusive approach.
On the other hand, the missing kinematic information of the $X_s$ system yields a complicated and inefficient background suppression process, which means that such measurements have a small signal-to-background ratio.
Moreover, the kinematic information of the $B$ decay cannot be accessed, which only allows a measurement of the photon energy in the \epem collision frame.
To reach the theoretically more desirable $B$ meson decay frame, additional modelling uncertainties have to be introduced.

\begin{figure}[htbp!]
    \centering
    \subcaptionbox{\label{fig:untagged_btosgamma_background}}{
        \includegraphics[width=0.45\textwidth]{figures/experiment_overview/untagged_background.pdf}
        }
    \subcaptionbox{\label{fig:untagged_btosgamma_measured}}{
    \includegraphics[width=0.45\textwidth]{figures/experiment_overview/untagged_btosgamma.pdf}
    }
    \caption{\label{fig:untagged_btosgamma}
    Photon energy spectra in \BtoXsgamma decays before (\Cref{fig:untagged_btosgamma_background}) and after (\Cref{fig:untagged_btosgamma_measured}) background suppression.
    Before the background suppression is performed, the signal fraction is roughly 1000 times smaller than the background.
    After background suppression, subtracting the remaining continuum (\epem\ra\qqbar decays) and \BB background yields the \BtoXsgamma spectrum (bottom panel).
    Both Figures show Belle II simulated data.
    \Cref{fig:untagged_btosgamma_measured} also includes official Belle II data from Ref.~\cite{Collaboration:2302}.
    \Cref{fig:untagged_btosgamma_background} was produced for illustrative purposes only. 
    }
\end{figure}

\subsection{Tagged techniques}\label{sec:had_tagged_overview}

To overcome multiple of the issues that come with untagged measurements presented in \Cref{sec:untagged} 
while still selecting an inclusive $X_s$ sample (conversely to \Cref{sec:sum_of_exclusive}), 
additional information about the second $B$ meson from the \FourS decay can be used.
The naming stems from the similarity to the \textit{tag-and-probe} methods.
In this approach, the partnering $B$ meson is fully reconstructed, or some of its decay products are identified.
This $B$ meson will henceforth be referred to as the \textit{tag}-$B$ meson.
The application of kinematic constraints on the event arising from the tag-$B$ is called \textit{tagging}.
The schematic idea of tagging is shown in \Cref{fig:tagging_schematic}.
Three main tagging techniques have been used in the past at $B$ factories:
\begin{itemize}
    \item \textit{lepton-tagging}, where a lepton originating from tag-$B$ decays is reconstructed;
    \item \textit{semileptonic-tagging}, where the tag-$B$ is reconstructed as a semileptonic $B$ decay of the form $B\to D^{(*)}\ell\bar{\nu}$;
    \item \textit{hadronic-tagging}, where the tag-$B$ is reconstructed as a decay that involves the final states $B\to\mathrm{hadrons}$, such as $B\to K\pi$.
\end{itemize}
The main advantages of these techniques are summarised in \Cref{fig:tagging_advantages}.

\begin{figure}[htbp!]
    \centering
    \subcaptionbox{\label{fig:tagging_schematic}}{
        \resizebox{0.25\textwidth}{!}{
            \input{tikz_diagrams/tagging.tex}
            }
    }
    \subcaptionbox{\label{fig:tagging_advantages}}{
    \includegraphics[width=0.7\textwidth]{figures/experiment_overview/tagging_advantages.png}
    }
    \caption{\label{fig:tagging_drawins} Schematic representation of tagging is shown in \Cref{fig:tagging_schematic}.
    The idea of tagging is using the tag-side $B$ decay products to apply kinematic constraints on the signal-side $B$ decay (\BtoXsgamma in the example).
    \Cref{fig:tagging_advantages} highlights the advantages and disadvantages related to using different $B$ decay products for tagging at $B$ factories.
    A detailed discussion of these techniques is given in the text.
    Credit to Dr. Markus Röhrken for \Cref{fig:tagging_advantages}.}
\end{figure}

Leptonic tagging has been used as a `successor' method for \BtoXsgamma untagged analyses by BaBar and Belle \cite{Belle:2009nth,BaBar:2012fqh,Belle:2016ufb}, providing a higher degree of background control while still retaining a larger efficiency.
In the past, lepton-tagged analyses achieved an average signal reconstruction efficiency of up to $3\%$ (increasing with photon energy).
Going a step further and reconstructing the charm meson and the lepton from the semileptonic $B$ decay gives an even higher degree of background control.
Measuring the angle between the reconstructed $D\ell$ system and the decaying $B$ meson, for example, provides excellent background suppression \cite{BaBar:2014omp}.
A major complication is the fact that the reconstruction of the semileptonic decay chain necessarily reduces the overall efficiency.
$B$ mesons have large semileptonic decay rates, meaning that the lower efficiency due to $D$ reconstruction is partially offset by the large statistical samples.
However, the presence of a neutrino in the final state complicates the technique further.
On average, this technique is at least an order of magnitude less efficient than the untagged approach and several times less efficient than the lepton-tagged method \cite{Belle-II:2018jsg}.
Despite successful application for missing energy modes, e.g., $B\rightarrow K^+\nu\nu$ \cite{BaBar:2009qvi}, semileptonic tagging was never used for \BtoXsgamma.

Of particular interest is hadronic tagging, which, in the context of \BtoXsgamma, has been performed only once by BaBar \cite{BaBar:2007yhb}, using roughly 50\% of their total dataset.
Compared to other tagged and untagged inclusive techniques, this is the only method which fully reconstructs the kinematics of the tag-$B$ due to the absence of neutrinos in the final state.
As a result, with the beam constraint requirements, one can calculate quantities such as \Mbc and \DeltaE (see \Cref{eq:mbc_exclusive,eq:deltae_inclusive}) for the tag-side $B$ meson.
The ability to rely on these distributions and, in particular, perform a signal extraction fit, similar to the one given in \Cref{fig:mbc_sum_of_exclusive}, allows suppressing previously dominant $\epem\ra\qqbar$ component to negligible levels.

Because both $B$ mesons at $B$ factories are created from a \FourS decay, 
the full knowledge of the tag-$B$ properties allows inferring the charge, momentum and flavour of the signal-$B$ meson, and consequentially, measuring the desired observables in the decaying-$B$ rest frame.
A mathematical description of a Lorentz transformation using tag-$B$ constraints is provided in \Cref{sec:appendix_boosting_to_b_frame}.
Therefore, one regains all the benefits that the sum-of-exclusive technique offers, while still ensuring that no selection requirements are imposed on the $X_s$ system.
However, a complication that follows is the fact that hadrons can have thousands of decay chains and an efficient reconstruction of a statistically significant tag-$B$ sample is complicated.
Compared to semileptonic tagging, the hadronic tagging technique has a several times lower efficiency, although this is compensated by a very high purity of the tagged data sample \cite{Belle-II:2018jsg}.
The BaBar analysis achieved a signal efficiency of $\mbox{\sim 0.2\%}$ depending on the \EB interval (increasing with photon energy).


Hadronic-tagged measurements of \BtoXsgamma have uncertainties that are mainly related to the modelling of \BB background, \Mbc fitting and correlation between tag-side decays and signal-side reconstruction.
The previous hadronic-tagged BaBar measurement \cite{BaBar:2007yhb} achieved a $16\%$ systematic uncertainty and a $23\%$ statistical uncertainty.
Both uncertainties are expected to be improved with larger data samples.
As evident from \Cref{tab:btosgamma_inclusive_summary}, historically, sum-of-exclusive and lepton-tagged methods have been the most precise measurements of the \BtoXsgamma spectrum.
The uncertainty of the hadronic-tagged measurement is higher but comparable to that of untagged, leptonic-tagged and sum-of-exclusive measurements, even though the hadronic-tagged analysis has been performed with only half of available BaBar data.
The hadronic-tagged technique can therefore provide one of the world's most accurate measurements with an increased data sample \cite{Belle-II:2022cgf}.

Moreover, because the hadronic-tagged analysis incurs different systematic uncertainties and relies less on simulation, it is a powerful cross-check of the other tagged analyses.
It is also important to note that these techniques produce samples that are not highly correlated due to different backgrounds specific to the analysis procedure (see e.g. Ref.~\cite{Belle:2009nth}).
Therefore, different tagging techniques \textit{complement} but do not \textit{compete with} each other.





