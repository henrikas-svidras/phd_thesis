\chapter{Experimental overview of inclusive rare radiative decays}

There are two main types of analyses which inherit the meaning from their theoretical description presented in \Cref{ch:theory}: \textit{inclusive} analyses and \textit{exclusive} analyses.
Another method, which is a mixture of the two methods, is known as \textit{sum-of-exclusive} method.
Theoeretical motivations for inclusive measurements in terms of \BtoXsgamma have already been discussed in \Cref{sec:btosgamma_totalrate_theory,sec:btosgamma_spectrum_theory}.
Similar motivations hold true for other electroweak decay channels such as, $B\to X_s\ell\bar{\ell}$ ($\ell\in\{\mu^-,e^-,\nu\}$).
However, compared to the latter, $B\rightarrow X_s\gamma$ contains one less interaction vertex and therefore is enhanced by $\sim\alpha_{\mathrm{em}}$, which makes it experimentally acessible with smaller datasets.
In this chapter, I will introduce the main methods of performing an inclusive measurement, while focusing solely on \BtoXsgamma, although, generally, these methods are applicable for other decay channels as well.

\section{Past measurements of \safeBtoXsgamma decays}

Inclusive measurements target to measure a wide selection of decay products; in the case of $X_{sd}$, the sum of all states that originate in $b\to s$ or $b\to d$ transitions.
In principle, such states include resonant and non-resonant states.
A notable resonant state is $B\rightarrow \Kstar(892)\gamma$.
Experimentally it is highly-accessible due to its narrow and isolated peak near the $b\to s\g$ two-body decay kinematic limit.
A similar state for $b\rightarrow d\g$ is the $B\rightarrow \rho(770)\gamma$.
Non-resonant states include combinations of one or more pions and, in case of $b\rightarrow s$, kaon.
While the goal of an inclusive measurement is to measure the energy of photons from all these decays simultaneously, an exclusive measurement attempts to select one or several particular states from the spectrum.

The world-average values of inclusive radiative $B$ decay measurements are~\cite{Amhis:2022mac,Workman:2022ynf}:
\begin{align}\label{eq:btosgamma_experimental}
    \begin{split}
    \mathcal{B}(\BtoXsgamma)=(3.49\pm 0.19)\times10^{-4}\\
    \mathcal{B}(\BtoXdgamma)=(0.09\pm 0.03)\times10^{-4}.
    \end{split}
\end{align}
The values of \Cref{eq:btosgamma_experimental} can be compared to the theoretical prediction in \Cref{eq:btosgamma_theoretical,eq:btodgamma_theoretical}.
\BtoXsgamma results show an excellent agreement between the theory and experiment.
For \BtoXdgamma, values agree within $2\sigma$, however, the experimental uncertainty is dominated statistically and performed using the \textit{sum-of-exclusive} approach, which may explain the lower value due to unaccounted $b\rightarrow d\gamma$ modes.

\Cref{tab:btosgamma_bfs} provides the experimental status of observed most prominent \BtoXsdgamma decay channels as of December 2022.
Due to final-state similarity and overlap between various resonant and non-resonant decay modes, only the most prominent resonant decays have been measured.
However, even in the case of a relatively islolated decay channel such as \BtoKstargamma, the overall precision of inclusive measurements is higher (compare to \Cref{eq:btosgamma_experimental}).

{\renewcommand{\arraystretch}{1.2}
\begin{table}[!htbp]
    \centering
    \caption{\label{tab:btosgamma_bfs} 
    Branching fractions of \BtoXsgamma modes for charged and neutral modes.
    The table only includes decay modes that have been observed and (for $\BtoXsgamma$ only) have a branching fraction $\gtrsim10^{-5}$.
    The \Bp decays are ordered in terms of the experimental precision $\mathcal{B}/\Delta\mathcal{B}$, whereas \Bz are ordered in relation to \Bp, where applicable.
    The values correspond to the averages of experimental measurements given in Refs. \cite{Amhis:2022mac,Workman:2022ynf}.
    }
    \begin{minipage}[c]{0.5\textwidth}
    \centering
    \BptoXsgamma exclusive modes
    \resizebox{1\textwidth}{!}{
    \begin{tabular}{l|c}
        Decay mode & Branching fraction ($\times 10^{-4}$) \\
        \hline
        \multicolumn{2}{c}{\textbf{Two decay products}}\\
        $\Bp \to \Kstar(892)^+ \gamma$  & $0.392 \pm 0.022$ \\
        $\Bp \to K_1(1270)^+  \gamma$   & $0.438 \pm ^{0.071}_{0.063}$ \\
        $\Bp \to K_2^*(1430)^+  \gamma$ & $0.138 \pm 0.040$ \\            
        $\Bp \to K^*(1410)^+  \gamma$   & $0.271 \pm ^{0.080}_{0.061}$ \\
        $\Bp \to K^*(1680)^+  \gamma$   & $0.670 \pm ^{0.170}_{0.140}$ \\
        $\Bp \to K_1(1400)^+  \gamma$   & $0.097 \pm ^{0.054}_{0.038}$ \\
        \multicolumn{2}{c}{\textbf{Three or more decay products}}\\
        $\Bp \to \Kstar(892)^0 \pi^+ \gamma$ & $0.233 \pm 0.012$ \\
        $\Bp \to K^+ \pi^+\pi^- \gamma$ & $0.258 \pm 0.015$ \\
        $\Bp \to K^0 \pi^+\pi^0 \gamma$ & $0.456 \pm 0.052$ \\
        $\Bp \to K^+ \pi^+\pi^- \gamma$ (non-resonant) & $0.099 \pm ^{0.017}_{0.020}$ \\
    \end{tabular}
    }
\end{minipage}
\begin{minipage}[c]{0.395\textwidth}
    \centering
    \BztoXsgamma exclusive modes
    \resizebox{1\textwidth}{!}{
    \begin{tabular}{l|c}
        Decay mode & Branching fraction ($\times 10^{-4}$) \\
        \hline
        \multicolumn{2}{c}{\textbf{Two decay products}}\\
        $\Bz \to \Kstar(892)^0 \gamma$ & $0.418 \pm 0.025$ \\
        - & -\\
        $\Bz \to K_2^*(1430)^0 \gamma$ & $0.124 \pm 0.024$ \\ 
        - & -\\
        - & -\\
        - & -\\
        \multicolumn{2}{c}{\textbf{Three or more decay products}}\\
        - & -\\
        $\Bz \to K^+ \pi^-\pi^0 \gamma$ & $0.407 \pm 0.038$\\
        $\Bz \to K^0 \pi^+\pi^- \gamma$ & $0.199 \pm 0.018$\\
        - & -\\
    \end{tabular}
    }
\end{minipage}

\vspace{10pt}

\begin{minipage}[c]{0.395\textwidth}
    \centering
    $B\to X_d\gamma$ exclusive modes
    \resizebox{1\textwidth}{!}{
    \begin{tabular}{l|c}
        Decay mode & Branching fraction ($\times 10^{-4}$) \\
        \hline
        $\Bp \to \rho^+(770)\gamma$ & $0.0098 \pm ^{0.0025}_{0.0024}$\\
        $\Bz \to \rho^0(770)\gamma$ & $0.0086 \pm 0.0015$\\
        $\Bz \to \omega(782)\gamma$ & $0.0044 \pm ^{0.0018}_{0.0016}$\\
    \end{tabular}
    }
\end{minipage}

\end{table}
}

Currently, inclusive measurements are only attainable at $B$ factories.
The relatively low-background environment offered by an $\epem$ collision allows to treat the $X_{sd}$ system as a `missing-energy' system with no explicit requirements.
Although the process of event reconstruction and detection by itself may introduce a bias to the inclusive system, this is expected to be a much smaller effect than other experimental factors, such as resolution and poissonian fluctations.
Conversely, at hadron colliders, large probability for multiple proton pairs interacting in a collision event creates a large QCD background, usually referred to as \textit{pileup}.
This makes it complicated to select an unbiased and model independent inclusive sample and,
at the time of writing this, no inclusive $B$ measurement has been performed outside of a $B$-factory experiment.

Even in the case of exclusive radiative measurements, such as $B\rightarrow \Kstar\gamma$, $B$-factory have historically outperformed hadron collider experiments (such as LHCb) due to a higher-degree of control of combinatorial background.
Final states that include neutral particles and photons are problematic to measure accurately in high pileup conditions.
Therefore, with several exceptions (e.g. \cite{Bellee:2019qbt}), the field of rare radiative $B$ decay measurements is dominantly probed by $B$-factory experiments.



\section{Techniques for inclusive \safeBtoXsgamma measurements}
 Historically, three different techniques were applied for inclusive \BtoXsgamma analyses at $B$ factories: 
 sum-of-exclusive measurements,
 untagged-inclusive measurements,
 tagged-inclusive measurements.
 The summary of the most-precise measurements that are used in experimental average in \Cref{eq:btosgamma_experimental} is given in \Cref{tab:btosgamma_inclusive_summary}.

 {\renewcommand{\arraystretch}{1.2}
 \begin{table}[!htbp]
     \centering
     \caption{\label{tab:btosgamma_inclusive_summary} 
     \todo[inline]{updated branching fraction}
     }
     \resizebox{1.\textwidth}{!}{
\begin{tabular}{llllll}

Year                       & Experiment                 & Technique          & Data used   & $\mathcal{B}(B\rightarrow X_s\gamma) \times 10^{-4}$ & Energy threshold\\ 
\hline
2007 & BaBar \cite{BaBar:2007yhb} & Hadronic-tagged           & 210~\invfb & $3.66 \pm 0.85 (\mathrm{stat.}) \pm 0.60 (\mathrm{syst.})$ & $\EB>1.9~\gev$ \\ 
2009 & Belle \cite{Belle:2009nth} & Untagged/lepton-tagged & 605~\invfb & $3.45 \pm 0.15  (\mathrm{stat.}) \pm 0.40 (\mathrm{syst.})$  & $E^B_{\gamma}>1.7~\gev$ \\ 
2012 & BaBar \cite{BaBar:2012fqh} & lepton-tagged                 & 347~\invfb & $3.21 \pm 0.15 (\mathrm{stat.})\pm 0.29 (\mathrm{syst.})$    & $E^B_{\gamma}>1.7~\gev$                            \\ 
2012 & BaBar \cite{BaBar:2012eja} & Sum-of-exclusive       & 429~\invfb & $3.29 \pm 0.19 (\mathrm{stat.}) \pm 0.48 (\mathrm{syst.})$ &  $E^B_{\gamma}>1.7~\gev$                            \\ 
2014 & Belle \cite{Belle:2014nmp} & Sum-of-exclusive       & 711~\invfb & & \\
2016 & Belle \cite{Belle:2016ufb} & lepton-tagged               & 711 fb$^{-1}$ & $3.12 \pm0.10 (\mathrm{stat.})\pm0.19(\mathrm{syst.})$ &              $E^B_{\gamma}>1.6~\gev$                 \\
%\hline
%2022 & {\bf Belle II}& Hadronic               & 189 fb$^{-1}$ & $3.54 \pm0.78 (\mathrm{stat.})\pm0.83(\mathrm{syst.})$ &              $E^B_{\gamma}>1.8~\gev$                 \\
\end{tabular}
}
 \end{table}
 }