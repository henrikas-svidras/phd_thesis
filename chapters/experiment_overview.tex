\chapter{Experimental overview of inclusive radiative decays}

There are two main types of analyses which inherit the meaning from their theoretical description presented in \Cref{ch:theory}: \textit{inclusive} analyses and \textit{exclusive} analyses.
Another method, which is a mixture of the two methodsm, is known as \textit{sum-of-exclusive} method.

The inclusive measurements target to measure a wide selection of final states; in the case of $X_{sd}$, any states that originate in $b\to s$ or $b\to d$ transitions.
In principle, such final states include resonant and non-resonant decays, e.g., $B\rightarrow \Kstar(892)\gamma$.
Experimentally it is highly-accessible due to its narrow and isolated peak near the $b\to s\g$ two-body decay kinematic limit.
A similar state for $b\rightarrow d\g$ is the $B\rightarrow \rho\gamma$.
Non-resonant states include combinations of multiple pions and, in case of $b\rightarrow s$, kaon.
While the goal of an inclusive measurement is to measure the energy of photons from all these decays simultaneously, an exclusive measurement attempts to select one or several particular states from the spectrum.

Currently, inclusive measurements are only attainable at $B$ factories.
The relatively low-background environment offered by an $\epem$ collision allows to treat the $X_{sd}$ system as a `missing-energy' system with no explicit requirements.
Although the process of event reconstruction and detection by itself may introduce a bias to the inclusive system, this is expected to be a much smaller effect than other experimental factors, such as resolution and poissonian fluctations.
Conversely, at hadron colliders, large probability for multiple proton pairs interacting in a collision event creates a large QCD background, usually referred to as \textit{pileup}.
This makes it complicated to select an unbiased and model independent inclusive sample and,
at the time of writing this, no inclusive $B$ measurement has been performed outside of a $B$-factory experiment.

Even in the case of exclusive radiative measurements, such as $B\rightarrow \Kstar\gamma$, $B$-factory have historically outperformed hadron collider experiments (such as LHCb) due to a higher-degree of control of combinatorial background.
Due to their superior particle identification subsystems, LHCb had an edge in channels where precise particle information is otherwise a leading source of uncertainty, for example states with many charged particles
The table XX provides the experimental status of \BtoXsgamma decays as of November 2022.
\todo[inline]{table experimental results here}

In this chapter, I will introduce the main methods of performing an inclusive measurement, while focusing solely on \BtoXsgamma.
Other decay channels, such as, $b\to s\ell\ell$ ($\ell=\{\ell^-,\nu\}$)

\section{Experimental methods for inclusive measurements}
test

\section{Status of experimental measurements}
Currently, inclusive measurements are only attainable at $B$ factories.
At hadron colliders, large probability for multiple proton-collisions