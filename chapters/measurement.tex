\chapter{Measurement of \safeBtoXsgamma with hadronic-tagging}

So far the thesis laid out the theoretical foundation which motivates the study of \BtoXsgamma 
and experimental techniques which allow to perform this measurement.
In this Chapter, the two will be connected, describing the main topic of the PhD thesis: 
a measurement of the photon-energy spectrum of \BtoXsgamma decays using the hadronic-tagging technique,
with the Belle II data.
This measurement is the first such measurement in Belle II, and generally, since the previously discussed BaBar result \cite{BaBar:2007yhb}.
It sets up the experimental procedure for the Belle II experiments following this analysis technique in the future.

The measurement is performed in several (interconnected) steps, which are broken down in REFHERE\todo{ref here}.
The implication of the results and the outlook for the analysis will be discussed in the following chapter CHAPTER NAME.


\section{Analysis strategy}\label{sec:analysis_strategy}
This Section serves as an overview of the analysis
and is intended to guide the reader through different steps in extracting the \BtoXsgamma photon energy spectrum.
It also introduces key terminology that is used throughout the entire text.
In this entire Chapter, superscripts are used to denote observables in a particular frame of reference.
Given an observable $\mathcal{X}$ (laboratory frame), the same value in the frame of colliding electrons (centre-of-mass) is denoted as $\mathcal{X}^*$.
Observables in the rest frame of the decaying $B$ meson adopt the notation $\mathcal{X}^B$.

The final goal of the analysis is to measure the partial branching fractions (see \Cref{eq:branching_fraction_definition})
as a function of \EB (i.e. the photon energy spectrum in the signal $B$ meson rest frame):
\begin{equation}\label{eq:branching_fraction_recipe}
    \left.\frac{d\mathcal{B}(\BtoXsgamma)}{d\EB}\right|_i = \mathcal{U}_i \cdot \frac{N_i(\BtoXsgamma)}{\varepsilon_i N_{B}},
\end{equation}
where $i$ is a given \EB interval,
$N_i(\BtoXsgamma)$ is the number of $B$ mesons measured in the the interval $i$, 
$\varepsilon_i$ is the average efficiency for selection and reconstruction of \BtoXsgamma decays in the interval $i$,
$N_B$ is the total number of $B$ mesons in the analysed sample,
and $\mathcal{U}_i$ are unfolding correction factors (bin-by-bin unfolding is implied here).
The results of the integrated branching fraction, $\mathcal{B}(\BtoXsgamma)$, are evaluated by performing a sum over partial branching fractions computed in every interval $i$.

The analysed data sets are introduced in \Cref{sec:data_samples}.
A simulated data set, based on the expectations of the Standard Model, is used to prepare the analysis procedure.
Only after the full analysis procedure is set up, the results on the Belle~II data will be unblinded (see \Cref{sec:blinding}).

\Cref{sec:event_reconstruction} introduces the reconstruction of Belle~II experimental and simulated \BtoXsgamma samples.
It also discusses the main processes which mimic \BtoXsgamma signal (backgrounds) and the strategies to suppress them.
The background sources for \BtoXsgamma can be divided as:
\begin{itemize}
    \item \textit{Signal-side background}: where a photon candidate is originating in a non-\BtoXsgamma decay.
    In particular, either \epem\ra\qqbar ($\q\in\{\u,\d,\s,\c\}$), henceforth -- \textit{continuum}, or from a different \B decay, e.g. where a $\piz\to\g\g$ is created in the decay chain, henceforth -- \textit{generic} $B$ background;
    \item \textit{Tag-side background}: where the $B$, recoiling against the candidate \BtoXsgamma, is reconstructed incorrectly, or \epem\ra\qqbar collision products are combined to resemble a \B decay.
    Such decays are referred to as \textit{combinatorial} $B$ or continuum background, respectively;
    \item \BtoXdgamma component: which is an irreducible background at the current analysis setup. 
\end{itemize}

The
multivariate optimisation strategies for signal and tag-$B$ meson background suppression are described in 
\Cref{sec:photon_selection,sec:continuum_suppression,sec:final_optimisation,sec:tag_selection}.
They rely on selections to suppress the signal-side background and a dedicated \BDT training which aims to suppress tag-side backgrounds.

After the full background suppression procedure, a thorough setup of the fitting model is described in \Cref{sec:fitting_mbc}.
The fitting procedure is aimed at removing the combinatorial tag-$B$ meson backgrounds, and hence the fit of the \Mbc variable is performed (\Cref{eq:mbc_exclusive}).

Lastly, an irreducible signal-side background component remains, particularly from generic signal-$B$ meson decays.
The \EB spectrum is extracted by the simulation-dependant subtraction of remaining background processes as described in \Cref{sec:background_subtraction}.

\Cref{sec:MC_validation} explains the validation procedure of the analysis technique on Belle~II simulated samples,
whereas \Cref{sec:corrections,sec:validation,sec:signal_modelling} explore that on the Belle~II recorded data samples.
\Cref{sec:systematic_uncertainty} condenses the observations from the validation and quantitively assigns systematic uncertainties.
The unblinding and the final extraction of results and unfolding of Belle~II data is presented in \Cref{sec:results}.


\section{Data samples}\label{sec:data_samples}

\subsection{Experimental data sets}\label{sec:data}

This measurement uses data sets of \epem collisions produced by the SuperKEKB accelerator and collected by the Belle II detector in 2019-2021.
There are two data-collection modes in Belle II:
\begin{itemize}
    \item \textit{on-resonance data}: data sets collected at the collision energy $\sqrt{s}\approx=10.58~\gev$, corresponding to the mass of \FourS meson;
    \item \textit{off-resonance data}: data sets collected 60~\mev below the on-resonance threshold. 
    Such data, by definition, does not contain $\BB$ events and is an excellent testing and validation sample to understand continuum processes.
\end{itemize}
The integrated luminosity, henceforth denoted as $\int~\hspace{-7pt}~\mathcal{L}$, corresponding to the on(off)-resonance sample is 189(18)~\invfb.
The on-resonance data set contains approximately $198$ million \BB pairs.

\subsection{Simulated data sets}\label{sec:MC}

In order to prepare the analysis procedure, calculate the signal-selection efficiencies and perform a validation adhering to the principles of a blinded-analysis, 
large simulated samples are utilised.
These samples are significantly larger than the data sets that were anticipated to be analysed by this analysis to ensure that uncertainties due to limited simulation samples are minimal.
The overview of the samples is discussed in this subsection, but a quick overview is provided in \Cref{tab:simulated_samples}

\begin{table}[htbp!]
    \centering
\caption{\label{tab:simulated_samples}The overview of simulated samples used in the measurement described by this thesis.
More in-depth discussion for each sample is present in the text.}
\resizebox{0.7\textwidth}{!}{
\begin{tabular}{lcc}

Simulated sample & Size  & Generators used \\ 
\hline
Generic-\Bz & \multirow{6}{*}{1+0.6~\invab}& \multirow{2}{*}{\texttt{EvtGen} \cite{Ryd:2005zz} }\\
Generic-\Bp &                          &\\
\cline{3-3}
continuum \uubar &                     & \multirow{4}{*}{\texttt{KKMC} \cite{Ward:2002qq} interfaced to Pythia 8 \cite{Sjostrand:2007gs}}\\
continuum \ddbar &                     &\\
continuum \ccbar &                     &\\
continuum \ssbar &                     &\\
\cline{2-3}
\BptoXsgamma      & \multirow{2}{*}{100 million}  & \multirow{2}{*}{\texttt{EvtGen}, \texttt{BTOXSGAMMA} model \cite{Ryd:2005zz}}\\
\BztoXsgamma      & & \\
\cline{2-3}
\Bp\to\Kstarp(782)\g    & \multirow{2}{*}{10 million} & \multirow{2}{*}{\texttt{EvtGen}, \texttt{SVP\_HELAMP} model \cite{Ryd:2005zz}}\\
\Bz\to\Kstarz(782)\g    &  & \\
\end{tabular}
}
\end{table}

All the simulated samples correspond to the 14th official Belle II simulation production campaign, and are based on Monte-Carlo methods.
Therefore, I will henceforth refer to the simulated samples as \MC.
For the analysis I use:
\begin{itemize}
    \item four \epem\ra\qqbar ($\q\in\{\u,\d,\s,\c\}$) simulated sample categories, referred to as continuum \MC,
    \item two \FourS\ra\BB categories for charged and neutral \B modes, referred to as generic-$B$ \MC.
\end{itemize}
Altogether, the above to categories are referred to as \textit{generic} \MC.
The analysis will be performed on 1~\invab of simulation, which is more than 5 times larger than the on-resonance data set for this analysis.
Furthermore, for the background subtraction step described in XXXX, which has the strongest dependance on limited-\MC sample size, I use an even larger 1.7~\invab dataset, utilising the total available Belle~II \MC dataset.
\todo[inline]{described where?}

The generic-\B \MC includes \BtoXsgamma decays, however the number of such events is expected to be small, further diminished by the efficiency of the hadronic-tagged analysis procedure, as discussed in \Cref{sec:had_tagged_overview}.
For this reason we utilise additional samples:
\begin{itemize}
    \item 100 million \BpBm sample, where at least one $B$ is guaranteed to decay as \BptoXsgamma based on the Kagan-Neubert model \cite{Kagan:1998ym} (see \Cref{sec:btosgamma_spectrum_theory});
    \item 100 million \BzBzb sample, where at least one $B$ is guaranteed to decay as \BztoXsgamma based on the Kagan-Neubert model \cite{Kagan:1998ym} (see \Cref{sec:btosgamma_spectrum_theory});
    \item 10 million \BpBm sample, where at least one $B$ is guaranteed to decay as \Bpm\ra\Kstarpm\gamma~based on the \texttt{EvtGen SVP\_HELAMP} model \cite{Ryd:2005zz};
    \item 10 million \BzBzb sample, where at least one $B$ is guaranteed to decay as \Bz\ra\Kstarz\gamma~based on the \texttt{EvtGen HELAMP} model \cite{Ryd:2005zz}.
\end{itemize}
The first two are used for general analysis setup in sections XXXX.
The two latter will be combined in with \BtoXsgamma in a hybrid-model approach \cite{Ramirez:1989yk} discussed in XXX.
\todo[inline]{discussed where? I need to decide myself where this goes. Maybe even here, right now.}

In all cases, the detector response and readout is simulated using Geant 4 \cite{GEANT4:2002zbu}.

\epem\ra\tautau events are negligible.

\section{Event reconstruction}\label{sec:event_reconstruction}

The aim of the inclusive \BtoXsgamma analysis is to reconstruct an inclusive sample of all possible \Xs states, 
as described before (e.g. \Cref{ch:exp_overview}).
This means that explicit requirements on the momentum, number of tracks, angles etc. of the \Xs system may introduce a direct bias on the `inclusiveness' of the measurement.
Assessing the impact of such selections on $X_s$ in a model-independent way is difficult.
Therefore, the \Xs system is treated in a completely `missing-momentum' approach, such that no direct requirements on it are imposed.
The reconstruction requirements are only applied on the candidate tag-$B$ meson and the candidate high energy photons from \BtoXsgamma.
\subsection{Tag-\texorpdfstring{$B$}{B} meson candidate reconstruction}\label{sec:tag_reconstruction}

The analysis begins with the reconstruction of tag-$B$ meson candidates in each event
using the Belle~II Full Event Intepretation (\FEI) algorithm \cite{Keck:2017mui,Keck:2018lcd}, which is part of \basftwo.
It is a hierarchical six-stage reconstruction chain, which begins with the identification of all tracks, displaced vertices (tracks that do not originate near the interaction point) and \ECL clusters.
The algorithm begins by combining track and ECL cluster information to reconstruct final-state candidate particles,
such as $\epm$, $\mupm$, $\gamma$,  $\pi^{\pm}$, $K^{\pm}$ and $K_L^0$.
In the next stage, the final-state particles are combined to form intermediate particles, such as $\piz, K_S^0, D^{(*)}$.
In later stages, intermediate and final-state particles are combined into $B$ mesons.
At every stage of the procedure, the probability for the combined particle to be correct is evaluated by a \BDT
which maps input features related to the particle (four-momentum, vertex position, angles between daughter particles etc.) to a single classifier
output score. The final output score related to the quality of reconstruction of the $B$ meson is denoted as \feiProb. 
The schematic visualisation of the reconstruction process is shown in \Cref{fig:fei_schematic}.
\begin{figure}[hbtp!]
    \centering
    \includegraphics[width=0.45\textwidth]{figures/event_reconstruction/FEI_tagging.png}
    \caption{\label{fig:fei_schematic} 
    The schematic overview of \FEI.
    The algorithm reconstructs $B^+$ ($B^0$) candidates in 36 (32) hadronic decay chains
    in six reconstruction stages that combine final-state and intermediate particles.
    Credit to Ref.~\cite{Keck:2018lcd}.
    }
\end{figure}

In total, \FEI reconstructs $\order(10000)$ distinct decay chains and provides $B^+$ meson candidates in 36 hadronic decay modes, and $B^0$ candidates in 32 hadronic decay modes.
As a result, two \FEI modes are differentiated: 
\feiBp, which combines \Bpm meson candidates;
and \feiBz, which combines \Bz meson candidates.
Each event may have more than one candidate reconstructed in the same and/or different decay channels and/or \FEI modes.
The reconstructed decay channels for \feiBp and \feiBz modes are shown in \Cref{tab:fei_modes}.

\begin{table}
    \centering
    \caption{\label{tab:fei_modes}
    The $B$ meson decay modes reconstructed by the \FEI algorithm.
    \FEI modes reconstructed as \feiBp and \feiBz are listed separately.
    }
    \begin{tabular}{|l|l|l|}
    \hline
    &\feiBp modes & \feiBz modes\\
    \hline
    1.&$B^{+} \rightarrow \bar{D}^{0} \pi^{+}$                         &     $B^{0} \rightarrow D^{-} \pi^{+}$ \\
    2.&$B^{+} \rightarrow \bar{D}^{0} \pi^{+} \pi^{0}$                 &     $B^{0} \rightarrow D^{-} \pi^{+} \pi^{0}$ \\
    3.&$B^{+} \rightarrow \bar{D}^{0} \pi^{+} \pi^{0} \pi^{0}$         &     $B^{0} \rightarrow D^{-} \pi^{+} \pi^{0} \pi^{0}$\\
    4.&$B^{+} \rightarrow \bar{D}^{0} \pi^{+} \pi^{+} \pi^{-}$         &     $B^{0} \rightarrow D^{-} \pi^{+} \pi^{+} \pi^{-}$\\
    5.&$B^{+} \rightarrow \bar{D}^{0} \pi^{+} \pi^{+} \pi^{-} \pi^{0}$  &     $B^{0} \rightarrow D^{-} \pi^{+} \pi^{+} \pi^{-} \pi^{0}$\\
    6.&$B^{+} \rightarrow \bar{D}^{0} D^{+}$                           &     $B^{0} \rightarrow \bar{D}^{0} \pi^{+} \pi^{-}$\\
    7.&$B^{+} \rightarrow \bar{D}^{0} D^{+} K_{S}^{0}$                 &     $B^{0} \rightarrow D^{-} D^{0} K^{+}$\\
    8.&$B^{+} \rightarrow \bar{D}^{0 *} D^{+} K_{S}^{0}$               &     $B^{0} \rightarrow D^{-} D^{0 *} K^{+}$\\
    9.&$B^{+} \rightarrow \bar{D}^{0} D^{+*} K_{S}^{0}$                &     $B^{0} \rightarrow D^{-*} D^{0} K^{+}$\\
    10.&$B^{+} \rightarrow \bar{D}^{0 *} D^{+*} K_{S}^{0}$              &     $B^{0} \rightarrow D^{-*} D^{0 *} K^{+}$\\
    11.&$B^{+} \rightarrow \bar{D}^{0} D^{0} K^{+}$                     &     $B^{0} \rightarrow D^{-} D^{+} K_{S}^{0}$\\
    12.&$B^{+} \rightarrow \bar{D}^{0 *} D^{0} K^{+}$                   &     $B^{0} \rightarrow D^{-*} D^{+} K_{S}^{0}$\\
    13.&$B^{+} \rightarrow \bar{D}^{0} D^{0 *} K^{+}$                   &     $B^{0} \rightarrow D^{-} D^{+*} K_{S}^{0}$\\
    14.&$B^{+} \rightarrow \bar{D}^{0 *} D^{0 *} K^{+}$                 &     $B^{0} \rightarrow D^{-*} D^{+*} K_{S}^{0}$\\
    15.&$B^{+} \rightarrow D_{s}^{+} \bar{D}^{0}$                       &       $B^{0} \rightarrow D_{s}^{+} D^{-}$\\
    16.&$B^{+} \rightarrow \bar{D}^{0 *} \pi^{+}$                       &     $B^{0} \rightarrow D^{-*} \pi^{+}$\\
    17.&$B^{+} \rightarrow \bar{D}^{0 *} \pi^{+} \pi^{0}$               &     $B^{0} \rightarrow D^{-*} \pi^{+} \pi^{0}$\\
    18.&$B^{+} \rightarrow \bar{D}^{0 *} \pi^{+} \pi^{0} \pi^{0}$       &     $B^{0} \rightarrow D^{-*} \pi^{+} \pi^{0} \pi^{0}$\\
    19.&$B^{+} \rightarrow \bar{D}^{0 *} \pi^{+} \pi^{+} \pi^{-}$       &     $B^{0} \rightarrow D^{-*} \pi^{+} \pi^{+} \pi^{-}$\\
    20.&$B^{+} \rightarrow \bar{D}^{0 *} \pi^{+} \pi^{+} \pi^{-} \pi^{0}$&     $B^{0} \rightarrow D^{-*} \pi^{+} \pi^{+} \pi^{-} \pi^{0}$\\
    21.&$B^{+} \rightarrow D_{s}^{+*} \bar{D}^{0}$                      &     $B^{0} \rightarrow D_{s}^{+*} D^{-}$\\
    22.&$B^{+} \rightarrow D_{s}^{+} \bar{D}^{0 *}$                     &     $B^{0} \rightarrow D_{s}^{+} D^{-*}$\\
    23.&$B^{+} \rightarrow \bar{D}^{0} K^{+}$                           &     $B^{0} \rightarrow D_{s}^{+*} D^{-*}$\\
    24.&$B^{+} \rightarrow D^{-} \pi^{+} \pi^{+}$                       &     $B^{0} \rightarrow J / \psi K_{S}^{0}$\\
    25.&$B^{+} \rightarrow D^{-} \pi^{+} \pi^{+} \pi^{0}$               &      $B^{0} \rightarrow J / \psi K^{+} \pi^{-}$\\
    26.&$B^{+} \rightarrow J / \psi K^{+}$                              &    $B^{0} \rightarrow J / \psi K_{S}^{0} \pi^{+} \pi^{-}$\\
    27.&$B^{+} \rightarrow J / \psi K^{+} \pi^{+} \pi^{-}$              &     $B^{0} \rightarrow \Lambda_{c}^{-} p \pi^{+} \pi^{-}$\\
    28.&$B^{+} \rightarrow J / \psi K^{+} \pi^{0}$                      &     $B^{0} \rightarrow \bar{D}^{0} p \bar{p}$\\
    29.&$B^{+} \rightarrow J / \psi K_{S}^{0} \pi^{+}$                  &     $B^{0} \rightarrow D^{-} p \bar{p} \pi^{+}$\\
    30.&$B^{+} \rightarrow \Lambda_{c}^{-} p \pi^{+} \pi^{0}$           &     $B^{0} \rightarrow D^{-*} p \bar{p} \pi^{+}$\\
    31.&$B^{+} \rightarrow \Lambda_{c}^{-} p \pi^{+} \pi^{-} \pi^{+}$   &     $B^{0} \rightarrow \bar{D}^{0} p \bar{p} \pi^{+} \pi^{-}$\\
    32.&$B^{+} \rightarrow \bar{D}^{0} p \bar{p} \pi^{+}$               &     $B^{0} \rightarrow \bar{D}^{0 *} p \bar{p} \pi^{+} \pi^{-}$\\
    33.&$B^{+} \rightarrow \bar{D}^{0 *} p \bar{p} \pi^{+}$             &     \\
    34.&$B^{+} \rightarrow D^{+} p \bar{p} \pi^{+} \pi^{-}$             & \\
    35.&$B^{+} \rightarrow D^{+*} p \bar{p} \pi^{+} \pi^{-}$            & \\ 
    36.& $B^{+} \rightarrow \Lambda_{c}^{-} p \pi^{+}$                   & \\
    \hline
\end{tabular}
\end{table}


This thesis uses data and simulation samples following the standard Belle~II approach, where sub-samples of data and \MC with the FEI algorithm applied are produced centrally, referred to as \textit{\FEI skims}.
In order to make the \FEI algorithm more computationally efficient, event selections are made to reject events highly incompatible with one of the $B$ mesons decaying hadronically.
This decision is based on tracks and clusters as per standard Belle~II reconstruction guidelines with additional selections, summarised in \Cref{tab:fei_objects}.
Overeall, they ensure that only energetic tracks originating from the interaction point are selected.
They also minimise the impact of beam background clusters or clusters for which no track information can be associated (outside of \CDC acceptance).

\begin{table}[hbtp!]
    \centering
     \caption{\label{tab:fei_objects} Definitions for objects used in \FEI selections.}
     \resizebox{0.75\textwidth}{!}{
     \begin{tabular}{|l|c|c|} 
        \hline

     Object name & Definition\\
     \hline
     Cleaned tracks & $  |d_0|<0.5~\cm,\quad z_0<2~\cm,\quad p_T>0.1~\gevc $\\
     Cleaned \ECL clusters & $17~\degrees<\theta<150~\degrees,\quad E>0.1~\gev$\\
     \hline

     \end{tabular}
     }
\end{table}
Using the definitions of cleaned tracks and \ECL clusters, reconstructed events are filtered and only the events that pass the requirements are analysed by the \FEI algorithm.
Such requirements are summarised in \Cref{tab:fei_precuts}.
The selection of events with at least 3 cleaned tracks in the event and at least 3 cleaned \ECL clusters is based on the fact that \BB events produce $\sim10$ charged tracks and neutral particles~\cite{BaBar:2014omp}.
Furthermore, the measured energy of the event is required to exceed 4~\gev in the \epem collision centre-of-mass frame.
This is a purely pragmatic requirement: because no neutrinos or missing-momentum are present in a hadronic decay, the energy cannot be much lower than 5.28~\gev.
Finally, the total deposited energy registered by the \ECL in the event is required to be between 2 and 7~\gev.
Hadronic events are expected to deposit significantly more energy than 2~\gev.
On the other hand, many lower energy particles should be stopped within \PXD, \SVD, \CDC or \TOP, meaning that their energy deposit in the \ECL would be negligible.
Therefore, a 7~\gev \ECL energy upper limit ensures that low-multiplicity events, such as \epem\ra\epem, are immediately removed to ensure a better-optimised workflow.
\begin{table}[hbtp!]
    \centering
     \caption{\label{tab:fei_precuts} Selections applied before running the \FEI algorithm.
     Cleaned tracks and clusters are defined in \Cref{tab:fei_objects}.
     }
     \resizebox{0.75\textwidth}{!}{
     \begin{tabular}{|l|c|} 
    \hline
     Selection description & Selection\\
     \hline
     Number of cleaned tracks in event \quad & $\geq 3$\\
     Number of cleaned \ECL clusters in event \quad & $\geq 3$\\
     Total measured centre-of-mass energy in the event \quad & $> 4~\gev$\\
     Total energy of cleaned \ECL clusters \quad &\multirow{2}{*}{$2~\gev<E<7~\gev$}\\
     and deposits associated with cleaned tracks \quad & \\
     \hline
    \end{tabular}
     }
\end{table}

The events that pass the requirements of \Cref{tab:fei_precuts} are analysed by the \FEI algorithm.
In each event, multiple \FEI candidates can be reconstructed (see \Cref{fig:fei_tag_reco_candidates}).
To focus only on the candidates that are correctly reconstructed, selections on \DeltaE and \Mbc are made, as well as a loose requirement on \feiProb.
The selections shown in \Cref{tab:fei_skim_cuts} are standard selections that are applied on the Belle~II \FEI skims.

\begin{table}[hbtp!]
    \centering
     \caption{\label{tab:fei_skim_cuts} 
     Additional selections that reduce the data sets after applying \FEI, focusing only on well-reconstructed tag-side candidates.
     These \FEI skim selections are the nominal ones, which are applied on all \FEI skimmed data sets in Belle~II.
     In this analysis, only the selection on the tag-\B meson is tightened in order to remove the edge effects.
     Such effects arise after applying a kinematic fit on the tag-side products.
     }
     \resizebox{0.75\textwidth}{!}{
        \begin{tabular}{l|c|c|}
        \hline
        Variable &    \FEI skim selections  & Selections in this analysis \\
        \hline
        $\Mbc (\mathrm{tag})$ & $>5.24~\gev$ & $>5.245~\gev$  \\
        $\Delta E (\mathrm{tag})$ & \multicolumn{2}{c}{$-0.15$ to $0.1$~\gev} \\
        \feiProb E & \multicolumn{2}{c}{$> 0.001$}\\
        
        \hline
        
\end{tabular}
    
     }
\end{table}

The tag-side candidates that pass the nominal \FEI requirements undergo a kinematic fit \cite{Belle-IIanalysissoftwareGroup:2019dlq}, 
where the particles used to reconstruct the tag-$B$ candidate are fitted with a common vertex constraint.
Candidates that fail the fit are rejected.
This improves the resolution of the $B$ meson momentum for correct candidates but may shift their momentum.
Therefore, to avoid distribution-edge effects, a tighter \Mbc selection is used in this analysis, as illustrated in \Cref{tab:fei_skim_cuts}.
The tag-side selections used in this analysis do not affect \BtoXsgamma, 
as correct tag-side candidates with 
lower \Mbc, 
higher $|\DeltaE|$ 
or lower \feiProb (compared to \Cref{tab:fei_skim_cuts}) are uncommon.


\subsection{Candidate photon reconstruction}\label{sec:gamma_reconstruction}

Only the photon from \BtoXsgamma can be reconstructed while ensuring a model-independent inclusive measurement.
In order to reduce the quickly growing number of background photon candidates, only events where at least one photon satisfies $\Estar>1.2~\gev$ are considered.
Photons must also be within the \CDC acceptance ($17-150~\degrees$).
These requirements are summarised in \Cref{tab:photon_requirements}.
Reconstructed photon energy is boosted to the signal $B$ meson rest frame based on the Lorentz transformation in \Cref{sec:appendix_boosting_to_b_frame}.
\begin{table}[hbtp!]
    \centering
     \caption{\label{tab:photon_requirements} Requirements for photons in reconstructed events.}
     \resizebox{0.75\textwidth}{!}{
     \begin{tabular}{|l|c|}
     \hline 
     Selection description & Selection\\
     \hline
     Number of photons with \Estar>1.2~\gev & $  N(\Estar>1.2~\gev)\geq1 $\\
     Polar angle of photon & $17~\degrees<\theta<150~\degrees$\\
     \hline
     \end{tabular}
     }
\end{table}

\subsection{Overview of the selected sample}\label{sec:reconstruction_overview}

After the reconstruction, based on the \MC samples, an event can have up to 20 tag-$B$ candidates.
The sample is broken down to show the relative fraction of the total number of tag-side $B$ meson candidates in 
\Cref{fig:fei_tag_reco_candidates}.
About 62\% (72\%) of events for \feiBp (\feiBz) modes have only one tag-side candidate.
About 21\% (18\%) of events for \feiBp (\feiBz) modes have two tag-side candidates, and 8\% (5\%) have three.
The number of candidates per event reduces quickly, but faster for \Bz modes, with roughly 2\% (1\%) 
of events having more than five candidates for \feiBp (\feiBz).
The same event can have a \Bp and \Bz candidate reconstructed.

A similar distribution for the number of signal-side photon candidates with a threshold of $\EB>1.4~\gev$ applied is shown in \Cref{fig:photon_reco_candidates}.
The highest energy photon is the sole candidate in the event in 98\% of the cases in generic \MC.
Similar studies on signal \MC show that the highest energy photon is expected to come from \BtoXsgamma in 99.9\% of the cases.
The $\EB>1.4~\gev$ selection is chosen pragmatically to maintain a reasonable data set memory size without losing signal events.
As the number of photon candidates grows swiftly with decreasing energy, this threshold still provides access to the majority of the \BtoXsgamma decay phase space.
\begin{figure}[hbtp!]
    \centering
    \subcaptionbox{\label{fig:fei_tag_reco_candidates}}{
    \includegraphics[width=0.45\textwidth]{figures/event_reconstruction/Bboth_total_tag_candidates.pdf}
    }
    \subcaptionbox{\label{fig:photon_reco_candidates}}{
        \includegraphics[width=0.45\textwidth]{figures/event_reconstruction/Bboth_total_photon_candidates.pdf}
    }
    \caption{\label{fig:reco_candidates} Relative fractions of events for the number of 
    reconstructed \B meson candidates (\subref{fig:fei_tag_reco_candidates}) and
    reconstructed photon candidates (\subref{fig:photon_reco_candidates}) in the generic \MC sample.
    In (\subref{fig:fei_tag_reco_candidates}), the overall volume of candidates is similar for \feiBp and \feiBz modes, 
    with around one in two events only having a single candidate per event.
    Conversely, as depicted in (\subref{fig:photon_reco_candidates}),
    the vast majority of events contain only a single signal photon with $\EB>1.4~\gev$.
    Two or more photon candidates are present only $\mathcal{O}(1)\%$ of the time.
    }
\end{figure}

The reconstructed \BtoXsgamma spectrum in generic \MC with the previously laid-out requirements are shown in \Cref{fig:spectrum_after_reco}.
Note that these events can contain multiple combinations of a photon and tag-side candidate per event.
Overall, it may seem that the \feiBz mode has a higher signal-to-background ratio compared to \feiBp.
However, one has to take into account that \feiBp and \feiBz modes result
from different reconstruction chains.
Furthermore, \feiBz has fewer modes than \feiBp (see \Cref{sec:tag_reconstruction}).
Therefore, without additional studies that follow in \Cref{sec:select_tag_between_modes,sec:select_best_candidate} such a conclusion cannot be unambiguously drawn. 
On the other hand, inspecting \Cref{fig:untagged_btosgamma_background,fig:spectrum_after_reco}, it is clear that a better signal-to-background ratio can already be observed even without any additional background treatment.

\begin{figure}[hbtp!]
    \centering
    \subcaptionbox{\label{fig:spectrum_after_reco_bplus}}{
        \includegraphics[width=0.45\textwidth]{figures/event_reconstruction/Bp_tagged_background.pdf}
        }
    \subcaptionbox{\label{fig:spectrum_after_reco_bzero}}{
    \includegraphics[width=0.45\textwidth]{figures/event_reconstruction/Bz_tagged_background.pdf}
    }
    \caption{\label{fig:spectrum_after_reco} \BtoXsgamma spectrum in generic \MC after event reconstruction in \feiBp and \feiBz modes.
    Overlaid are events from signal \MC, where the photon comes from \BtoXsgamma, multiplied by a scaling factor.
    These Figures may include multiple tag-$B$ and photon entries per event and can be compared with \Cref{fig:untagged_btosgamma_background}.
    }
\end{figure}

The tag-side probability distributions provided by the \FEI classifier are shown in \Cref{fig:sigprob_after_reco}.
They further emphasise the differences between tag candidates reconstructed in \feiBp and \feiBz modes.
A selection on the \feiProb variable is not trivial;
even after disregarding the differences between \feiBp and \feiBz modes, 
the \feiProb values may be different within the same-charged $B$ mode.
This is shown and discussed in \Cref{sec:appendix_fei_signal_probabilities}.
Tight direct selection may result in a selection of reconstruction channels but not necessarily the quality of reconstruction, as seen in \Cref{fig:feisigprobs1,fig:feisigprobs2,fig:feisigprobs3,fig:feisigprobs4}.
To avoid such bias, further \feiProb thresholds are not considered in this analysis.

\begin{figure}[hbtp!]
    \centering
    \subcaptionbox{\label{fig:sigprob_after_reco_bplus}}{
        \includegraphics[width=0.45\textwidth]{figures/event_reconstruction/Bp_tagged_background_feiSigProb.pdf}
        }
    \subcaptionbox{\label{fig:sigprob_after_reco_bzero}}{
    \includegraphics[width=0.45\textwidth]{figures/event_reconstruction/Bz_tagged_background_feiSigProb.pdf}
    }
    \caption{\label{fig:sigprob_after_reco} Tag-side \feiProb after reconstructing \BtoXsgamma events in generic \MC in \feiBp and \feiBz modes.
    Overlaid are events from signal \MC, where the photon comes from \BtoXsgamma, multiplied by a scaling factor.
    These Figures may include multiple tag-$B$ and photon entries per event.
    }
\end{figure}

% \subsection{Event topology reconstrunction}
% Many quantities that parametrise the \epem collision product topology in terms of tracks and neutral clusters in the event will be used.
% Using a definition, equivalent to cleaned tracks and clusters we recon

\section{Photon candidate selection}\label{sec:photon_selection}
The previous section overviewed the samples that are reconstructed using basic requirements laid out in \Cref{sec:tag_reconstruction,sec:gamma_reconstruction}.
In this section concrete selections will be discussed that will lead to background suppression and a best photon candidate
and best tag-candidate selection.

\subsection{Primary photon candidate selection}\label{sec:primary_photon_candidate_selection}
Contrary to the tag side, a selection of the the best photon candidate in the range $\EB>1.4~\gev$ is effectively trivial based on the discussion in \Cref{sec:event_reconstruction}.
As for 99.7\% and 99.8\% of the signal \MC sample the highest \EB photon is the correct photon originating from \BtoXsgamma decay,
this is chosen as then best photon-candidate requirement with virtually no signal efficiency loss.
Judging from \Cref{fig:photon_reco_candidates}, this provides an approximately 3\% background suppression.
For the rest of the thesis, this selection on photon candidates will always implied in figures and calculations.
\subsection{Main photon backgroundes sources}\label{sec:main_background_sources}

Based on \Cref{fig:spectrum_after_reco}, number of photon and tag candidates originating 
in non-\BB events is significantly larger than number of \B meson events.
The proportion of \qqbar to \BB event candidates is 92.5\% to 7.5\% for \FEI \Bp mode;
and 91.7\% to 8.3\% for \FEI \Bz.

The majority of background photon candidates originate in $\piz\ra\g\g$ or $\eta\ra\g\g$ decays.
This in total accounts for roughly 85\% of background photon candidates.
Photon candidates, broken down by their mother-particle, are shown in \Cref{fig:photon_sources}.
Other sources, such as initial-state radiation, neutron annihilation, parton-shower final-state radation, $\omega(782)$, $\eta'$ decays (in decreasing order) individually make up between 0.5\%-3\% of sample.
All other sources individually make up less than 0.5\% and include various hadron decays that are produced in continuum or \B events.
The backgrounds are similar for both \FEI \Bp and \Bz modes.
Note that some photon candidates can be misidentified, in particular neutrons (discussed further in \Cref{sec:selection_clusZMVA}).

The reason why $\piz\ra\g\g$ and $\eta\ra\g\g$ decays are such a prominent background is related to the fact that they can often be produced in hadronic decays and tend to decay asymmetrically -- where one photon has a much larger energy than the other.
The hadronic decay, overall, mimics the hadronised $X_s$ system, whereas the high-energy photon is taken as the high-energy photon candidate, therefore resulting is similar kinematics.
However, for \B decays, this background drops off rapidly with photon energy, and at high-\EB becomes negligibly small because processes producing photons with $\EB\approx m_B/2$ in \B decays are rare.
No such constraint exist for continuum events where light-hadrons can be created in large numbers.
Therefore, \piz and \eta suppression, while important for \B decays, also highly coincides with continuum event suppression.

\begin{figure}[htbp!]
    \centering
    \subcaptionbox{\label{fig:bp_photon_sources}}{
        \includegraphics[width=0.395\textwidth]{figures/event_selection/Bp_photon_sources.pdf}
    }
    \subcaptionbox{\label{fig:bz_photon_sources}}{
        \includegraphics[width=0.395\textwidth]{figures/event_selection/Bz_photon_sources.pdf}
        }
    \caption{\label{fig:photon_sources} The background photon distribution after reconstruction, stacked by the photon mother-particle species.
    A scaled \BtoXsgamma spectrum is also overlaid.
    Only one photon candidate per event is shown, but at this stage it may still be paired with multiple tag-side candidates.
    Roughly 85\% of candidates originate in \pi\ra\g\g or \eta\ra\g\g decays.
    Other important backgrounds are photons from initial-state radiation and bremmstrahlung; and neutron-annihilation processes.
    These account for approximately 3\% each.
    The leftover 10\% originate in various other decays.}
\end{figure}

At this stage, \BtoXsgamma events make up 0.05\% of the \FEI \Bp sample and 0.07\% of the \FEI \Bz sample.
To reduce the discussed background components the following strategy is adopted:
\begin{itemize}
    \item Suppress misidentified photons (different particle species);
    \item Suppress $\piz\ra\g\g$ and $\eta\ra\g\g$ decays;
    \item Suppress \epem\ra\qqbar events;
    \item Reoptimise all selections simultaneously to adopt a final set of selections.
\end{itemize}

\subsection{Misidentified photon suppression}\label{sec:selection_clusZMVA}

Neutrons, $K_L^0$, protons, electrons where tracking for the particle did not succeed and, less commonly, other charged hadrons, can leave clusters in the \ECL which are misidentified as photons.
The photon misidentification rate is given in \Cref{tab:misidentified_photons}.
The main misidentified photon candidates originate from neutrons, with small contribution from electron and $K_L^0$ showers.

\begin{table}[htbp!]
    \centering
    \caption{\label{tab:misidentified_photons} Photon misidentification rates after reconstruction.
    The majority of photons are identified correctly, with the largest component coming from misidentified neutron showers and $K_L^0$ deposits.
    The rates are similar for \FEI \Bp and \Bz modes which is consistent with the fact that this property is independent on the decayin $B$ charge.}
    \resizebox{1\textwidth}{!}{
\begin{tabular}{|l||c|c||c|c||c|c||c|c||c|c||c|}
    \hline
    Particle species & \multicolumn{2}{c||}{$\gamma$} & \multicolumn{2}{c||}{$n^0$} & \multicolumn{2}{c||}{$\en$} & \multicolumn{2}{c||}{$K_L^0$} & \multicolumn{2}{c||}{$p^-$} & Other \\
    \hline
    Candidate rate (\FEI \Bp | \Bz) & 96.1\% & 96.0\% & 2.4\% & 2.5\% & 0.5\% & 0.5\% & 0.5\% & 0.5\%  & 0.3\% & 0.3\% & 0.2\% \\
    \hline
\end{tabular}
}
\end{table}

The interactions of these particle species in the \ECL \textit{shower} -- produce a cascade of secondary particles which may produce tertiary particles etc.
Generally, the total energy deposit and distribution between \ECL crystals, also known as \textit{shower-shape}, is different depending on the particle-species due to their different radiation lengths.
This can be used to distinguish photon clusters using \MVA methods.
A technique achieving this, which uses the moments of Zernike polynomials, is documented in Ref.\cite{Hershenhorn:2468}.
This approach is implemented in Belle II analysis software and used in this analysis.
Here, I provide a condensed overview of the approach.

A complete set of complex two-dimensional polynomial is defined as:
\begin{equation}
    V_nm(\rho\cos\alpha,\rho\sin\alpha) = R_{nm}(\rho)\exp{im\alpha},
\end{equation}
where ${x=\rho\cos\alpha, y=\rho\sin\alpha}$ are polar coordinates, $m$ is an integer and $R_{nm}(\rho)=V(\rho,0)$ is a polynomial of degree $n$.
The expression for Zernike polynomial is given as:
\begin{equation}
    R_{nm}(\rho) = \sum^{\frac{n-|m|}{2}}_{s=0}(-1)^s \frac{(n-s)!}{ s! \left(\frac{n+|m|}{2}-s \right) ! \left( \frac{n-|m|}{2}-s\right) !}\rho^{n-2s}.
\end{equation}
The moments of a function $f(\rho\cos\alpha,\rho\sin\alpha)$ is expressed in terms of $V_{nm}$ as:
\begin{equation}
    Z_{nm} = \frac{n+1}{\pi} \int_0^{2\pi}\int^1_0 V^*_{nm}(\rho\cos\alpha,\rho\sin\alpha)f(\rho\cos\alpha, \rho\sin\alpha)\rho d\rho d\alpha.
\end{equation}
$Z_{nm}$ are called Zernike moments.
They have many useful properties that make them usable in image recognition, field of optics, and more importantly, particle physics particle identification algorithms (see. Ref.\cite{Hershenhorn:2468} and references therein).

A Dirac comb function is defined to parametrise a particle-shower in the \ECL as:
\begin{equation}
    f(\vec{x}) = \sum_i \delta(\vec{x}-\vec{x}_i)\frac{w_iE_i}{\sum w_iE_i},
\end{equation}
where $x$ is a dimensionless crystal position in the perpendicular plane, $i$ is a crystal index, summing over all crystals in a given particle shower, $E_i$ is the energy of the $i$-th crystal.
As showers can overlap, $w_i$ is the fraction of energy in a crystal that is associated with the currently investigated shower.
It can be shown, that Zernike moments for \ECL showers can then be expressed as:
\begin{equation}
    |Z_{nm}| = \frac{n+1}{\pi}\frac{1}{\sum_iw_iE_i}\left|\sum_iR_{nm}(\rho_i)\exp(-im\alpha_i)w_iE_i\right|.
\end{equation}

The work in Ref.\cite{Hershenhorn:2468} selects the best combination of eleven $|Z_{nm}|$ which provides a strongest separation between hadronic showers and electromagnetic showers.
The chosen combination of $|Z_{nm}|$ is combined using a \BDT and produces a single output, hereafter referred to as $\ZMVA\in(0,1)$.
The \ZMVA distribution for \BtoXsgamma candidates in generic \MC and signal \MC events is shown in \Cref{fig:zmva_distribution}.

\begin{figure}[htbp!]
    \centering
    \subcaptionbox{\label{fig:bp_zmva_distribution}}{
        \includegraphics[width=0.395\textwidth]{figures/event_selection/Bp_zernikeMVA_mcPDG.pdf}
    }
    \subcaptionbox{\label{fig:bz_zmva_distribution}}{
        \includegraphics[width=0.395\textwidth]{figures/event_selection/Bz_zernikeMVA_mcPDG.pdf}
        }
    \caption{\label{fig:zmva_distribution} The distributions of the \ZMVA for different particle species that are reconstructed as photon candidates.
    The candidates presented in these figures are the same as those in \Cref{fig:photon_sources}.
    A scaled \ZMVA distribution for \BtoXsgamma events is overlaid.
    A good separation can observed between real photons and hadronic showers misidentified as photons.}
\end{figure}

Interestingly, \ZMVA is able to provide a good separation against real photon candidates that originate in neutron annihilation events.
Overall, for true \BtoXsgamma photons this distribution is strongly peaking in 0.8 -- 1 region.
For non-\BtoXsgamma photon candidates this distribution drops off slower is relatively uniform from 0 to 0.8.
The \ZMVA candidate for true photon candidates exclusively is shown in \Cref{fig:zmva_distribution_sources}.
Therefore, this variable is able to perform an efficient suppression of candidates that are fake-photon or photons originating in some undesired processes.

\begin{figure}[htbp!]
    \centering
    \subcaptionbox{\label{fig:bp_zmva_distribution_sources}}{
        \includegraphics[width=0.395\textwidth]{figures/event_selection/Bp_zernikeMVA_true_photons_only.pdf}
    }
    \subcaptionbox{\label{fig:bz_zmva_distribution_sources}}{
        \includegraphics[width=0.395\textwidth]{figures/event_selection/Bz_zernikeMVA_true_photons_only.pdf}
        }
    \caption{\label{fig:zmva_distribution_sources} The distributions of the \ZMVA for different photon sources in generic \MC.
    The candidates presented here are only those which are true photons in \Cref{fig:zmva_distribution}.
    A scaled \ZMVA distribution for \BtoXsgamma events is overlaid.
    Photons associated with neutron annihilation events are clearly separated.}
\end{figure}

\subsection{Suppression of \texorpdfstring{\piz}{pi0} and \texorpdfstring{\eta}{eta} diphoton decays}\label{sec:selection_vetos}

85\% of backgroun photons in this analysis originates from photons that are produced in $\piz\ra\g\g$ or $\eta\ra\g\g$ decays.
A lot of such light mesons originate in continuum events, but even in \BB events many \piz and \eta decays are produced.
Therefore an efficient mechanism to suppress \piz and \eta related photon candidates is required.

In this analysis, a suppression tool, called \textit{\piz~and~\eta~veto} is utilised.
It is implemented as part of \basftwo and is a standard Belle II approach for suppressions of radiative backgrounds from light-mesons.
Here, I provide an overview of the training and the validation process which is performed by an independent analysis.

The general idea of the \textit{\piz~and~\eta~veto} tool is to pair the high-energy photon signal candidate (\textit{hard} photon) candidate with lower-energy photons (\textit{soft} photons) in the event.
The compatibility of the combination with a $\piz\ra\g\g$ or $\eta\ra\g\g$ decay is evaluated and a probability-like quantity is calculated to quantify it.

The soft-photon candidate is selected with an energy $30~\mev$ ($40~\mev$ in backward \ECL endcap) for $\piz$ or $60~\mev$ for $\eta$.
The photon is also required to have deposited the energy in 2 or more crystals.
Furthermore, photon candidates required to have an associated cluster time no more than one standard deviation away from 0.
These selections ensure that beam-backround photons, neutral hadrons misidentified as photons and misreconstructed charged particles are not included in the soft-photon sample.

The soft photons that pass these selections are combined with the high-energy photon signal candidate.
The following observables are then calculated and used to train a \MVA classifier:
\begin{itemize}
    \item Invariant mass of the soft photon and hard photon combination;
    \item Soft photon energy in the laboratory frame;
    \item Soft photon \ECL cluster polar angle;
    \item Distance between the soft photon \ECL cluster and the nearest track extrapolated to the \ECL;
    \item Helicity angle of the combination.
\end{itemize}
The classifier for $\eta\ra\g\g$ includes additional observables to increase the separation power:
\begin{itemize}
    \item \ZMVA of the soft photon;
    \item Number of crystals where the soft photon has deposited energy;
    \item Ratio of soft photon energy in 3-by-3 crystals around the central crystal to soft photon energy in the 5-by-5 crystals with the corner crystals removed.
\end{itemize}
For every combination of a soft and hard photon the \MVA produces an output between 0 and 1.
The same hard photon is paired with all soft photons in a given event, and the largest \MVA output is assigned to it as the $\piz$ or $\eta$ probability.
This \MVA output is denoted as \piVeto or \etaVeto, respectively.
Note that despite the nomenclature, this variable is only probability-like (i.e. $\mathcal{P}\in(0-1)$), and does not truly represent a probability.

The distributions for \piVeto and \etaVeto are shown in \Cref{fig:vetos}.
In all cases, \BtoXsgamma can be seen to be strongly peaking near 0, consistent with photons that do not originate from light unflavoured meson decay.

\begin{figure}[htbp!]
    \centering
    \subcaptionbox{\label{fig:bp_piveto}}{
        \includegraphics[width=0.395\textwidth]{figures/event_selection/Bp_piVeto.pdf}
        }
    \subcaptionbox{\label{fig:bz_piveto}}{
        \includegraphics[width=0.395\textwidth]{figures/event_selection/Bz_piVeto.pdf}
        }
    \subcaptionbox{\label{fig:bp_etaveto}}{
            \includegraphics[width=0.395\textwidth]{figures/event_selection/Bp_etaVeto.pdf}
        }
    \subcaptionbox{\label{fig:bz_etaveto}}{
            \includegraphics[width=0.395\textwidth]{figures/event_selection/Bz_etaVeto.pdf}
        }
    \caption{\label{fig:vetos} The distributions of \piVeto (\Cref{fig:bp_piveto,fig:bz_piveto}) and \etaVeto \Cref{fig:bp_etaveto,fig:bz_etaveto} 
    for different photon sources in generic \MC stacked.
    This is shown for all photon candidates included in \Cref{fig:photon_sources}.
    Scaled respective veto probability distributions for \BtoXsgamma events are overlaid.
    The separation power of \etaVeto is hidden by a large number of $\piz\ra\g\g$ events in \Cref{fig:bp_etaveto,fig:bz_etaveto}.
    It is clearly visible when such events are removed as shown in \Cref{fig:vetos_nopi}
    }
\end{figure}

For the case of \piz veto, shown in \Cref{fig:bp_piveto,fig:bz_piveto}, an excellent separation is observed between photons originating in \piz decays and other photons.
\BtoXsgamma and other photon candidate also show a small peak at high-\piVeto values, which aludes to a small inefficiency of the algorithm.
However, compared to the separation power removing large amounts of $\piz\ra\g\g$ background this an acceptable trade-off.
For \etaVeto the separation is less clear.
The reason for this is the fact that the generic \MC sample is dominated by $\piz\ra\g\g$ decays which are not targeted by the \etaVeto classifier.
Removing $\piz\ra\g\g$ decays from the sample, a clear separation of photon candidates originating in $\eta$ decays from other types of decays becomes apparent (see \Cref{fig:vetos_nopi}).

\begin{figure}[htbp!]
    \centering
    \subcaptionbox{\label{fig:bp_etaveto_nopi}}{
            \includegraphics[width=0.395\textwidth]{figures/event_selection/Bp_etaVeto_nopi.pdf}
        }
    \subcaptionbox{\label{fig:bz_etaveto_nopi}}{
            \includegraphics[width=0.395\textwidth]{figures/event_selection/Bz_etaVeto_nopi.pdf}
        }
    \caption{\label{fig:vetos_nopi} The distributions of \etaVeto \Cref{fig:bp_etaveto,fig:bz_etaveto} 
    for different photon sources in generic \MC stacked, but with photons that are associated with $\piz\ra\g\g$ removed.
    This is equivalent to \Cref{fig:bp_etaveto,fig:bz_etaveto} with the aforementioned event stack not included.
    A scaled \etaVeto distribution for \BtoXsgamma events is overlaid.
    Although the separation power is not as strong as in the case of \piVeto (\Cref{fig:bp_piveto,fig:bz_piveto}), a clear peak at low-\etaVeto can be seen for \BtoXsgamma.
    }
\end{figure}

\subsection{Signal-photon background suppression correlation}\label{sec:signal_photon_correlation}

Even though no direct selection is applied on the $X_s$ system, through direct or higher-order correlations with \EB, a bias may be introduced to the photon energy.
To ensure that no such correlation is introduced, a correlation study is performed for \piVeto, \etaVeto and \ZMVA observables.
In principle, it is not important if the selection introduces a bias to the background -- as long as this bias is well reproduced in simulation.
The latter will be validated in \Cref{sec:corrections,sec:signal_modelling}.
Therefore the study is performed exclusively for signal \MC, only focusing on \BtoXsgamma events, as it was aimed to ensure that the photon energy spectrum itself is minimally biased.

\begin{figure}[htbp!]
    \centering
    \subcaptionbox{\label{fig:bp_zmva_correlation}}{
            \includegraphics[width=0.3\textwidth]{figures/event_selection/Bp_zernikeMVA_correlation.pdf}
        }
    \subcaptionbox{\label{fig:bp_piveto_correlation}}{
            \includegraphics[width=0.3\textwidth]{figures/event_selection/Bp_piVeto_correlation.pdf}
        }
    \subcaptionbox{\label{fig:bp_etaveto_correlation}}{
            \includegraphics[width=0.3\textwidth]{figures/event_selection/Bp_etaVeto_correlation.pdf}
        }
    \subcaptionbox{\label{fig:bz_zmva_correlation}}{
            \includegraphics[width=0.3\textwidth]{figures/event_selection/Bz_zernikeMVA_correlation.pdf}
        }
    \subcaptionbox{\label{fig:bz_piveto_correlation}}{
            \includegraphics[width=0.3\textwidth]{figures/event_selection/Bz_piVeto_correlation.pdf}
        }
    \subcaptionbox{\label{fig:bz_etaveto_correlation}}{
            \includegraphics[width=0.3\textwidth]{figures/event_selection/Bz_etaVeto_correlation.pdf}
        }
    \caption{\label{fig:selection_correlations} Correlation tests for background suppression observables described in \Cref{sec:photon_selection}, depicted as a 2D histogram.
    Each row is normalised, such that all bins within that row add up to 1.
    For signal \MC \BptoXsgamma events the tests are shown
    in \Cref{fig:bp_zmva_correlation,fig:bp_piveto_correlation,fig:bp_etaveto_correlation},
    and for \BztoXsgamma in \Cref{fig:bz_zmva_correlation,fig:bz_piveto_correlation,fig:bz_etaveto_correlation}.
    All figures share the same legend provided in \Cref{fig:bp_zmva_correlation}.
    In red line, the average photon energy, $\expval{\EB}$, is shown as a function of the tested observable.
    In black and black-dotted lines -- median and $\pm 1 \sigma$ percentile values of \EB, respectively.
    No strong dependance can be observed in any of the quantities or the 2D maps.
    }
\end{figure}

A two dimensional map of ${\piVeto,\etaVeto,\ZMVA}$ versus \EB is given in \Cref{fig:selection_correlations}.
Because the distributions of the three variables used for background suppression are not uniform, each row is normalised, such that the sum of each row is equal to unity.
This makes the comparison between different number of entries in different bins simpler.
The figure also denotes the average, the median and $\pm 1\sigma$ percentiles of \EB.
It is clear that no strong bias is introduced by any of the observables to any of these quantities.
Furthermore, the structure itself remains constant across all bins and no clear dependance on \EB can be seen.
No significant differences between different \Bp or \Bz samples is observed.
It is therefore concluded that the selections are unbiasing and suitable for signal-side photon background suppression.
The exact selections on these observables will be optimised simultaneously with continuum event suppression in \Cref{sec:final_optimisation}.

\section{\texorpdfstring{\MakeLowercase{\epem\ra\qqbar}}{e+e-->qqbar} event suppression}\label{sec:continuum_suppression}
\Cref{sec:photon_selection} introduced the best-photon candidate strategy, 
as well as selection to suppress events 
where the photon is misidentified or originates in sources different than \BtoXsgamma.
However, as was seen before in, for example. \Cref{fig:photon_reco_candidates_bplus,fig:photon_reco_candidates_bzero},
\epem\ra\qqbar events provide the vast majority of photon candidates.
Therefore, a dedicated event-selection for this type of background was devised.
It takes advantage of  different event topologies expected for $\FourS\to\qqbar$ and $\epem\ra\qqbar$ events.
Loosely speaking, events where a $\FourS\to\BB$ decay is present tend to be more `spherical', when compared with $\epem\ra\qqbar$ events that exhibit a `jet-like' distribution.
This is related to the fact that \epem collissions at $B$ dactory experiments have just enough energy to produce a $B$ pair almost at rest, which means that its decays products, on average, tend to be distributed uniformly in polar and azimuthal angle.
On the other hand, light-quark pairs, produced in \epem collision events, also gain a substantial amount of back-to-back momentum which tends to spread their decay products. 
The schematic idea of this is shown in \Cref{fig:continuum_schematic}.
This section will provide an in-depth discussion on how the discrimination between \BtoXsgamma and continuum is achieved using a \BDT.

\begin{figure}[htbp!]
    \centering
    \includegraphics[width=0.6\textwidth]{figures/continuum_suppression/figure_continuum_suppression_event_shapes.pdf}
    \caption{\label{fig:continuum_schematic} Schematic illustration of continuum and \BB events created after an \epem collision in $B$ factories.
    Events where a $B$ meson is produced are generally more spherical, due to the fact that \FourS is produced at rest and its decays products tend to not have a preferred direction.
    Typical momenta of light-quark and \BB mesons are shown.
    The specific directions shown are illustrative only. 
    }
\end{figure}

\subsection{Training event pre-selection}\label{sec:preselection}

Before a \BDT is trained, it is generally desirable to prepare the datasets such that the classifier 
is able to learn based on relevant data.
Such data preprocessing will be performed based on variables described in \Cref{sec:photon_selection}.
The classifier will then be trained on the reduced dataset

In order to find optimal selections, a figure-of-merit study is performed for each observable.
Two figure of merit options were considered for this analysis, a more standard figure-of-merit $\mathrm{FOM}_1$:
\begin{equation}\label{eq:soversqrtsplusb}
    \mathrm{FOM}_1 = \frac{\mathsf{S}}{\sqrt{\mathsf{S}+\mathsf{B}}},
\end{equation}
and $\mathrm{FOM}_2$ defined in Ref.\cite{Punzi:2003bu} (often referred to as `Punzi' figure-of-merit):
\begin{equation}\label{eq:punzi_fom}
    \mathrm{FOM}_2 = \frac{\mathsf{S}}{\mathsf{S}_0} \frac{1}{\frac{3}{2}+\sqrt{\mathsf{B}}}.
\end{equation}
In both equations, $\mathsf{S}$ is the number of signal events after selection, 
$\mathsf{B}$ is the number of background events after selection, 
and $\mathsf{S}_0$ is the number of signal events before selection.
Although \Cref{eq:punzi_fom} was derived with search-like analyses in mind, 
it is utilised in this analysis to minimise signal model dependancy: the ratio $\mathsf{S}/\mathsf{S}_0$ present in the definition reduces out many model-dependant effects.

For each figure of merit calculation, background events ($\mathsf{B}$) are calculated based on generic \MC,
whereas signal events, $\mathsf{S}$, are calculated based on signal \MC, to ensure a high statistics sample.
In the case of \Cref{eq:soversqrtsplusb}, an appropriate scaling for $\mathsf{S}$ is also used.
Each dataset has duplicate tag-candidate events randomly removed, by picking a random tag-side candidate.
Each figure of merit is then calculated for 200 equally spaced selections in the target observable.
The maximum figure-of-merit point is taken as the optimal pre-selection for each of the variables.
This procedure is shown for $\mathrm{FOM}_2$ in \Cref{fig:selection_optimisations}.
Results for $\mathrm{FOM}_1$ are used as a cross-check for $\mathrm{FOM_2}$ but turn out to be consistent.
Equivalent procedure for $\mathrm{FOM}_1$ is shown in \Cref{sec:appendix_sqrtsplusb_optimisation}.
The resulting optimal selections are also shown in the Figures.

\begin{figure}[htbp!]
    \centering
    \subcaptionbox{\label{fig:bp_zmva_optimisation}}{
            \includegraphics[width=0.3\textwidth]{figures/continuum_suppression/Bp_zernikeMVA_optimisation_punzi.pdf}
        }
    \subcaptionbox{\label{fig:bp_piveto_optimisation}}{
            \includegraphics[width=0.3\textwidth]{figures/continuum_suppression/Bp_piVeto_optimisation_punzi.pdf}
        }
    \subcaptionbox{\label{fig:bp_etaveto_optimisation}}{
            \includegraphics[width=0.3\textwidth]{figures/continuum_suppression/Bp_etaVeto_optimisation_punzi.pdf}
        }
    \subcaptionbox{\label{fig:bz_zmva_optimisation}}{
            \includegraphics[width=0.3\textwidth]{figures/continuum_suppression/Bz_zernikeMVA_optimisation_punzi.pdf}
        }
    \subcaptionbox{\label{fig:bz_piveto_optimisation}}{
            \includegraphics[width=0.3\textwidth]{figures/continuum_suppression/Bz_piVeto_optimisation_punzi.pdf}
        }
    \subcaptionbox{\label{fig:bz_etaveto_optimisation}}{
            \includegraphics[width=0.3\textwidth]{figures/continuum_suppression/Bz_etaVeto_optimisation_punzi.pdf}
        }
    \caption{\label{fig:selection_optimisations} Optimal selection calculation for observables
    described in \Cref{sec:photon_selection} based on $\mathrm{FOM}_2$ (see \Cref{eq:punzi_fom}).
    For \BptoXsgamma events the tests are shown
    in \Cref{fig:bp_zmva_optimisation,fig:bp_piveto_optimisation,fig:bp_etaveto_optimisation},
    and for \BztoXsgamma in \Cref{fig:bz_zmva_optimisation,fig:bz_piveto_optimisation,fig:bz_etaveto_optimisation}.
    The figures show efficiency and $\mathrm{FOM}_2$ score calculated for 200 selections of \piVeto, \etaVeto and \ZMVA.
    The maximum value of $\mathrm{FOM}_2$, the corresponding selection and efficiency are shown as well.
    }
\end{figure}

The results between \Bp and \Bz modes, as well as using $\mathrm{FOM}_1$ and $\mathrm{FOM}_2$ are consistent, with only marginal differences.
Therefore, due to the model-independance of $\mathrm{FOM}_2$, this figure-of-merit will be the only one discussed henceforth.
At this stage, it is unnecessary to choose the `best' selection, as another simultaneous optimisation will be performed later, together with continuum suppression \BDT output.
\todo[inline]{(see XXX when I optimise)}. 
The main goal is to reduce the sample size to include only relevant data, such that the trained \BDT can make decisions for difficult cases that are not easily distinguishable using simple selection.
Based on \Cref{fig:selection_optimisations}, pre-selections are chosen, which suppress background but retain most of the signal.
The requirement for a loose selection are tailored such that more roughly 75\% of \BtoXsgamma candidates are retained and are shown in \Cref{tab:preselections}.
They are chosen to be no tighter than their optimal selection and preferrably considerably looser.

\begin{table}[htbp!]
    \centering
    \caption{\label{tab:preselections} Selections that remove background and misreconstructed candidates,
    preparing the reconstructed datasets \Cref{sec:reconstruction_overview} for continuum \BDT training (\Cref{sec:continuum_training}).
    A later optimisation will be used for a final candidate selection XXXX
    \todo[inline]{add here later}
    }
    
    \begin{tabular}{lr}
    \hline
    Variable &    Loose cut \\
    \hline
    Candidate cleanup \\
    $\Mbc (\mathrm{tag})$ & $>5.245~\gev$\\
    %$\Delta E (\mathrm{tag})$ & $-0.15$ to $0.1$\\
    \Egamma rank & == 1\\
    \hline
    Loose signal side selection \\
    \ZMVA & $>0.5$ \\
    \piVeto            & $<0.4$ \\
    \etaVeto           & $<0.4$ \\
    
    
    \hline
    
    $B^+$ mode: $\gamma$ candidate retention efficiency: 77.9\% \\
    $B^0$ mode: $\gamma$ candidate retention efficiency: 78.9\% \\
    
    \end{tabular}

\end{table}

The pre-selections improve the signal-to-background ratio by roughly an order of magnitude.
This can be clearly seen by comparing \Cref{fig:preselected_photons} with \Cref{fig:spectrum_after_reco}.
The scale at which \BtoXsgamma signal \MC scaled differs at about a factor of 10, meaning that the background reduced by approximately that amount due to these selections.


\begin{figure}[htbp!]
    \centering
    \subcaptionbox{\label{fig:bp_preselected_photons}}{
        \includegraphics[width=0.395\textwidth]{figures/continuum_suppression/Bp_tagged_background_preselection.pdf}
    }
    \subcaptionbox{\label{fig:bz_preselected_photons}}{
        \includegraphics[width=0.395\textwidth]{figures/continuum_suppression/Bz_tagged_background_preselection.pdf}
        }
    \caption{\label{fig:preselected_photons} \BtoXsgamma spectrum in generic \MC after pre-selection for training the continuum suppression \BDT classifier.
    Overlaid are events from signal \MC where the photon comes from \BtoXsgamma, multiplied by a scaling factor.
    Compared to \Cref{fig:spectrum_after_reco}, the effectiveness of background suppression so far is apparent.
    These figures may include multiple tag entries per event.
    }
\end{figure}

Finally, many combinatorial tag-side candidates in \BB events may still appear contribute to the analysis at this stage.
A more detailed definition for a `well-reconstructed' tag will be explored later 
\todo[inline]{add later}.
At this stage, it is sufficient to acknowledge the fact that the vast majority of correctly reconstructed tag-side candidates is expected to lie $\Mbc>5.27~\gevcc$.
Therefore, this requirement is also adopted for studies and training in \Cref{sec:continuum_features}--XXX
\todo[inline]{add X}.

\subsection{Continuum suppression feature selection}\label{sec:continuum_features}

$B$ factory experiments and Belle II have a large selection of observables that can be utilised for continuum suppression and are suitable to be input as variables to a \BDT.
These observables combine describe the event-topology or particles in the event and are optimised to provide separation between \BB and \qqbar events.
There are two caveats that have to be kept in mind for the \BtoXsgamma analysis:
\begin{itemize}
    \item Generally, \BtoXsgamma event-topology may be different compared to \BB events. 
    \BtoXsgamma decays have a single jet-like $X_s$ system (while the other $B$ meson decays hadronically).
    This leads to a somewhat middle-case between a generic-\BB event and a \epem\ra\qqbar event, as it was illustrated in \Cref{fig:continuum_schematic}.
    \item Many of these observables contain momenta, angles or other parameters of some (or even all) particles in the event -- including the $X_s$ system and the photon.
    This may lead to a bias to the spectrum of $X_s$ system.
    Furthermore, even relatively small biases, over many different different training features used, may be learnt by the \BDT and introduced to the spectrum.
    \item Some of the features may perform differently in real-data compared to simulation, due to unexpected differences in alignment, calibration or background distributions.
    As we use simulated data sets to train a \BDT in this analysis, such a comparison is crucial.
\end{itemize}
Given the mentioned points, it is important to test that observables used for the training provide adequate separation between \BtoXsgamma and \qqbar, while no bias is introduced to the photon energy spectrum.
Furthermore, this has to be well-represented in data.

In this analysis, the following observable categories are considered for separation between \epem\ra\qqbar and \BtoXsgamma:
\begin{itemize}
    \item Various thrust-based observables (\Cref{sec:thrusts});
    \item Sphericity and aplanarity (\Cref{sec:sphericity_aplanarity});
    \item Harmonic moments (\Cref{sec:harmonic_moments});
    \item Fox-Wolfram moments (\Cref{sec:fox_wolfram_moments});
    \item Modified Fox-Wolfram moments (\Cref{sec:modified_fox_wolfram_moments});
    \item CLEO cones (\Cref{sec:cleo_cones});
    % \item Angular tag-$B$ meson observables (\Cref{sec:angular_features});
    \item Tag-$B$ meson vertex observables (\Cref{sec:vertex_features});
    \item Flavour tagger output for the tag-$B$ meson(\Cref{sec:flavour_tagger_outputs});
\end{itemize}

In total, this provides 75 potential training features that are tested to be uncorrelated to the photon energy spectrum and adequately described in simulation.
The tests use a metric of \textit{total divergence to the average} (often called \textit{Jensen-Shannon distance}) \cite{Lin:1991abc},
which is used to quantitatively evaluate the similarity between two distributions.
The Jensen-Shannon distance is bounded by 1 for two given probability distributions, $\mathbb{X}_1$ and $\mathbb{X}_2$:
\begin{equation}\label{eq:js_distance}
    0\leq\mathrm{JSD}(\mathbb{X}_1||\mathbb{X}_2) \leq1,
\end{equation}
where exactly similar distributions have a score of 0, and the score tends towards 1 when the distributions are highly-different.

Two tests are performed:
\begin{itemize}
    \item \textbf{Test 1}: $\mathbf{\EB,~\Estar~and~tag\mbox{-}side~\Mbc~bias~test}$:
    to ensure that the classifier does not indirectly select particular $X_s$ or tag-side $B$ channels,
    each tested potential training feature is separated into 5 equally populated regions (\textit{slices}) of \BtoXsgamma events in signal \MC.
    For this test, \BptoXsgamma and \BztoXsgamma are merged.
    In each of these regions, the distribution of \EB, \Estar and \Mbc is compared.
    The Jensen-Shannon distance is required to not be larger than 6\% between any two given slices of a training feature.
    The requirement to pass \textbf{Test 1} has been chosen by observing the typical values of the agreement shown by the tested unbiased distributions.
    If this requirement is not passed by at least one of the distributions (\EB, \Estar or \Mbc), the feature is excluded from the list of final \BDT training features.
    \item \textbf{Test 2}: $\mathbf{\epem\ra\qqbar~data\mbox{-}simulation~similarity~test}$:
    to ensure that simulations adequatly describe the data sets that we aim to suppress.
    This test is only performed if \textbf{Test~1} is passed.
    As this is a blinded analysis, off-resonance data samples are used, which only contain \epem\ra\qqbar events.
    In this case, the Jensen-Shannon distance is calculated between 
    the area-normalised distribution of a training feature in off-resonance data set,
    and the area-normalised distribution of a training feature in \epem\ra\qqbar simulation.
    The metric is required to be no more than 10\%.
    The looser requirement is adopted here, due to the fact that some difference is expected between the distributions,
    as the collision energy in off-resonance data set is different to the one in on-resonance simulation that is used in this analysis.
    Furthermore, an overall smaller off-resonance data set ($\sim19~\invfb$, see \Cref{sec:data_samples}) may have certain difference due to poissonian fluctuations.
\end{itemize}

The tested distributions have the selections from \Cref{sec:preselection} included, except for the case of off-resonance dataset, where the $\Mbc>5.27~\gevcc$ requirement is lifted.
For every event, when more than one tag-side candidate $B$ candidate exists, a random one is picked.
The application of \textbf{Test~1} with exact definitions for the 81 observables are shown in \Cref{sec:appendix_continuum_features}.

Out of 75 potential training features, 27 pass through the requirements of \textbf{Test~1}.
These are passed to \textbf{Test~2}
\textbf{Test~2}, owing to excellent calibration and simulation quality of Belle II, removes a single feature.
The results for the 26 final observables that pass both test requirements and will be used as features in the \BDT training
are shown in \Cref{tab:passing_test1}.

\begin{table}[htbp!]
    \centering
    \caption{\label{tab:passing_test1}The training features for the \epem\ra\qqbar suppression
    that pass the requirements of \textbf{Test~1} (see \Cref{sec:appendix_continuum_features}) and \textbf{Test~2} (see \Cref{sec:appendix_continuum_features_datamc}).
    The table also shows the value of the Jensen-Shannon distances for each observable for the different requirements of both tests.
    Exact definitions of these quantities is provided in \Cref{sec:appendix_continuum_features}.
    Observable groups follow those introduced in the text.
    }   
    \begin{tabular}{l|c|c|c|c|}
    \multirow{3}{*}{Feature name} & \multicolumn{4}{|c|}{Jensen Shannon Distances [$\sqrt{\mathrm{bit}}$]}\\
                                  & \multicolumn{3}{|c|}{Test~1} & Test~2\\
                                  & \EB& \Estar& \Mbc & Data-Sim.\\
    \hline
    \multicolumn{5}{c|}{\textbf{Thrust related}}\\
    \hline
    $\cos\theta_{\mathrm{TB}\wedge\mathrm{TO}}$ & 0.03 & 0.01 & 0.04 & 0.02\\
    $\cos\theta_{\mathrm{TB}\wedge\mathrm{z}}$ & 0.01 & 0.01 & 0.02 & 0.01\\
    $T_{\mathrm{B}}$ & 0.04 & 0.03 & 0.04 & 0.06\\
    $\cos\theta_{\mathrm{T}}$ & 0.02 & 0.02 & 0.01 & 0.02\\
    \hline
    \multicolumn{5}{c|}{\textbf{Harmonic moments}}\\
    \hline
    $B_{1}^T$ & 0.05 & 0.05 & 0.02 & 0.01\\
    $B_{3}^T$ & 0.04 & 0.03 & 0.01 & 0.03\\
    \hline
    \multicolumn{5}{c|}{\textbf{CLEO cones}}\\
    \hline
    $\mathtt{CC}^B_0$ & 0.04 & 0.03 & 0.03 & 0.03\\
    $\mathtt{CC}^B_1$ & 0.03 & 0.03 & 0.02 & 0.03\\
    $\mathtt{CC}^B_2$ & 0.04 & 0.03 & 0.02 & 0.01\\
    $\mathtt{CC}^B_3$ & 0.05 & 0.05 & 0.02 & 0.02\\
    $\mathtt{CC}_0$ & 0.05 & 0.05 & 0.02 & 0.02\\
    $\mathtt{CC}_3$ & 0.06 & 0.06 & 0.02 & 0.02\\
    \hline
    \multicolumn{5}{c|}{\textbf{Modified Fox-Wolfram moments}}\\
    \hline
    $H_{c4}^{so}$ & 0.05 & 0.04 & 0.02 & 0.02\\
    $H_{m2}^{so}$ & 0.03 & 0.03 & 0.02 & 0.02\\
    $H_{m4}^{so}$ & 0.03 & 0.03 & 0.01 & 0.01\\
    $H_{0}^{oo}$ & 0.03 & 0.02 & 0.02 & 0.04\\
    \hline
    \multicolumn{5}{c|}{\textbf{Tag-$\mathbf{B}$ meson vertex observables}}\\
    \hline
    $z$ of tag-B & 0.01 & 0.01 & 0.01 & 0.02 \\
    $\Delta x$ of tag-B & 0.02 & 0.02 & 0.03 & 0.03\\
    $\Delta y$ of tag-B & 0.01 & 0.01 & 0.03 & 0.04\\
    $\Delta z$ of tag-B & 0.02 & 0.02 & 0.04 & 0.02\\
    $\Delta \tau$ & 0.04 & 0.03 & 0.02 & 0.02\\
    $\Delta z$ & 0.04 & 0.03 & 0.02 & 0.02\\
    $\Delta z_B$ & 0.04 & 0.03 & 0.02 & 0.02\\
    $\chi^2_{B_{\mathrm{ROE}};\mathrm{IP}}$ & 0.05 & 0.05 & 0.01 & 0.00\\
    $x_{B_{\mathrm{ROE}}}$ & 0.03 & 0.03 & 0.01 & 0.02\\
    $z_{B_{\mathrm{ROE}}}$ & 0.03 & 0.03 & 0.01 & 0.05\\
\end{tabular}

\end{table}

\subsection{Continuum suppression training}\label{sec:continuum_training}

As it was argued in \Cref{sec:continuum_features}, events containing \BtoXsgamma decays may have slightly different kinematic properties compared to a generic-\BB.
Although these differences, as seen in \Cref{sec:appendix_continuum_features_datamc} are not large, training a classifier to separate generic $\FourS\ra\BB$ and \epem\ra\qqbar events may lead to a suboptimal separation of \BtoXsgamma.

A more effective setup is to remove \BB events from generic \MC and supplement the leftover events with \BtoXsgamma events from signal \MC. 
In such a scenario, the classifier learns to distinguish between the signal decays and continuum events, without the additional ambiguity of including generic \BB decays.
All selections from \Cref{tab:preselections} are employed for the training datasets.

The training samples are prepared by creating a mixture of $100000$ \epem\ra\qqbar events and 100000 \BtoXsgamma events from the signal MC sample.
In each event, one $\gamma$ and $\B_{\mathrm{tag}}$ candidate combination is randomly chosen.
This requirement ensures that the same event cannot contribute multiple training entries.
The target variable for the training is defined as a $\mathtt{flag}$, which folows
\begin{equation}
    \mathtt{flag}=\begin{cases}
      0,\quad \mathrm{for} \epem\ra\qqbar events,\\ 
      1,\quad \mathrm{for} \BtoXsgamma events.
      \end{cases}
\end{equation}
Half of the \epem\ra\qqbar training sample is taken \feiBp mode, and the other half is from \feiBz modes. 
For signal, half is taken from $\BptoXsgamma$ signal mode, the other half from $B^0$, irrespective of the \FEI mode.
Equivalent sample is prepared as the testing sample for the training.

The training is performed using a \texttt{FastBDT} algorithm, introduced in \Cref{sec:BDTs_theory}.
Four hyper-parameters within \texttt{FastBDT} framework have to be chosen.
Hyper parameter optimisation is performed in a grid-like search, based on two quantities:
\begin{align}\label{eq:optimisation_criteria}
    \begin{split}
    \mathtt{AUC}_{\mathrm{test}}&;\\
    \Delta \mathtt{AUC} \equiv |\mathtt{AUC}_{\mathrm{train}}& - \mathtt{AUC}_{\mathrm{test}}|,\\
    \end{split}
\end{align}
where $\mathtt{AUC}_{\mathrm{test}}$ is the \AUC score for training or testing samples.
A set of hyper-parameters is sought, such that $ \mathtt{AUC}_{\mathrm{test}}$ is maximised, while $\Delta \mathtt{AUC}$ is minimised.
The results of hyper-parameter optimisation are summarised in \Cref{tab:grid_search}.
Larger depths, number of trees values are not explored, to avoid non-linear correlations which may be learnt by the classifier and would require additional studies to pinpoint.
Large shrinkage values are undesired, to ensure a slower learning rate of the classifier.

\begin{table}[htbp!]
    \centering
    \caption{\label{tab:grid_search}Hyper-parameter optimisation based on a grid-search method.
    The four hyperparameters for the \texttt{FastBDT} algorithm are defined in \Cref{sec:BDTs_theory}.
    The optimal values are chosen based on criteria defined in \Cref{eq:optimisation_criteria}.
    They are shown in the right most column are taken as the parameters for the training.
    }
    \begin{tabular}{l|c|c|}
    Hyperparameter     & Tested grid values       & Chosen optimal parameter \\
    \hline 
    depth              & \{1,2,3\}                  & 2    \\
    number of trees    & \{100,200,400,600,1000\} & 400  \\
    shrinkage          & \{0.01,0.05,0.1,0.3,0.5\}          & 0.1  \\
    training subsample & \{0.2,0.4,0.5,0.6,0.8\}  & 0.8  \\
\end{tabular}

\end{table}

The training, with features from \Cref{tab:passing_test1} and hyper parameters from \Cref{tab:grid_search} is performed.
The normalised classifier output for test and train samples is shown in.
It is clearly seen that the classifier shows almost no bias, as the train and test samples agree very well.
This is further aluded by inspecting the ROC curve in.
The \AUC score is higher for the training sample (as expected in the case of a good training) but only by a minor amount.
Furthermore, the ROC curves themselves are practically indistinguishable.

\begin{figure}[htbp!]
    \subcaptionbox{\label{fig:separation_curve}}{
    \includegraphics[width=0.45\textwidth]{figures/continuum_suppression/separation_curves.pdf}
    }
    \subcaptionbox{\label{fig:roc_curve}}{
        \includegraphics[width=0.45\textwidth]{figures/continuum_suppression/roc_curve.png}
    }
    \caption{\label{fig:training_evaluation} The training evaluation for this analysis.
    \Cref{fig:separation_curve} shows excellent separation between \epem\ra\qqbar and \BtoXsgamma samples, and excellent agreement between corresponding test and train samples.
    \Cref{fig:roc_curve} shows the \ROC curve of the training. 
    The test and train sample \AUC scores are close to unity and similar, alluding to a high-separation power that is observed, and no evidence of overtraining.
    }
\end{figure}

Using the tools provided by the \texttt{FastBDT} algorithm, a \textit{relative feature importance} is computed.
Particularly for \texttt{FastBDT}, it is computed by evaluating the decrease of the \AUC score if the feature is not included in the training dataset (for more details see Ref.\cite{Keck:2017gsv}.)
Therefore, it can be considered a quantitative measure of the impact of a feature on the final classifier output.
The figure depicting relative training observable importances is shown in \Cref{fig:feature_importance}. 
It is clearly seen that cosTBTO and XX by far dominate, which is not surprising after inspecting their individual separation power for the current problem in \Cref{sec:appendix_continuum_features_datamc}.

\todo[inline]{wrap up here}

\begin{figure}[htbp!]
    \includegraphics[width=0.45\textwidth]{figures/continuum_suppression/feature_importance.pdf}
    \caption{\label{fig:feature_importance} The relative feature importances of different observables used in the training.
    The definitions of these observables are provided in \Cref{sec:appendix_continuum_features}.
    The feature importance highlights a relative change in \AUC score when the observable is not included in the training.
    }
\end{figure}

Finally, the effect of the same \textbf{Tests~1\&2} are performed for the photon energy spectrum to further test the validity of the training.


\section{Final selection optimisation}\label{sec:final_optimisation}
The last two Sections introduced the selection to suppress photons originating in non-\mbox{$\BtoXsgamma$} decays, particularly from \piz and $\eta$ decays (\Cref{sec:photon_selection}),
and the strategy to suppress \mbox{\epem\ra\qqbar} events that were the dominant component in the selected data set (\Cref{sec:continuum_suppression}).
Although a preselection was already developed to prepare an adequate training sample for the \BDT, a more optimal (`tighter') selection is desired to ensure the optimal efficiency and purity of the selected sample.
This Section describes the approach taken to find such optimal selection and calculate the efficiency loss for all applied selections.

\subsection{Simultaneous selection optimisation}\label{sec:simultaneous_optimisation}

After the pre-selection that prepared the data for training a \BDT in \Cref{tab:preselections}, a more robust strategy for tighter selections is developed.
In particular, \BDT output, \piVeto, \etaVeto and \ZMVA may be interconnected in the sense that applying the selection on one of them influences a selection on the others. 
To find an optimal selection point, each threshold is optimised in an iteration-based approach. 
At each step, one variable threshold is optimised to a value that gives the best figure-of-merit score,
while keeping the other requirements unvaried.
Then, this is repeated for other variables.
Each individual optimisation is equivalent to that shown in \Cref{fig:selection_optimisations} and uses figure-of-merit $\mathtt{FOM}_2$, defined in \Cref{eq:punzi_fom}.

In order to maximise the efficiency of the optimisation on correctly reconstructed events without adhering to a more strict definition at this stage, 
this procedure is performed on one randomly selected peaking tag-$B$ candidate per event ($\Mbc >5.27~\gevcc$) combined with the highest energy photon.
The starting point for each selection corresponds to the values in \Cref{tab:preselections}.
The starting $\mathtt{BDT~output}$ selection is chosen at 0.5. 
The \BtoXsgamma admixture of charged and neutral modes is used.
The \epem\ra\qqbar and generic \BB background events from \feiBp and \feiBz modes are merged.
This aims to reproduce `realistic' data conditions, where different background efficiencies may be observed due to different behaviours of \feiBp and \feiBz modes.

After performing the optimisation for each selection, the optimisation steps are repeated 9 more times.
The selections converge and do not vary after round 3 of optimisation.
The converged values are shown in \Cref{tab:interative_optimisation}.

\begin{table}[htbp!]
    \centering
    \caption{\label{tab:interative_optimisation} Optimal selections chosen for this analysis, based on the iterative approach described in \Cref{sec:simultaneous_optimisation}.
    The values for $\mathtt{BDT~output}$ and \ZMVA are chosen near those that are found optimal.
    For \piVeto and \etaVeto the choice is made based on the availability of data-simulation agreement studies performed at Belle II.
    At the time of preparing the analysis, 
    only studies with \piVeto and \etaVeto thresholds 
    up to 0.4 were performed (see \Cref{sec:piz_eta_calibration}).
    }
    \begin{tabular}{|l|c|c|}
        \hline
        Variable &    Figure-of-merit maximised at & Final chosen \\
        \hline
        \ZMVA                      & $>0.629$ & 0.6\\
        \piVeto                    & $<0.258$ & 0.4\\
        \etaVeto                   & $<0.036$ & 0.4\\
        $\mathtt{BDT~output}$      & $>0.798$ & 0.8\\
        \hline
\end{tabular}
\end{table}

For \piVeto and \etaVeto, the found selection is relatively tight, if inspecting \Cref{fig:bp_piveto,fig:bz_piveto,fig:bp_etaveto,fig:bz_etaveto}.
Furthermore, at the time of preparation of the analysis described, studies regarding the \piVeto and \etaVeto applicability to such tight selections were not available.
Therefore, it was decided to not tighten this selection further than the pre-selection value obtained in \Cref{tab:preselections}.
Repeating the study while keeping \piVeto and \etaVeto selection at 0.4 yields compatible results to those shown in \Cref{tab:interative_optimisation}.
Other selections are retained based on the optimal value from the initial 10 iterations.

\subsection{Summary and efficiency of all analysis selections}\label{sec:selection_summary}

\Cref{tab:cutflow} summarises all the selections and \BDT training results from \Cref{sec:photon_selection,sec:continuum_suppression,sec:final_optimisation},
and lists the final \BtoXsgamma candidate retention.
The retention, in this case, is defined as:
\begin{equation}\label{eq:loose_efficiency}
    r_{\mathrm{cand}} = \frac{N_{\BtoXsgamma}~\mathrm{candidates~after~cut}}{N_{\BtoXsgamma}~\mathrm{no~cut}},
\end{equation}
which is an approximation as it may include multiple tag-$B$ candidates.
In the table, the $\Mbc>5.27~\gevcc$ requirement is no longer applied and all tag-$B$ meson candidates are kept (i.e. high energy photon candidates may contribute more than once per event).

\begin{table}[htbp!]
    \centering
    \caption{\label{tab:cutflow} The summary table of all selections and their retentions, based on \Cref{eq:loose_efficiency}.
    The selections listed here are applied on official Belle II \feiBp and \feiBz samples, described in \Cref{sec:reconstruction_overview}.
    The columns show efficiency for \BtoXsgamma events, calculated on signal \MC, continuum and \BB events, both of which are calculated on generic \MC.
    It can be seen that continuum events are suppressed by roughly two orders of magnitude, whereas generic-\BB decays by more than an order of magnitude.
    }
    \centering
\begin{minipage}[c]{0.49\textwidth}
    \centering
    \feiBp mode reconstruction
    \resizebox{1\textwidth}{!}{
        \begin{tabular}{lrrr}
            \multirow{2}{*}{Selection}   & \BtoXsgamma & Continuum &    \BB events \\
                                         & \multicolumn{3}{c}{Retention}     \\       
            \hline                                        
            none                  & 1.0000 & 1.0000    & 1.0000 \\
            $E_{\gamma}$ rank $= 1$     & 0.9979 & 0.9660    & 0.9762 \\
            $\ZMVA>0.6$        & 0.9435 & 0.6543    & 0.6957 \\
            $\piVeto<0.4$ & 0.8309 & 0.2145    & 0.3140 \\
            $\etaVeto<0.4$  & 0.9212 & 0.7637    & 0.7676 \\
            $\mathtt{BDT~output}>0.8$   & 0.5615 & 0.0253    & 0.4854 \\
            tag-$\Mbc>5.245~\gevcc$         & 0.9485 & 0.8863    & 0.9287 \\
            \hline
            all                   & 0.4211 & 0.0045    & 0.0731 \\
            \end{tabular}
            
    }
\end{minipage}
\begin{minipage}[c]{0.49\textwidth}
    \centering
    \feiBz mode reconstruction
    \resizebox{1\textwidth}{!}{
        \begin{tabular}{lrrr}
            \multirow{2}{*}{Selection}   & \BtoXsgamma & Continuum &    \BB events \\
                                         & \multicolumn{3}{c}{Retention} \\
            \hline 
            none                  &1.0000 & 1.0000 & 1.0000 \\
            $E_{\gamma}$ rank $= 1$     &0.9982 & 0.9680 & 0.9791 \\
            $\ZMVA>0.6$        &0.9449 & 0.6570 & 0.6899 \\
            $\piVeto<0.4$ &0.8411 & 0.2221 & 0.3235 \\
            $\etaVeto<0.4$  &0.9272 & 0.7824 & 0.7739 \\
            $\mathtt{BDT~output}>0.8$  &0.5538 & 0.0251 & 0.4791 \\
            tag-$\Mbc>5.245~\gevcc$          &0.9461 & 0.8837 & 0.9230 \\
            \hline
            all                   &0.4206 & 0.0047 & 0.0735 \\
            \end{tabular}
    }
\end{minipage}
\end{table}

\Cref{tab:cutflow} shows that the background suppression procedure roughly halves the number of available \BtoXsgamma events in the sample.
However, the background candidates from \mbox{$\epem\ra\qqbar$} processes are reduced 200 times: to less than 0.5\% of the original value.
Furthermore, generic-\BB event contribution is estimated at 7\% of the original, which means more that an order of magnitude suppression is achieved.
The photon energy spectrum, after these selections have been applied is shown in \Cref{fig:spectrum_after_optimisation}.
Compared with the previous versions of this Figure, e.g.\Cref{fig:spectrum_after_reco}, a much better signal-to-background ratio is visible.

\begin{figure}[htbp!]
    \centering
    \subcaptionbox{\label{fig:spectrum_after_optimisation_bp}}{
        \includegraphics[width=0.45\textwidth]{figures/final_optimisation/Bp_tagged_background_optimal.pdf}
    }    
    \subcaptionbox{\label{fig:spectrum_after_optimisation_bz}}{
        \includegraphics[width=0.45\textwidth]{figures/final_optimisation/Bz_tagged_background_optimal.pdf}
    }    
    \caption{\label{fig:spectrum_after_optimisation}
    \BtoXsgamma spectrum in generic \MC after event reconstruction in \feiBp and \feiBz modes with optimal background suppression selections listed in \Cref{tab:cutflow}.
    Overlaid are events from signal \MC where the photon comes from \BtoXsgamma, multiplied by a scaling factor, with the same selections applied.
    These figures may include a high energy photon combined with multiple tag-$B$ entries per event and can be compared directly with \Cref{fig:spectrum_after_reco} where it is seen that
    the signal-to-background ratio for \BtoXsgamma is 100 times higher.}
\end{figure}


\section{Tag-side \texorpdfstring{\B}{B} candidate selection}\label{sec:tag_selection}
In \Cref{sec:reconstruction_overview} it was discussed that in about a half of all reconstructed events there exists more than one tag-side candidate.
That does not take into account the overlap between \feiBp and \feiBz mode -- which further enhances this effect.
Performing the best tag-side $B$ candidate selection is important, as multiple entries per event should not be included in the final sample.
However, the interest in this analysis lies in the signal side which decays as \BtoXsgamma, which means that a standard Belle II `truth-matching' procedure is too strict.
In principle, our requirement is to only reconstruct a sample of tag $B$ mesons that \textit{give provide good kinematic constraints to the signal side}.
In this section, these definition for best-candidate selection are discussed more broadly and
 a concrete definition for tag-$B$ mesons with correctly reconstructed kinematic properties is introduced.

\subsection{Selection within the same \texorpdfstring{\FEI}{FEI} mode}

The number of tag-side candidates for \feiBp and \feiBz modes, after the optimised selections in \Cref{tab:cutflow},
is shown in \Cref{fig:fei_tag_reco_candidates_post_optimisation}.
Overall, comparing to \Cref{fig:fei_tag_reco_candidates}, the candidate fractions are similar --
which attests to the fact that background (and particularly continuum) suppression was done without introducing a bias in preferentially selecting events with large \feiProb in \Cref{sec:continuum_suppression,sec:photon_selection}.

\begin{figure}[htbp!]
    \centering
    \includegraphics[width=0.45\textwidth]{figures/event_reconstruction/Bboth_total_tag_candidates.pdf}

    \caption{\label{fig:fei_tag_reco_candidates_post_optimisation} 
    Relative fractions of events for the number of \B meson candidates in the generic \MC dataset after the background suppression selections in \Cref{tab:cutflow}.
    This figure can be directly compared with \Cref{fig:fei_tag_reco_candidates}.
    The overall fractions are similar.
    About 67\%(74\%) of events for \Bp and \Bz \FEI modes have only 1 tag-side candidate.
    About 19\%(17\%) of events for \Bp and \Bz \FEI modes have two tag-side candidates and 7\%(5\%) has 3.
    The number of candidates per event reduces quickly, but faster for \Bz modes, with roughly 2\%(1\%) of events having more than 5 candidates for \Bp and \Bz.
    Note that the same event can have a \Bp and \Bz event reconstructed.
    }
\end{figure}

It is clear, however, that even after all selections there may still exist more than one \B meson + photon combination.
The first step in this is a selection of best-tag-candidate per \FEI mode, i.e. the best candidate between \feiBp and \feiBz modes.
While a general approach could be developed, it was observed that at this stage a particular choice of the tag does not influence the resolution or the average value of the spectrum strongly.
This is visualised in \Cref{fig:same_mode_best_tag_selection}.
For both neutral and charged \BtoXsgamma modes the distributions look similar whether the highest-\feiProb candidate is selected in each event, or a random tag-$B$ meson is chosen as the main candidate.
On the other hand, the \Mbc distribution, as expected, has a higher peak for the case when the highest \feiProb candidate is picked in each event.
The latter result for \BtoXsgamma is shown in \Cref{fig:same_mbc_best_tag_selection}.
The figure also includes a similar \Mbc test for the continuum events.

\begin{figure}[htbp!]
    \subcaptionbox{\label{fig:bp_same_mode_best_tag_selection}}{
        \includegraphics[width=0.4\textwidth]{figures/best_tag_selection/bp_spectrum_with_random_best_tag_selection.pdf}
    }
    \subcaptionbox{\label{fig:bz_same_mode_best_tag_selection}}{
        \includegraphics[width=0.4\textwidth]{figures/best_tag_selection/bp_spectrum_with_random_best_tag_selection.pdf}
    }
    \caption{\label{fig:same_mode_best_tag_selection}Photon energy spectrum (\Cref{fig:bp_same_mode_best_tag_selection}) after selecting a single tag $B$ meson candidate per event either randomly or by requiring the largest \feiProb.
    This is shown for \BptoXsgamma events in \Cref{fig:bp_same_mode_best_tag_selection} 
    and \BztoXsgamma in \Cref{fig:bz_same_mode_best_tag_selection}.
    The figures are normalised to their total integral value such that a shape comparison can be performed.
    Overall, the difference in resolution is small and evaluated at about \order(0.1\%), which is at least an order of magnitude smaller than expected statistical precision of the analysis.
    \todo[inline]{ref resolution chapter}.
    }
\end{figure}

\begin{figure}[htbp!]
    \centering
    \subcaptionbox{\label{fig:bp_mbc_mode_best_tag_selection}}{
        \includegraphics[width=0.4\textwidth]{figures/best_tag_selection/bp_Mbc_with_random_best_tag_selection.pdf}
    }
    \subcaptionbox{\label{fig:bz_mbc_mode_best_tag_selection}}{
        \includegraphics[width=0.4\textwidth]{figures/best_tag_selection/bz_Mbc_with_random_best_tag_selection.pdf}
    }
    \subcaptionbox{\label{fig:bp_continuum_mode_best_tag_selection}}{
        \includegraphics[width=0.4\textwidth]{figures/best_tag_selection/bp_continuum_Mbc_with_random_best_tag_selection.pdf}
    }
    \subcaptionbox{\label{fig:bz_continuum_mode_best_tag_selection}}{
        \includegraphics[width=0.4\textwidth]{figures/best_tag_selection/bz_continuum_Mbc_with_random_best_tag_selection.pdf}
    }
    \caption{\label{fig:same_mbc_best_tag_selection}\Mbc shapes for \BtoXsgamma signal \MC (\Cref{fig:bp_mbc_mode_best_tag_selection,fig:bz_mbc_mode_best_tag_selection}) 
    and \mbox{\epem\ra\qqbar} events from generic-\MC (\Cref{fig:bp_continuum_mode_best_tag_selection,fig:bz_continuum_mode_best_tag_selection}) after selecting a single tag $B$ meson candidate per event either randomly or by requiring the large \feiProb.
    \Cref{fig:bp_mbc_mode_best_tag_selection,fig:bp_continuum_mode_best_tag_selection} show that the difference in the \Mbc distribution for \BptoXsgamma and \BztoXsgamma mostly pronounced in the peak region.
    On the other hand, \Cref{fig:bp_continuum_mode_best_tag_selection,fig:bz_continuum_mode_best_tag_selection} shows not strong dependance in shape irrespective of the way the tag-side candidate is chosen.
    The figures are normalised to their total integral value such that a shape comparison can be performed.
    This observation motivates to selected the highest-\feiProb candidate.
    }    
\end{figure}

As it is desirable to emphasise the contrast between continuum and $B$ events for the fitting step that will follow (see XXX)
\todo[inline]{XXX}
the highest \feiProb candidate in each event is chosen as the $B$ candidate with virtually no bias to the resolution.
However, the study here, as of yet, does not address the cases when there is a candiate in the same event reconstructed in \feiBp and \feiBz.
Therefore, for now, both candidates are kept in such events and the study is continued in \Cref{sec:select_best_candidate}.

\subsection{Selection between \texorpdfstring{\feiBp}{feiB+} and \texorpdfstring{\feiBz}{feiB0} mode}\label{sec:select_best_candidate}

The last section showed that one can select the highest \feiProb candidate from \feiBp and \feiBz without a significant effect on the \EB resolution, and with an enhancement of the \Mbc distribution peak.
As such it reduced each event to a signle tag-$B$ and photon combination in most events.
However, it did not address the case when there is a candidate reconstructed in both \feiBp and \feiBz modes: implying that events may still have up to two combinations.
Such cases are evaluated to happen roughly 10.5\% of time. 
For the sample where two $B$ candidates exist, two quantities are calculated
\begin{equation}\label{eq:asymmetry_tag}
    \mathcal{A}_{\mathrm{tag}} = \frac{\mathcal{P}_{\mathrm{tag}}(\feiBp) - \mathcal{P}_{\mathrm{tag}}(\feiBz)}{\mathcal{P}_{\mathrm{tag}}(\feiBp) + \mathcal{P}_{\mathrm{tag}}(\feiBz)},
\end{equation}
which is called the asymmetry of \feiProb between a \feiBp and \feiBz candidate in the same event, and
\begin{equation}\label{eq:delta_mbc}
    \Delta(\Mbc) = \Mbc(\Bp) - \Mbc(\Bz),
\end{equation}
which is the difference in \Mbc value of the two candidates.
The $\mathcal{A}_{\mathrm{tag}}$ tends to zero if they both have a similar \feiProb and to $\pm$ unity if one of the candidates has a much larger \feiProb.
The second $\Delta(\Mbc)$ is a difference in \Mbc between both of the candidates.
These quantities are visualised in a two-dimensional grid in \Cref{fig:selecting_tag_mode}.
In the figure, the sample is split into two samples, where a real \Bp or \Bz hadronic decay is present on the tag-side, in \Cref{fig:bp_selecting_tag_mode} and \Cref{fig:bz_selecting_tag_mode}, respectively.
If a $\Bp$ candidate is present $\mathcal{A}_{\mathrm{tag}}\approx1$ and $\Delta(\Mbc)\gtrsim 0$ for majority of the candidates.
For $\Bz$ candidates the opposite is true: $\mathcal{A}_{\mathrm{tag}}\approx-1$ and $\Delta(\Mbc)\lesssim 0$ for majority of the candidates.
This result implies that if the true candidate is a \Bp(\Bz), then the value of \feiProb will be higher from \feiBp (\feiBz) candidates the opposite \FEI mode.
Furthermore, the fact that $\Delta(\Mbc)$ tends to be higher can simply be understood that the incorrect candidate is more likely to be present in the tail, rather than peak region of \Mbc.

\begin{figure}[htbp!]
    \centering
    \subcaptionbox{\label{fig:bp_selecting_tag_mode}}{
        \includegraphics[width=0.4\textwidth]{figures/best_tag_selection/Bp_selecting_between_bp_bz.pdf}
    }
    \subcaptionbox{\label{fig:bz_selecting_tag_mode}}{
        \includegraphics[width=0.4\textwidth]{figures/best_tag_selection/Bz_selecting_between_bp_bz.pdf}
    }
    \caption{\label{fig:selecting_tag_mode} A two-dimensional grid of $\mathcal{A}_{\mathrm{tag}}$ (\Cref{eq:asymmetry_tag})
    and $\Delta(\Mbc)$ (\Cref{eq:delta_mbc}) for events that have 2 $B$ candidates from \feiBp and \feiBz modes.
    Candidates with a real tag-side $B^+$ decay (\Cref{fig:bp_selecting_tag_mode})
    and a real tag-side $B^0$ decay (\Cref{fig:bz_selecting_tag_mode}) are shown.
    Most candidates lie near $\mathcal{A}_{\mathrm{tag}}\approx +1 (-1)$ and.
    $\Delta(\Mbc)$ value tends to be positive (negative) for $\Bp(\Bz)$ candidates.
    }
\end{figure}

Based on the observations discussed in this sections, it can be concluded that it is appropriate to select the \FEI candidate with the highest signal probability even when selecting between different \feiBp and \feiBz modes.
This selections finalises the best-candidate selection in this analysis as now it is ensured that every photon candidate corresponds to a single, unique tag-side $B$ meson candidate.

\subsection{Estimation of kinematically-consistent \texorpdfstring{$B$}{B}-meson}\label{sec:good_tag_definition}


\section{\texorpdfstring{\Mbc}{Mbc} fitting setup}\label{sec:fitting_mbc}
By this point in the analysis all the selections have been finalised and discussed.
Furthermore, clear definitions for tag-$B$ mesons that properly kinematically constrain \BtoXsgamma are presented, such that \EB is evaluated accurately.
However, as can be seen \Cref{fig:spectrum_after_optimisation}, even though continuum and $\BB$ backgrounds is suppressed heavily compared to the \EB distributions that was begun with (\Cref{fig:spectrum_after_reco}),
there is still a significantly larger number of background processes than \BtoXsgamma signal events.
Many of these, particularly continuum background, originate from incorrect tag-$B$ mesons (see \Cref{fig:good_tag_definitions}) and can therefore be estimated in data using an \Mbc fitting procedure.
In this section, a thorough overview of the \Mbc fit will be presented which will extract the counts of good-$B$ tags in different \EB intervals.

\subsection{Components in the dataset}\label{sec:fitting_components}



\section{Remaining \texorpdfstring{\BB}{BB} background subtraction}\label{sec:background_subtraction}
After the fitting procedure in \Cref{sec:fitting_mbc} the good tag-\B counts are estimated, and in particular:
\begin{itemize}
    \item all \epem\ra\qqbar contributions have now been removed;
    \item events where the tag-side \B mesons are not properly reconstructed have been removed.
\end{itemize}

However, it may seem odd that after performing the \Mbc fit (see \Cref{fig:mc_fit_yield_comparisons}), the result is still not quite comparable to an \EB spectrum,
resembling that of the background seen in, e.g., \Cref{fig:spectrum_after_optimisation}.
Since in the inclusive treatment the $X_s$ is not constrained, a component of \textit{irreducible} background will always be present.
In the case of this analysis, the good tag-\B counts after the optimal selection in \Cref{tab:cutflow} and the fitting procedure presented here contain:
\begin{itemize}
    \item correctly tagged \BtoXsgamma events,
    \item correctly tagged-\BB events other than \BtoXsgamma.
\end{itemize}
Although in the future versions of this analysis it may be possible to further diminish the second component through improved selection and fitting procedures (see discussion in \Cref{sec:future_prospects}), 
some irreducible background component will always remain.

The final step before the full \BtoXsgamma spectrum extraction in \MC is evaluating the size of the remaining \BB background.
Two strategies are considered:
\begin{itemize}
    \item Count good tag-\B mesons in each \EB interval that correspond to non-\BtoXsgamma in \MC and use these values to subtract remaining \BB events from the fitted values.
    This is a simple method which is independent of the fitting procedure.
    \item Perform the fit again on the simulated sample with \BtoXsgamma events removed.
    This way, good tag-\B meson counts are estimated with fitting effects included.
    This effectively corresponds to the data points in \Cref{fig:no_bxsgamma_mc_fit_yield_comparisons}.
\end{itemize}
Although both choices are valid and have certain advantages, in this analysis, the second method is chosen.
The main reason for this choice is the fact that biases or systematic effects in the fit result, if such exist, could be suppressed when subtracting.
It also requires a similar validation procedure (e.g. to be an unbiased estimator) as the total fit.
Therefore, after the \Mbc fit, the number of correctly-tagged \BtoXsgamma events in data will be measured as follows:
\begin{equation}\label{eq:background_subtraction}
    N_{\mathrm{DATA}}^{\BtoXsgamma} = \mathcal{N}_{\mathrm{DATA}}^{\mathrm{good~tags}}  - \frac{189~\invfb}{1600~\invfb}\cdot \mathcal{N}_{\mathrm{MC}}^{\mathrm{good~tags~with~\BtoXsgamma~removed}}.
\end{equation}

\section{Analysis strategy validation in simulation}\label{sec:MC_validation}
\Cref{sec:fitting_mbc,sec:background_subtraction} introduced the fitting procedure and \BB background subtraction.
Together with the optimal selections from \Cref{sec:final_optimisation}, this fully defines the analysis strategy from the Belle~II simulated datasets to the \BtoXsgamma spectrum.
However, the defined fitter has to be validated in simulation to give an unbiased estimation of \BtoXsgamma events, with a good resoltion and signal efficiency.
The studies in this Section will show such results.

\subsection{Validation of \texorpdfstring{\Mbc}{Mbc} fit on reduced sambple size}\label{sec:mbc_fit_validation_misreconstructed}

The results that were shown in \Cref{fig:mc_fit_yield_comparisons} only provided results for fitting 1.6~\invab -- a dataset about an order of magnitude larger than is expected in the case of this analysis.
Therefore, the generic \MC dataset is pseudorandomly split into 10 smaller subsets, corresponding to 160~\invfb, and each of them are fitted independently.
The choice of 160~\invfb, rather than 190~\invfb which is the sample size of the Belle~II data used in the analysis, is due to anticipated data-simulation differences.
\todo[inline]{see XXX and next paragraph performed where?}
Indeed, a 190~\invfb dataset should correspond to roughly 160~\invfb in simulation due to differences in tag-\B reconstruction efficiency.

The resulting 10 fits and the estimated $\mathcal{N}_{CB}$ corresponding to each \EB bin are shown in \Cref{fig:extracted_validation_mc}.
The expected number of events in each fit is always equal to one tenth of that in the total generic \MC dataset.
It can be observed that all data points, and their average, are statistically compatible with the expectation.
These results indicate, that despite using a 10 times larger dataset to define the \PDF{s}, this \Mbc fit model produces reliable results.
Further tests, particularly a test ensuring that the fitter is unbiased, are performed later.

\begin{figure}[htbp!]
    \includegraphics[width=0.9\textwidth]{figures/mc_validation/extracted_signal_generic_mc.pdf}
    \caption{\label{fig:extracted_validation_mc}The estimated $\mathcal{N}_{CB}$ values from fits on one tenth of generic \MC, corresponding to 160~\invfb of simulation.
    The dashed lines represent different \EB bins, each bin showing one data point corresponding to a simultaneous fit of all \EB bins.
    The dotted lines show the average of all 10 points in each bin, whereas the full lines show the number of good tag-\B events in the original 1.6~\invab dataset, scaled down 10 times (`expected').
    The subpanels show the pull of each datapoint from the expected number of events.
    These results show that the fit is able to extract a result on a dataset that is an order of magnitude smaller.
    }
\end{figure}

\subsection{Validation of subtraction of remaining-\texorpdfstring{\BB}{BB} background}\label{sec:background_subtraction_validation_mc}

\begin{figure}[htbp!]
    \includegraphics[width=0.9\textwidth]{figures/mc_validation/subtracted_signal_generic_mc.pdf}
    \caption{\label{fig:subtracted_validation_mc}
    The estimated $\mathcal{N}_{CB}$ with subtracted background.
    }
\end{figure}


\section{Simulation-to-data corrections}\label{sec:corrections}
The fitter and background subtraction procedure, introduced in \Cref{sec:fitting_setup,sec:background_subtraction},
have been thoroughly validated in \Cref{sec:MC_validation} -- in simulation.
The real challenge, as usual, is ensuring that the conclusions and results observed in simulation will generalise correctly to real Belle II data.
The key concept of a blinded analysis dictates that one must validate the analysis procedure in control samples or regions -- collections of data that are abundant, well-understood and provide insight to the behaviour of signal in the detector.
In this Section, the discussion about \FEI validation, \piz and $\eta$ veto validation, photon detection efficiency and background modelling will be presented.

\subsection{Calibration of the \texorpdfstring{\FEI}{FEI} algorithm}\label{sec:fei_calibration}

The working principle of \FEI has already been discussed in XXXX
\todo[inline]{xxxx}.
It combines many classifiers which perform reconstructions in various decay chains of the hadronic decays of \B mesons.
Furthermore, the training of the algorithm happens in simulation.
To ensure that the algorithm appropriately acts on data, its performance on Belle II data must be studied, or \textit{calibrated}.
The calibration study is performed on data collected by the Belle~II, for every simulation campaign, and the work is not part of my original work.
Full details of the calibration method are presented in Ref.\cite{Belle-II:2020fst}, but here I will summarise the main details that are relevant to the work of the thesis.

The calibration study uses $B\rightarrow X_{u,c} \ell \nu$ decays, due to branching fraction of almost 20~\% and a clean experimental signature, 
where $X_{u,c}$ denotes an inclusive state originating from the $c$ or $u$ quark, similarly to the $X_s$ notation.
Firstly, in each event, only the highest \FEI probability tag-\B candidate is selected, with loose requirements on Fox-Wolfram moments (see \Cref{sec:fox_wolfram_moments}) and $\Delta E$ to ensure adequate \epem\ra\qqbar suppression.
Next, a high-energy lepton $p_{\ell}^B>1~\gev$ is required in each event.
This lepton candidate is required to originate near the interaction point and its identification information from all sub-detectors is required to be consistent with a lepton.

After the selection, a binned likelihood fit for \Mbc is set up, which contains three binned \PDF{s}: signal $B\rightarrow X_{u,c}\ell\nu$ decays, 
secondary or misidentified leptons, \epem\ra\qqbar events. Here secondary leptons are used to describe leptons that arise in the decay chains of $B$ meson as opposed to the $B\rightarrow X_{u,c}\ell\nu$ decay.
Misidentified leptons are used as a broad term for hadrons whose identification information is consistent with that of either an electron or a muon.
The signal $B\rightarrow X_{u,c}\ell\nu$ \PDF is composed of four sub-\PDF{s}, particularly: $B\rightarrow D\ell\nu$, $B\rightarrow D^*ell\nu$, $B\rightarrow X_u\ell\nu$ and the rest of  $B\rightarrow X_c\ell\nu$ modes.
The fit is performed separately for the following combinations of tag-\B mesons and lepton:
\begin{itemize}
    \item $B^+$ and $e^-$,
    \item $B^+$ and $\mu^-$,
    \item $B^0$ and $e^-$,
    \item $B^0$ and $\mu^-$.
\end{itemize}
This is shown in \Cref{fig:fei_calib}.
\begin{figure}[htbp!]
    \subcaptionbox{\label{fig:fei_calib_bpluseminus}}{
        \clipbox*{{0\width} {0.5\height} {0.5\width} {1\height}}{%
            \includegraphics[width=1\textwidth]{figures/data_sim_corrections/tag_calibration.png}
        }
    }
    \subcaptionbox{\label{fig:fei_calib_bplusmuminus}}{
        \clipbox*{{0.5\width} {0.5\height} {1\width} {1\height}}{%
            \includegraphics[width=1\textwidth]{figures/data_sim_corrections/tag_calibration.png}
        }
    }
    \subcaptionbox{\label{fig:fei_calib_bzeroeminus}}{
        \clipbox*{{0\width} {0\height} {0.5\width} {0.5\height}}{%
            \includegraphics[width=1\textwidth]{figures/data_sim_corrections/tag_calibration.png}
        }
    }
    \subcaptionbox{\label{fig:fei_calib_bzeromuminus}}{
        \clipbox*{{0.5\width} {0\height} {1\width} {0.5\height}}{%
            \includegraphics[width=1\textwidth]{figures/data_sim_corrections/tag_calibration.png}
        }
    }
    \caption{\label{fig:fei_calib} Illustration of the fits to \B\to$X_{u,c}\ell\nu$ decays in the \FEI calibration study.
    Results for the combinations of charged and neutral tag-$B$ modes with $e^-$ and $\mu^-$
    are shown in \Cref{fig:fei_calib_bpluseminus,fig:fei_calib_bplusmuminus,fig:fei_calib_bzeroeminus,fig:fei_calib_bzeromuminus}.
    Different fit components are shown in the legend and the subpanels contain the pulls of the fit.
    Figures taken from \cite{Belle-II:2020fst}.
    }
\end{figure}

Branching fractions of $B\rightarrow X_{u,c}\ell\nu$ are evaluated from the fitted distributions.
These values are then directly compared with the experimentally known values of branching fractions of these decays.
A correction factor, $\mathcal{C}_{\mathrm{FEI}}$ is derived, such that the two values are compatible.
The leading evaluated systematic uncertainties are found to be the imperfect experimental knowledge of the $B\rightarrow X_u\ell\nu$ branching fractions and their form factors, the fit model composition, tracking and particle identification uncertainties.
For the Belle II simulation campaign used in this analysis and averaged for both lepton modes, the result is as follows:
\begin{equation}\label{eq:fei_calibration}
    \mathcal{C}_{\mathrm{FEI}}(B^+) = 0.6599 \pm 0.225 \quad \mathcal{C}_{\mathrm{FEI}}(B^0) = 0.6695 \pm 0.0237,
\end{equation}
where two different calibration factors are presented for \feiBp and \feiBz modes, respectively.
Therefore, for an adequate comparison with Belle II data, any Belle II simulation involving the use of \FEI will be henceforth scaled appropriately.

\subsection{Calibration of \texorpdfstring{\piz}{pi0} and \texorpdfstring{$\eta$}{eta} suppression tools}\label{sec:piz_eta_calibration}
It was seen in \Cref{sec:selection_vetos}, that one of the strongest tools for background suppression in this analysis is the $\piz$ and $\eta$ suppression tool.
Consequentially, any data-simulation discrepancies will have a high impact on the final result.
The calibration of the \piz and $\eta$ is performed in an independent study and is not part of original work presented in this thesis.
For clarity, the calibration study is discussed in this Section.
\etaVeto calibration is henceforth implied, although only \piVeto is mentioned, as their calibration is equivalent.
Althrough the calibration study only studies \piVeto, it is assumed that the corrections are also valid for \etaVeto selections.

The main concern for this analysis is the primary (signal) photon efficiency: the number of photons that do not originate in light-meson decays and get rejected given a certain \piVeto selection.

The calibration study uses $B^+\to \bar{D}^0[\to K^+\pi^-]\pi^+$ decays and $B^0\to D^-[\to K^+\pi^-\pi^-]\pi^+$, where the square brackets denote a subdecay of the $D$ meson.
The $pi^+$, originating in the primary $B$ decay, is combined with all other photon candidates in the event, in a strategy described in \Cref{sec:selection_vetos}.
This produces many `\piz'-like combinations ($\mathit{pseudo-\piz}$) which yield a \piVeto score with minimal background from real \piz decays.

The reconstruction requires all charged tracks to have good-quality tracks that originate near the interaction point.
The identification information from Belle II subdetectors is used to distinguish pions and kaons.
Because $\pi^+$ from the primary $B$ decay is combined with other photons, a massless hypothesis is used for calculations of invariant mass and the helicity angles for the MVA.
After constructing the pseudo-\piz, selections on \piVeto are performed, corresponding to selections chosen in the analysis.
Therefore a distribution with no \piVeto selection, and a subset distribution with $\piVeto<0.4$ are created.
In both cases the charged and neutral $B$ channels are combined.

An unbinned \Mbc fit is performed on distributions with and without the \piVeto selections.
The \Mbc is modelled by a Crystal Ball function for signal decays and ARGUS function for continuum background.
All parameters of Crystal Ball and continuum are unconstrained.
An additional \PDF to model the peaking non-signal \BB components are used.
This \PDF is initialised in simulation, as a sum of a Gaussian and an Argus.
The shape parameters, and normalisation of this \BB background \PDF are not estimated, but kept at the initialised values.
The fits on data in the case of no \piVeto selection, and a $\piVeto<0.95$ selection are given in \Cref{fig:pivetofit}.
\begin{figure}[htbp!]
    \centering
    \subcaptionbox{\label{fig:pivetofit_nocut}}{
    \includegraphics[width=0.4\textwidth]{figures/data_sim_corrections/data_fit_pi0_no_cut.png}
    }
    \subcaptionbox{\label{fig:pivetofit_cut}}{
        \includegraphics[width=0.4\textwidth]{figures/data_sim_corrections/data_fit_pi0_with_cut.png}
        }
    \caption{\label{fig:pivetofit} The fit estimating the number of $B\to D \pi^+$ events in Belle II data.
    Different \PDF{s} used in the fit are shown in the legend and explained in text.
    The fit is performed on a sample without (\Cref{fig:pivetofit_nocut}) and with (\Cref{fig:pivetofit_cut}) \piVeto selection applied.
    The extracted values from simulation and data are then combined to calculate \piVeto correction factors (see \Cref{eq:piveto_correction}).
    These figures are produced by a Belle II internal study of the \piz veto, and not part of the original work in this thesis.
    Only the labels and legends have been adapted.    
    }
\end{figure}

The fit extracts the counts of $B\to D\pi^+$ events, $N_{B\to D\pi^+}$ as the normalisation parameter of the Crystal Ball.
An efficiency, $\varepsilon\equiv N_{B\to D\pi^+}/N^{\piVeto<0.4}_{B\to D\pi^+}$ is defined, which corresponds to the primary-photon efficiency.
If the fit is performed in simulation (\MC) and data, an efficiency ratio can be used as a correction factor:
\begin{equation}\label{eq:piveto_correction}
    R_{\piVeto} = \frac{N_{B\to D\pi^+}/N^{\piVeto<0.4}_{B\to D\pi^+}|_{\mathrm{data}}}{N_{B\to D\pi^+}/N^{\piVeto<0.4}_{B\to D\pi^+}|_{\mathrm{MC}}}.
\end{equation}
The corrections are calculated in $200~\mev$ intervals of the lab-frame energy of the primary $\pi^+$.
The results for the selection chosen in this analysis are given in \Cref{fig:piveto_corrections}.
The internal Belle II study providing these corrections was only performed in the range of 1.8 to 3.0 ~\gev in the laboratory frame energy of the primary $\pi^+$.
Therefore, a linear extrapolation based is performed to estimate the value for values outside the range.
It is observed that the linear extrapolation is consisttent, within errors, with the corrections in the $1.8-3.0~\gev$.
Therefore, a correction factor of $1.1\pm0.5$ is chosen for events outside the range covered by the calubration study.
This value is consistent with all other correction factors.
\begin{figure}[htbp!]
    \centering
    \includegraphics[width=0.45\textwidth]{figures/data_sim_corrections/pi0veto_corrections.pdf}
    \caption{\label{fig:piveto_corrections} The corrections $R_{\piVeto}$ for the $\piVeto<0.4$ selection used in this analysis.
    The results cover $1.8-3.0~\gev$ energies in the \textit{laboratory frame}, $\Egamma$.
    Because the laboratory frame energies cannot be trivially transformed to the $B$ meson rest-frame energies, a linear extrapolation to lower energies is performed.
    For missing $\Egamma$ phase space, in this analysis a $1.1\pm0.5$ value is adopted.
    }
\end{figure}

\subsection{Belle II calorimeter photon detection efficiency}\label{sec:photon_efficiency}

One of the main necessities of a measurement involving photons in the final state, is, of course, a precise and accurate simulation of the electromagnetic calorimeter -- the main tool of neutral particle detection.
In Belle~II, the \ECL provide this function.
Although it is known that its precision and resolution are excellent suited for flavour physics studies, exact data/simulation differences have to be known.
As part of the original work presented in this thesis, a Belle~II calorimeter photon detection efficiency study has been performed.
The initial setup of the analysis has been prepared by Dr. Natalia Kovalchuk and Dr. Torben Ferber, but I have updated the procedure, reworked the covered
and performed the full systematic uncertainty evaluation during my doctoral studies.
It is a critical study for the \BtoXsgamma analysis, however, the results of this study are also routinely used in other analyses that utilise photons in their final states.
The study is summarised in a Belle II public note \cite{Henrikas:2604}.
Here, the main measurement concepts and the results relevant to the measurement of \BtoXsgamma are presented.

To measure the photon detection efficiency, one must first have the knowledge that a photon has been created in an event, and then search for it within the calorimeter.
In this efficiency study $\epem\ra\mumu$ scattering events are employed.
In particular, collision events where a high energy photon is radiated in the initial state are sought.
The concept of the efficiency measurement is sketched in \Cref{fig:photon_efficiency_measurement}.
\begin{figure}[htbp!]
    \centering
    \subcaptionbox{\label{fig:eemumugamma_feynman}}{
    \includegraphics[width=0.4\textwidth]{figures/data_sim_corrections/eemumugamma.png}
    }
    \subcaptionbox{\label{fig:measurement_principle_gammaeff}}{
        \includegraphics[width=0.4\textwidth]{figures/data_sim_corrections/measurement_principle.png}
        }
    \caption{\label{fig:photon_efficiency_measurement} The main concepts of the measurement of the Belle II photon detection efficiency.
    The \Cref{fig:eemumugamma_feynman} shows the Feynman diagram of the $\epem\ra\mumu$ events, where an photon is radiated in the initial collision state.
    Due to the radiated photon, the resulting dimuon system will have missing momentum with respect to the usual collision energy $\sqrt{s}\approx10.58~\gev$.
    The direction of the missing momentum can be extrapolated to search for photon clusters in the calorimeter, as sketched in \Cref{fig:measurement_principle_gammaeff}.
    \Cref{fig:measurement_principle_gammaeff}
    }
\end{figure}

The main goal is to reconstruct two muon tracks in each event and evaluate their total momentum and energy.
Because the high-energy photon (further called initial-state radiation or \ISR) is created before the collision, the energy of the dimuon system will have a certain degree of missing-momentum, whose direction will coincide with that of the emitted photon.
This missing momentum direction is called \textit{recoil}, and is charactarised by the magnitude, polar angle and the azimuthal angle.
Therefore, by selecting events where such recoil are present, one is able to look for photon clusters corresponding to the angle and the energy within the calorimeter.
This gives a good efficiency estimate through a simple counting relation:
\begin{equation}\label{eq:photon_efficiency}
    \varepsilon_{\gamma}(|\vec{p}_{\mathrm{recoil}}|, \theta_{\mathrm{recoil}}, \phi_{\mathrm{recoil}}) = \frac{N(\mathrm{photon~found} \& \mathrm{recoil~found})}{N(\mathrm{recoil~found})},
\end{equation}
which can be evaluated as a function of missing momentum, with a magnitude $|\vec{p}_{\mathrm{recoil}}|$ and corresponding angles $\theta$ and $\phi$.




\section{Background validation studies}\label{sec:validation}
Up until now the disucssion of the analysis revolved around simulation studies.
The full analysis procedure on simulation was defined and shown to produce an unbiased, stable result in \Cref{sec:MC_validation}.
Then, \Cref{sec:corrections} looked at the corrections required to accurately correct differences expected in simulation to better represent data.
At this stage, all appropriate measures have been taken and the analysis can be fully applied on Belle II data.
The concepts of a blinded analysis, however, dictate that to ensure no biases are present the full analysis procedure must be performed in validation samples.
For the \BtoXsgamma analysis four validation samples are defined:
\begin{itemize}
    \item \epem\ra\qqbar sample, that contains collision data, collected 60~\mev below the \FourS center-of-mass energy.
    \item Sample with enhanced \epem\ra\qqbar, where the \texttt{BDT~output} score requirement is inverted: $\mathtt{BDT~output}<0.4$.
    \item Sample with enhanced \BB background, where the $\piz$ and $\eta$ veto requirements are inverted.
    \item $1.4<\EB<1.8$~\gev and $2.7<\EB$~\gev regions, where signal-to-background ratio is small, and signal is kinematically forbidden, respectively.
    Note that some signal events may still be present in $2.7<\EB$~\gev due to resolution effects, but they are not expected to be statistically significant.
\end{itemize}
In this Section, the analysis selections, \Mbc fitting and leftover-\BB background subtraction will be investigated thoroughly using these samples.

\subsection{Validation on the \texorpdfstring{\epem\ra\qqbar}{e+e- -> qqbar} off-resonance sample}\label{sec:continuum_spectrum_validation}

The validation on \epem\ra\qqbar events is performed using only \epem\ra\qqbar simulation.
The goal of this validation is to ensure that continuum backgrounds are described by the simulated samples correctly.
Although \Cref{sec:continuum_validation} partially ensures this, only the distribution shape requirements were tested there.
Furthermore, the best candidate selection, which was developped on \BB samples, may change the conclusions that were found earlier.

All the corrections for \piz and \eta veto, \FEI calibrations are applied as discussed in \Cref{sec:corrections}.
Full analysis selection procedure involving the most optimal selections and the best-tag side candidate selection are applied, as presented in \Cref{sec:selection_summary,sec:select_best_candidate}.
The results are shown in \Cref{fig:offresonance_validation}.
\begin{figure}[htbp!]
    \subcaptionbox{\label{fig:bplus_offresonance_eb}}{
        \includegraphics[width=0.31\textwidth]{figures/data_validation/Bplus_offresonance_EB.pdf}
    }
    \subcaptionbox{\label{fig:bzero_offresonance_eb}}{
        \includegraphics[width=0.31\textwidth]{figures/data_validation/Bzero_offresonance_EB.pdf}
    }
    \subcaptionbox{\label{fig:bboth_offresonance_eb}}{
        \includegraphics[width=0.31\textwidth]{figures/data_validation/Bboth_offresonance_EB.pdf}
    }
    \caption{\label{fig:offresonance_validation} Validation of the \EB distribution of \epem\ra\qqbar events.
    Excellent agreement is observed in \feiBp (\Cref{fig:bplus_offresonance_eb}, \feiBz (\Cref{fig:bzero_offresonance_eb}),
    and the combined sample (\Cref{fig:bboth_offresonance_eb}).
    The uncertainty for data contains only the statistical component.
    The simulation uncertainty contains statistical and systematic uncertainties corresponding to \Cref{tab:correction_table}.
    }
\end{figure}

Overall, the agreement of continuum data and simulation is excellent.
Due to rather low size of the off-resonance data sample and the strong continuum suppression in this analysis, the amount of statistical uncertainties are relatively large.
It is concluded, that the \EB distribution of \epem\ra\qqbar events is well-modelled in simulation and follows the expectations seen in earlier Sections.

Due to known issues with beam energy values in the off-resonance data, which affected the \Mbc calculation (but not overall validity of other values), 
the \epem\ra\qqbar off-resonance \Mbc distributions do not accurately represent the \Mbc values of continuum events.
Therefore, these samples are not used for \Mbc distribution and \Mbc fitting validation.

\subsection{Validation on \texorpdfstring{\epem\ra\qqbar}{e+e- -> qqbar} enhanced sample}\label{sec:continuum_mbc_validation}

As it was mentioned in \Cref{sec:continuum_spectrum_validation}, due to problems in \Mbc distribution, 
the dedicated \epem\ra\qqbar-only samples were not used to validate the continuum simulation.
However, an alternative validation sample is prepared, where the continuum component is enhanced.
This is achieved by inverting the \texttt{BDT~output} selection (see \Cref{sec:continuum_suppression}) thereby suppressing \BB events.
To ensure a minimal amount of \BtoXsgamma events in the sample (as the \texttt{BDT~output} is not perfectly efficient), here \piVeto and \etaVeto
are also inverted.
This creates a sample with mostly \epem\ra\qqbar events and small components of \BB events.
The inverted values are chosen as $\mathtt{BDT~output}<0.2$ and $\piVeto>0.4$ and $\etaVeto>0.4$.
The resulting \EB spectra for both \FEI modes are shown in \Cref{fig:qqbar_enhanced_eb_validation}.
\begin{figure}[htbp!]
    \subcaptionbox{\label{fig:bplus_qqbar_enhanced_eb}}{
        \includegraphics[width=0.31\textwidth]{figures/data_validation/Bplus_qqbar_enhanced_eb.pdf}
    }
    \subcaptionbox{\label{fig:bzero_qqbar_enhanced_eb}}{
        \includegraphics[width=0.31\textwidth]{figures/data_validation/Bzero_qqbar_enhanced_eb.pdf}
    }
    \subcaptionbox{\label{fig:bboth_qqbar_enhanced_eb}}{
        \includegraphics[width=0.31\textwidth]{figures/data_validation/Bboth_qqbar_enhanced_eb.pdf}
    }
    \caption{\label{fig:qqbar_enhanced_eb_validation} The \EB distribution of \qqbar enhanced samples (see \Cref{sec:continuum_spectrum_validation}).
    Adequate agreement is observed in \feiBp (\Cref{fig:bplus_offresonance_eb}, \feiBz (\Cref{fig:bzero_offresonance_eb}),
    and the combined sample (\Cref{fig:bboth_offresonance_eb}).
    The uncertainty for data contains only the statistical component.
    The simulation uncertainty contains statistical and systematic uncertainties corresponding to \Cref{tab:correction_table}.
    }
\end{figure}

Although the agreement is generally adequate, particularly in the signal region, although a small excess of events is observed in low-\EB region.
Similarly, the resulting \Mbc distributions are shown in \Cref{fig:qqbar_enhanced_mbc}.
A striking difference from the generally good agreement observed so far can be seen for \feiBp, \feiBz and the combined sample.
A larger amount of continuum events, particularly at low-\Mbc is present and,
moreover, the high-endpoint of the \Mbc is shifted.
The overall data-to-simulation discrepancy can be at around twenty percent.
\begin{figure}[htbp!]
    \centering
    \subcaptionbox{\label{fig:Bplus_qqbar_enhanced_mbc}}{
        \includegraphics[width=0.31\textwidth]{figures/data_validation/Bplus_qqbar_enhanced_mbc.pdf}
    }
    \subcaptionbox{\label{fig:Bzero_qqbar_enhanced_mbc}}{
        \includegraphics[width=0.31\textwidth]{figures/data_validation/Bzero_qqbar_enhanced_mbc.pdf}
    }
    \subcaptionbox{\label{fig:Bboth_qqbar_enhanced_mbc}}{
        \includegraphics[width=0.31\textwidth]{figures/data_validation/Bboth_qqbar_enhanced_mbc.pdf}
    }
    \caption{\label{fig:qqbar_enhanced_mbc}
    The \Mbc distribution of \qqbar enhanced samples (see \Cref{sec:continuum_spectrum_validation}).
    Some clear differences in \feiBp (\Cref{fig:Bplus_qqbar_enhanced_mbc}), \feiBz (\Cref{fig:Bzero_qqbar_enhanced_mbc}),
    and the combined sample (\Cref{fig:Bboth_qqbar_enhanced_mbc}) can be observed.
    Particularly, it is evident that there are more continuum events at low-\Mbc, 
    and the \Mbc high-endpoint is shifted to lower values.
    These results motivate the modification of the \Mbc fitting procedure, 
    summarised in \Cref{tab:fitting_init_params_updated}.
    }
\end{figure}

These differences are understood as a result of two reasons related to the fact that data-taking period independent simulation is used in this analysis.
In normal data collection conditions, the collision energy, $\sqrt{s}$ is not perfectly stable: minute variations or drifts can occur over time.
These can be related to many reasons including intentional, such as various tests, and unintentional, such as 
The data-taking period independent simulation does not account for these changes, with a collision energy simply set to a predetermined target value.
The \Mbc endpoint is directly affected by the set $\sqrt{s}$, as seen in \Cref{eq:mbc_exclusive}, with lower values of $\sqrt{s}$ making some values of \Mbc kinematically forbidden.
On the other hand, a larger overall amount of \qqbar events is understood as a consequence of the fact that collecting the data at lower collision energies (but now lower than \FourS energy) enhances the \epem\ra\qqbar process cross-section.
Altogether, this leads to more continuum events present in the data sample than predicted by the data-taking period independent simulation.

Note that for correctly reconstruced \B mesons (i.e. good tag-\B mesons) the shift would not occur.
This is a result of the fact that $p_B^*$ (as seen in \Cref{eq:mbc_exclusive}) is directly related to the total energy of the collision.
Lower collision energies will simply lower $p_B^*$, as the resonant-like behaviour in \Mbc is driven by the \B meson mass.
Such constraints are not present for misreconstructed events, therefore shifts are expected to happen there.

Although the most robust solution is the usage of data-taking period independent simulation, at the time of preparation of this analysis such simulation was noy yet fully available at Belle II.
While future studies will be able to rely on it, in this analysis additional steps were taken to account for this.

In particular, while \Mbc is strongly affected, the \EB spectrum is still well-described.
This is a consequence of the fact that \EB and \Mbc are not strongly correlated, or more plainly, \EB does not depend as strongly on $\sqrt{s}$ as \Mbc.
Therefore, a correction is only necessary for \Mbc distribution \textit{and} only for combinatorial-\BB and continuum events.
An \textit{ad hoc} approach is developed, where the \Mbc distribution for simulated events only is shifted manually. 
The procedure is as follows:
\begin{itemize}
    \item Count the frequencies of each $\sqrt{s}$ value occurring in the Belle II on-resonance dataset.
    \item Randomly remove half of the beam energies in simulated Belle II dataset of \textit{events where no good tag-\B mesons are present}.
    \item Replace the removed beam energies with the values of the first step, based on the frequencies they occur at in the Belle~II on-resonance dataset.
\end{itemize}
The reason why only $50\%$ of energies are replaced is to minimise any potential bias that such a procedure could introduce.
Note that the replacement of $\sqrt{s}$ only affects the \Mbc calculation and not other observables which may be rely on $\sqrt{s}$ in their definition.
The result of the correction on the \Mbc distribution of the sample of both \FEI modes combined is shown in \Cref{fig:qqbar_enhanced_mbccorrected}.
\begin{figure}[htbp!]
    \centering
    \includegraphics[width=0.45\textwidth]{figures/data_validation/Bboth_qqbar_enhanced_mbccorrected.pdf}
    \caption{\label{fig:qqbar_enhanced_mbccorrected} The \Mbc distribution of \qqbar enhanced samples (see \Cref{sec:continuum_spectrum_validation}),
    where an \Mbc correction has been applied to the simulated distribution.
    Although the correction does not perfectly correct the distributions, the signal region ($\Mbc\approx5.28~\gevcc$) is described correctly.}
\end{figure}
Although perfect correction is not achieved via this \textit{ad hoc} correction, but the peak-region is described correctly, which is evident when comparing \Cref{fig:qqbar_enhanced_mbccorrected} and \Cref{fig:qqbar_enhanced_mbc}.

A key point to discuss here is the effect the different \Mbc shape may have on the \Mbc fitter.
The differences in the tail, and the end-point are expected to not strongly affect the result, because the \Mbc fitter is prepared with shape differences and accounts for them
by estimating parameters $c$ and $m_0$ of the argus distribution, as well ass the overall $\frac{\mathcal{N}_{\mathrm{CHEB}}}{\mathcal{N}_{\mathrm{ARGUS}}}$ ratio.
However, the \Mbc fitter initial parameters in \Cref{tab:fitting_init_params} have to be updated to emphasise that a different `corrected'-\Mbc is the new fitting observable.
The \Mbc fitter is therefore updated, following the exact same procedures as \Cref{sec:fitting_setup} and the new values are given in \Cref{tab:fitting_init_params_updated}.
\begin{table}[htbp!]
    \centering
    \caption{\label{tab:fitting_init_params_updated} The summary of the fitting model used in this analysis for the \Mbc fit after updating the initial values to correspond for the correction in \Mbc distributions of background, as discussed in \Cref{sec:continuum_spectrum_validation}.
    The paramaters are initialised at the values that are listed, corresponding to the ones determined in the primary fitting steps, explained in \Cref{sec:crystal_ball_prefit,sec:chebyshev_prefit,sec:argus_prefit}, with \Mbc replaced by a `corrected'-\Mbc value.
    The values that are bolded in the table are not estimated from the final \Mbc fit, but are kept at their initialised values.
    On the other hand, all non-bolded values are estimated from the final fitter.
    The uncertainties are those estimated using the \texttt{HESSE} method.
    }
    \resizebox{1\textwidth}{!}{
    {\def\arraystretch{1.5}\tabcolsep=5pt
        \begin{tabular}{|c|c|c|c|c|c|c|c|c|c|c|c|c|c|c|}

            \hline
            \multirow{2}{*}{\EB bin} & \multicolumn{5}{|c|}{Crystal Ball} & \multicolumn{6}{c|}{Chebyshev}    & \multicolumn{3}{c|}{Argus} \\
            \cline{2-15}
                                       & $\mathcal{N}_{\mathrm{CB}}$ &$\boldsymbol{\mu}$ & $\boldsymbol{\sigma}$ & $\boldsymbol{\alpha}$ & $\mathbf{n}$ & $\mathcal{N}_{\mathrm{cheb}}$  & $\mathbf{k_1}$ & $\mathbf{k_2}$ & $\mathbf{k_3}$ & $\mathbf{k_4}$ & $\mathbf{k_5}$ & $\mathcal{N}_{\mathrm{ARGUS}}$ & $c$ & $m_0$                 \\
            \hline
            1.4 -- 1.6               & $17294\pm131$ &\multirow{11}{*}{$\mathbf{5.279}$} & \multirow{11}{*}{$\mathbf{0.003}$} & \multirow{11}{*}{$\mathbf{1.573\pm0.035}$} & \multirow{11}{*}{$\mathbf{3.561\pm0.22}$} & $70507\pm 266$& $\mathbf{-0.150\pm0.007}$ & $\mathbf{-0.382\pm0.007}$ & $\mathbf{-0.272\pm0.006}$ & $\mathbf{-0.132\pm0.006}$  & $\mathbf{-0.003\pm0.006}$  & $76798\pm277$ &  $-26.35 \pm    0.81$ & \multirow{11}{*}{$5.2897$} \\
            \cline{1-2}\cline{7-14}
            1.6 -- 1.8               & $10218\pm101$&                                      &                                        &                                   &                                  & $33666\pm183$ & $\mathbf{-0.084 \pm 0.010}$ & $\mathbf{-0.411\pm0.010}$ & $\mathbf{-0.300 \pm  0.009}$ & $\mathbf{-0.140\pm0.009}$ & $\mathbf{-0.003\pm0.009}$ & $50658\pm225$ & $-21.08 \pm 0.99$ & \\               
            \cline{1-2}\cline{7-14}
            1.8 -- 2.0               & $5947\pm77$&                                      &                                        &                                   &                                  &$16192\pm127$ & \multirow{9}{*}{$\mathbf{0.030\pm0.011}$} & \multirow{9}{*}{$\mathbf{-0.438\pm0.011}$} & \multirow{9}{*}{$\mathbf{-0.377\pm 0.009}$} & \multirow{9}{*}{$\mathbf{-0.175\pm0.010}$} & \multirow{9}{*}{$\mathbf{-0.007 \pm 0.010}$} & $31228\pm176$ & \multirow{9}{*}{$-18.64 \pm 0.93$} & \\                         
            \cline{1-2}\cline{7-7}\cline{13-13}
            2.0 -- 2.1               & $1938\pm44$&                                      &                                        &                                   &                                  & $4279\pm65$ &                 &                  &                   &                  &                    & $9983\pm100$ &                   & \\                                                                  
            \cline{1-2}\cline{7-7}\cline{13-13}
            2.1 -- 2.2               & $1246\pm35$&                                      &                                        &                                   &                                  & $2589\pm51$&                 &                  &                   &                  &                    & $6951\pm83$&                   & \\   
            \cline{1-2}\cline{7-7}\cline{13-13}
            2.2 -- 2.3               & $909\pm30$&                                      &                                        &                                   &                                  & $1581\pm40$&                &                  &                   &                  &                    &  $4470\pm67$&                  & \\   
            \cline{1-2}\cline{7-7}\cline{13-13}
            2.3 -- 2.4               & $985\pm31$&                                      &                                        &                                   &                                  & $1197\pm35$&                &                  &                   &                  &                    & $2663\pm52$&                   & \\   
            \cline{1-2}\cline{7-7}\cline{13-13}
            2.4 -- 2.5               & $1213\pm35$&                                     &                                        &                                   &                                  &$779\pm28$  &                &                  &                   &                  &                    & $1482\pm39$ &                  & \\   
            \cline{1-2}\cline{7-7}\cline{13-13}
            2.5 -- 2.6               & $626\pm25$&                                     &                                        &                                   &                                  &$310\pm18$ &                 &                  &                   &                  &                    &  $662\pm26$&                  & \\   
            \cline{1-2}\cline{7-7}\cline{13-13}
            2.6 -- 2.7               & $62\pm8$&                                     &                                        &                                   &                                  & $52\pm7$ &                &                  &                   &                  &                    & $211\pm15$ &                   & \\   
            \cline{1-2}\cline{7-7}\cline{13-13}
            2.7 -- 5.0               & $1\pm1$&                                     &                                        &                                   &                                  & $6\pm2$&                &                  &                   &                  &                    & $73\pm9$&                  & \\   
            \hline
        \end{tabular}
    }
}

\end{table}

This fitter is applied on the continuum-enhanced samples of Belle~II data used in this analysis,
and the  fitted \Mbc distributions are shown in \Cref{tab:fitting_init_params_updated}.
The figures also contain the number of good tag-\B mesons that are estimated by the fitter.
It can be seen that no statistically significant peak in \Mbc is extracted.
Therefore it can be concluded that the validation of the \Mbc fitter is successful: enough flexibility is seen
to describe the slight difference in \Mbc shapes between simulation and real data, without introducing a bias towards the extracted number of good tag-\B mesons.


\begin{figure}[htbp!]
    \centering
    \subcaptionbox{\label{fig:mbc_qqbar_ehnhanced_data_1p4}}{
        \includegraphics[width=0.3\textwidth]{figures/data_validation/qqbar_ehnaced_fits/full_MbcFit_1p4to1p6ppdf.pdf}
    }
    \subcaptionbox{\label{fig:mbc_qqbar_ehnhanced_data_1p6}}{
        \includegraphics[width=0.3\textwidth]{figures/data_validation/qqbar_ehnaced_fits/full_MbcFit_1p6to1p8ppdf.pdf}
    }
    \subcaptionbox{\label{fig:mbc_qqbar_ehnhanced_data_1p8}}{
        \includegraphics[width=0.3\textwidth]{figures/data_validation/qqbar_ehnaced_fits/full_MbcFit_1p8to2p0ppdf.pdf}
    }
    \subcaptionbox{\label{fig:mbc_qqbar_ehnhanced_data_2p0}}{
        \includegraphics[width=0.3\textwidth]{figures/data_validation/qqbar_ehnaced_fits/full_MbcFit_2p0to2p1ppdf.pdf}
    }
    \subcaptionbox{\label{fig:mbc_qqbar_ehnhanced_data_2p1}}{
        \includegraphics[width=0.3\textwidth]{figures/data_validation/qqbar_ehnaced_fits/full_MbcFit_2p1to2p2ppdf.pdf}
    }
    \subcaptionbox{\label{fig:mbc_qqbar_ehnhanced_data_2p2}}{
        \includegraphics[width=0.3\textwidth]{figures/data_validation/qqbar_ehnaced_fits/full_MbcFit_2p2to2p3ppdf.pdf}
    }
    \subcaptionbox{\label{fig:mbc_qqbar_ehnhanced_data_2p3}}{
        \includegraphics[width=0.3\textwidth]{figures/data_validation/qqbar_ehnaced_fits/full_MbcFit_2p3to2p4ppdf.pdf}
    }
    \subcaptionbox{\label{fig:mbc_qqbar_ehnhanced_data_2p4}}{
        \includegraphics[width=0.3\textwidth]{figures/data_validation/qqbar_ehnaced_fits/full_MbcFit_2p4to2p5ppdf.pdf}
    }
    \subcaptionbox{\label{fig:mbc_qqbar_ehnhanced_data_2p5}}{
        \includegraphics[width=0.3\textwidth]{figures/data_validation/qqbar_ehnaced_fits/full_MbcFit_2p5to2p6ppdf.pdf}
    }
    \subcaptionbox{\label{fig:mbc_qqbar_ehnhanced_data_2p6}}{
        \includegraphics[width=0.3\textwidth]{figures/data_validation/qqbar_ehnaced_fits/full_MbcFit_2p6to2p7ppdf.pdf}
    }
    \subcaptionbox{\label{fig:mbc_qqbar_ehnhanced_data_2p7}}{
        \includegraphics[width=0.3\textwidth]{figures/data_validation/qqbar_ehnaced_fits/full_MbcFit_2p7to5p0ppdf.pdf}
    }
    \caption{\label{fig:mbc_qqbar_ehnhanced_fits}The fits on Belle II dataset corresponding to 189~\invfb with selection that enhances \epem\ra\qqbar events, as discussed in \Cref{sec:continuum_spectrum_validation}.
    The fitting model from \Cref{tab:fitting_init_params_updated} is used, which is defined on the `corrected'-\Mbc to account for variations $\sqrt{s}$ in Belle II data.
    Good description of the \Mbc distributions can be seen throughout the \EB bins.
    As the continuum component in each \EB bin is enhanced, no good tag-\B mesons are expected and indeed - no statistically significant peak is extracted from the results.
    }
\end{figure}

\subsection{Validation on \texorpdfstring{\BB}{BB}-background enhanced sample}\label{sec:bb_background_validation}

In the last Section, the validation on a sample with \epem\ra\qqbar events was performed.
Another important validation given the background subtraction step (\Cref{sec:background_subtraction}),
is the \BB background decription in simulated data.

Firstly, a \BB-background enhanced sample is prepared.
This is done in almost the same manner as \Cref{sec:continuum_mbc_validation}, except with the continuum suppression requirement unchanged from the optimal selection.
In this case, only the \piVeto and \etaVeto selections are inverted.
The selections are chosen as $\piVeto>0.6$ and $\etaVeto>0.6$, as these selection ensure that the signal-to-background ratio is less than 0.1\%.
As the selections are inverted, so must the corrections that account for them be modified (\Cref{tab:correction_table}).
In particular, the corresponding corrections for the \piVeto and \etaVeto are transformed as follows:
\begin{equation}\label{eq:correction_transform}
    \mathrm{Corr}_{>0.x} = \frac{1}{\mathrm{Corr_{<0.x}}}; \quad \sigma(\mathrm{Corr}_{>0.x}) =  \frac{1}{\mathrm{Corr_{<0.x}}^2} \times \sigma(\mathrm{Corr}_{<0.x}).
\end{equation}

The \EB distribution of the \BB-enhanced sample is shown in \Cref{fig:bbbar_enhanced_eb}.
Overall, the distributions show excellent agreement between data and simulation.
The previously seen discrepancy in normalisation (see \Cref{fig:qqbar_enhanced_eb_validation}) is no longer apparent.
This can be interpreted based on the fact that here \epem\ra\qqbar events are strongly suppressed, namely 
by the requirements of \texttt{BDT~output}.
\begin{figure}[htbp!]
    \centering
    \subcaptionbox{\label{fig:Bplus_bbbar_enhanced_eb}}{
        \includegraphics[width=0.3\textwidth]{figures/data_validation/Bplus_bbbar_enhanced_eb.pdf}
    }
    \subcaptionbox{\label{fig:Bzero_bbbar_enhanced_eb}}{
        \includegraphics[width=0.3\textwidth]{figures/data_validation/Bzero_bbbar_enhanced_eb.pdf}
    }
    \subcaptionbox{\label{fig:Bboth_bbbar_enhanced_eb}}{
        \includegraphics[width=0.3\textwidth]{figures/data_validation/Bboth_bbbar_enhanced_eb.pdf}
    }
    \caption{\label{fig:bbbar_enhanced_eb} The \EB distribution of \BB-background enhanced samples (see \Cref{sec:bb_background_validation}).
    Compared to \Cref{fig:bboth_offresonance_eb}, it is clear that the \BB background drops off faster with increasing \EB than \epem\ra\qqbar.
    Overall, the data-simulation agreement is excellent and this is attributed to the fact that continuum events, which were accredited to causing a discrepancy in \Cref{sec:continuum_spectrum_validation}, are highly suppressed in the \BB-background enhanced sample.
    }
\end{figure}

Following the observations in \Cref{sec:continuum_mbc_validation}, similar modifications to \Mbc are necessary here, too.
The corrected-\Mbc distribution is shown in \Cref{fig:bbbar_enhanced_mbccorrected}.
The results are shown for the combined \feiBp and \feiBz sample only, although the individual \FEI modes also show similar results.
In particular, one can see that the agreement between data and simulation is closer than what was observed in \Cref{sec:continuum_mbc_validation}.
This goes in line with the previous statements that \epem\ra\qqbar modelling was related to the discrepancies observed so far, which are not strongly pronounced here due to their suppression.
Observing the pull distribution in \Cref{fig:bbbar_enhanced_mbccorrected} it is clear that, generally, fewer events seem to be present in the peak region and more in the tail region.
Again, this is in line with the expectation: the variations of $\sqrt{s}$ in real data modify the \BB-production and \epem\ra\qqbar cross-sections.
\begin{figure}[htbp!]
    \centering
    \includegraphics[width=0.45\textwidth]{figures/data_validation/Bboth_bbbar_enhanced_mbccorrected.pdf}
    \caption{\label{fig:bbbar_enhanced_mbccorrected} The \Mbc distribution of \BB-background enhanced samples (see \Cref{sec:bb_background_validation}),
    where an \Mbc correction has been applied to the simulated distribution.
    Only the \feiBp and \feiBz combined sample is shown, but the individual ones show similar results.
    Due to a lower number of \epem\ra\qqbar events, the low-\Mbc region disagreement is less pronounced than in \Cref{fig:qqbar_enhanced_mbccorrected}.
    }
\end{figure}

Finally, a fit of the \BB-enhanced performed of the \Mbc distributions, according to the modified \Mbc fitter model in \Cref{tab:fitting_init_params_updated}.
Unlike in \Cref{fig:mbc_qqbar_ehnhanced_fits}, the number of expected-\BB events is no longer expected to be negligible.
Therefore, this validation serves as a test of fit-bias and background-subtraction procedure.
The fits on Belle II simulation are shown in \Cref{fig:mbc_bbar_ehnhanced_fits_mc}, whereas the fits on Belle II data \Cref{fig:mbc_bbar_ehnhanced_fits_data}.
The figures also shown the extracted good tag-\B meson counts as normalisations of the Crystall Ball \PDF, $\mathcal{N}_{\mathrm{CB}}$.
\begin{figure}[htbp!]
    \centering
    \subcaptionbox{\label{fig:mbc_bbar_ehnhanced_mc_1p4}}{
        \includegraphics[width=0.3\textwidth]{figures/data_validation/bbbar_enhanced_fits/MC_MbcFit_1p4to1p6ppdf.pdf}
    }
    \subcaptionbox{\label{fig:mbc_bbar_ehnhanced_mc_1p6}}{
        \includegraphics[width=0.3\textwidth]{figures/data_validation/bbbar_enhanced_fits/MC_MbcFit_1p6to1p8ppdf.pdf}
    }
    \subcaptionbox{\label{fig:mbc_bbar_ehnhanced_mc_1p8}}{
        \includegraphics[width=0.3\textwidth]{figures/data_validation/bbbar_enhanced_fits/MC_MbcFit_1p8to2p0ppdf.pdf}
    }
    \subcaptionbox{\label{fig:mbc_bbar_ehnhanced_mc_2p0}}{
        \includegraphics[width=0.3\textwidth]{figures/data_validation/bbbar_enhanced_fits/MC_MbcFit_2p0to2p1ppdf.pdf}
    }
    \subcaptionbox{\label{fig:mbc_bbar_ehnhanced_mc_2p1}}{
        \includegraphics[width=0.3\textwidth]{figures/data_validation/bbbar_enhanced_fits/MC_MbcFit_2p1to2p2ppdf.pdf}
    }
    \subcaptionbox{\label{fig:mbc_bbar_ehnhanced_mc_2p2}}{
        \includegraphics[width=0.3\textwidth]{figures/data_validation/bbbar_enhanced_fits/MC_MbcFit_2p2to2p3ppdf.pdf}
    }
    \subcaptionbox{\label{fig:mbc_bbar_ehnhanced_mc_2p3}}{
        \includegraphics[width=0.3\textwidth]{figures/data_validation/bbbar_enhanced_fits/MC_MbcFit_2p3to2p4ppdf.pdf}
    }
    \subcaptionbox{\label{fig:mbc_bbar_ehnhanced_mc_2p4}}{
        \includegraphics[width=0.3\textwidth]{figures/data_validation/bbbar_enhanced_fits/MC_MbcFit_2p4to2p5ppdf.pdf}
    }
    \subcaptionbox{\label{fig:mbc_bbar_ehnhanced_mc_2p5}}{
        \includegraphics[width=0.3\textwidth]{figures/data_validation/bbbar_enhanced_fits/MC_MbcFit_2p5to2p6ppdf.pdf}
    }
    \subcaptionbox{\label{fig:mbc_bbar_ehnhanced_mc_2p6}}{
        \includegraphics[width=0.3\textwidth]{figures/data_validation/bbbar_enhanced_fits/MC_MbcFit_2p6to2p7ppdf.pdf}
    }
    \subcaptionbox{\label{fig:mbc_bbar_ehnhanced_mc_2p7}}{
        \includegraphics[width=0.3\textwidth]{figures/data_validation/bbbar_enhanced_fits/MC_MbcFit_2p7to5p0ppdf.pdf}
    }
    \caption{\label{fig:mbc_bbar_ehnhanced_fits_mc}
    The fits on Belle II simulation corresponding to 1.6~\invfb with selection 
    that enhances non-\BtoXsgamma events, as discussed in \Cref{sec:continuum_spectrum_validation}.
    The fitting model from \Cref{tab:fitting_init_params_updated} is used,
    which is defined on the `corrected'-\Mbc to account for variations $\sqrt{s}$ in Belle II data.
    Good description of the \Mbc distributions can be seen throughout the \EB bins.
    }
\end{figure}
\begin{figure}[htbp!]
    \centering
    \subcaptionbox{\label{fig:mbc_bbar_ehnhanced_data_1p4}}{
        \includegraphics[width=0.3\textwidth]{figures/data_validation/bbbar_enhanced_fits/DATA_MbcFit_1p4to1p6ppdf.pdf}
    }
    \subcaptionbox{\label{fig:mbc_bbar_ehnhanced_data_1p6}}{
        \includegraphics[width=0.3\textwidth]{figures/data_validation/bbbar_enhanced_fits/DATA_MbcFit_1p6to1p8ppdf.pdf}
    }
    \subcaptionbox{\label{fig:mbc_bbar_ehnhanced_data_1p8}}{
        \includegraphics[width=0.3\textwidth]{figures/data_validation/bbbar_enhanced_fits/DATA_MbcFit_1p8to2p0ppdf.pdf}
    }
    \subcaptionbox{\label{fig:mbc_bbar_ehnhanced_data_2p0}}{
        \includegraphics[width=0.3\textwidth]{figures/data_validation/bbbar_enhanced_fits/DATA_MbcFit_2p0to2p1ppdf.pdf}
    }
    \subcaptionbox{\label{fig:mbc_bbar_ehnhanced_data_2p1}}{
        \includegraphics[width=0.3\textwidth]{figures/data_validation/bbbar_enhanced_fits/DATA_MbcFit_2p1to2p2ppdf.pdf}
    }
    \subcaptionbox{\label{fig:mbc_bbar_ehnhanced_data_2p2}}{
        \includegraphics[width=0.3\textwidth]{figures/data_validation/bbbar_enhanced_fits/DATA_MbcFit_2p2to2p3ppdf.pdf}
    }
    \subcaptionbox{\label{fig:mbc_bbar_ehnhanced_data_2p3}}{
        \includegraphics[width=0.3\textwidth]{figures/data_validation/bbbar_enhanced_fits/DATA_MbcFit_2p3to2p4ppdf.pdf}
    }
    \subcaptionbox{\label{fig:mbc_bbar_ehnhanced_data_2p4}}{
        \includegraphics[width=0.3\textwidth]{figures/data_validation/bbbar_enhanced_fits/DATA_MbcFit_2p4to2p5ppdf.pdf}
    }
    \subcaptionbox{\label{fig:mbc_bbar_ehnhanced_data_2p5}}{
        \includegraphics[width=0.3\textwidth]{figures/data_validation/bbbar_enhanced_fits/DATA_MbcFit_2p5to2p6ppdf.pdf}
    }
    \subcaptionbox{\label{fig:mbc_bbar_ehnhanced_data_2p6}}{
        \includegraphics[width=0.3\textwidth]{figures/data_validation/bbbar_enhanced_fits/DATA_MbcFit_2p6to2p7ppdf.pdf}
    }
    \subcaptionbox{\label{fig:mbc_bbar_ehnhanced_data_2p7}}{
        \includegraphics[width=0.3\textwidth]{figures/data_validation/bbbar_enhanced_fits/DATA_MbcFit_2p7to5p0ppdf.pdf}
    }
    \caption{\label{fig:mbc_bbar_ehnhanced_fits_data}
    The fits on Belle II data corresponding to 189~\invfb with selection 
    that enhances non-\BtoXsgamma events, as discussed in \Cref{sec:continuum_spectrum_validation}.
    The fitting model from \Cref{tab:fitting_init_params_updated} is used,
    which is defined on the `corrected'-\Mbc to account for variations $\sqrt{s}$ in Belle II data.
    Good description of the \Mbc distributions can be seen throughout the \EB bins.
    }
\end{figure}

Despite a vastly varying number of events and shapes of the total distribution throughout different \EB intervals,
the fitter performs well.
The extracted $\mathcal{N}_{\mathrm{CB}}^{\mathrm{DATA}}$ $\mathcal{N}_{\mathrm{CB}}^{\mathrm{MC}}$ (corresponding to good tag-\B meson yields in data and simulation, respectively)
are directly compared.
Correcting the simulation based on \Cref{tab:correction_table} (with inverted \piVeto and \etaVeto correction as shown in \Cref{eq:correction_transform}) and calculating the difference between the two is expected to yield a value consistent with zero.
The resulting difference, with appropriate statistical uncertainties and uncertainties from corrections applied are shown in \Cref{fig:bbar_enhanced_background_subtraction}.
The hypothesised result is observed, confirming the adequacy of the fitter, the validity of background-subtraction procedure and the corrections, that were applied.

\begin{figure}[htbp!]
    \centering
    \includegraphics[width=0.45\textwidth]{figures/data_validation/bbar_enhanced_event_counts.pdf}
    \caption{\label{fig:bbar_enhanced_background_subtraction}
    The number of events after subtracting good tag-\B meson counts extracted from fits in Belle~II simulation (\Cref{fig:mbc_bbar_ehnhanced_fits_mc}),
    from that in Belle~II data (\Cref{fig:mbc_bbar_ehnhanced_fits_data}).
    The simulated values are corrected for luminosity and to better represent data based on studies in \Cref{sec:corrections}.
    The background subtraction procedure is further detailed in \Cref{sec:background_subtraction,sec:background_subtraction_validation_mc}.
    Here, an agreement with 0 is expected in all \EB bins (no \BtoXsgamma events are present).
    }
\end{figure}

\subsection{Validation outside of the \texorpdfstring{\EB}{EB} signal-region}\label{sec:sidebands_validation}

In \Cref{sec:continuum_spectrum_validation,sec:continuum_mbc_validation,sec:bb_background_validation} it was seen that the background simulation of \qqbar and \BB events,
althought not perfect, is described adequatly by the \Mbc fitter yielding correct and valid estimation of good tag-\B mesons in data and simulation.
The last validation performed for background simulation is done outside of \EB signal region.
As discussed in \Cref{sec:binning}, the $\EB\in(1.4-1.8)~\gev$ and $2.7<\EB~\gev$ intervals were selected as sideband regions, 
due to a small number of \BtoXsgamma events and a low signal-to-background ratio expected there.
The same argumentation make the regions excellent for background validation.

The \EB distribution, for the three \EB sideband intervals are shown in \Cref{fig:sidebands_eb}.
A striking, nearly 20\%, difference in normalisation is observed, 
which seems similar to that observed in \Cref{fig:qqbar_enhanced_eb_validation}.
Interestingly, here, the \epem\ra\qqbar component is thought to be strongly suppressed by \texttt{BDT~output}.
To better understand this discrepancy, the corrected-\Mbc distributions in each \EB sideband bin are inspected.
This is shown in \Cref{fig:sidebands_mbc}.
\begin{figure}[htbp!]
    \centering
    \subcaptionbox{\label{fig:Bplus_sidebands_eb}}{
        \includegraphics[width=0.31\textwidth]{figures/data_validation/Bplus_sidebands_eb.pdf}
    }
    \subcaptionbox{\label{fig:Bzero_sidebands_eb}}{
        \includegraphics[width=0.31\textwidth]{figures/data_validation/Bzero_sidebands_eb.pdf}
    }
    \subcaptionbox{\label{fig:Bboth_sidebands_eb}}{
        \includegraphics[width=0.31\textwidth]{figures/data_validation/Bboth_sidebands_eb.pdf}
    }
    \caption{\label{fig:sidebands_eb} 
    The \EB distribution of the \EB sideband regions (see \Cref{sec:sidebands_validation}).
    The low-\EB region side sees a roughly 20\% discrepancy.
    The shaded area represents the signal region which is blinded: during the validation step it was not observed.
    }
\end{figure}
\begin{figure}[htbp!]
    \centering
    \subcaptionbox{\label{fig:sidebands_mbc1}}{
        \includegraphics[width=0.31\textwidth]{figures/data_validation/sidebands_mbc_1.pdf}
    }
    \subcaptionbox{\label{fig:sidebands_mbc2}}{
        \includegraphics[width=0.31\textwidth]{figures/data_validation/sidebands_mbc_2.pdf}
    }
    \subcaptionbox{\label{fig:sidebands_mbc3}}{
        \includegraphics[width=0.3\textwidth]{figures/data_validation/sidebands_mbc_3.pdf}
    }1
    \caption{\label{fig:sidebands_mbc} 
    The \Mbc distribution of the \EB sideband regions (see \Cref{sec:sidebands_validation}).
    The figures showcase different \EB ranges, as indicated on the right corner of each Figure.
    Interestingly, a similar low-\Mbc discrepancy is observer as that with the enhanced-continuum sample, shown in \Cref{fig:qqbar_enhanced_mbc}.
    }
\end{figure}

The results of \Cref{fig:sidebands_eb,fig:sidebands_mbc} indicate that Belle~II data, especially the low-\Mbc region, has a clear excess compared to simulation.
While some discrepancy is expected considering the results of \Cref{fig:qqbar_enhanced_mbccorrected},
the larger scale of the discrepancy is confusing -- given the fact that this was not observed in \Cref{fig:bbbar_enhanced_mbccorrected}.
Although more studies on this subject are necessary to fully understand the discrepancy, 
it is attributed to a \BB-background component which is not well modelled in Belle~II simulation, the potential origin of which is shortly discussed here.

In particular, consider the removal of \texttt{zernikeMVA} selection.
The resulting \EB sideband distribution and the \Mbc distribution are shown in \Cref{fig:nozmva_test}.
In this case, the agreement between data and simulation in the \EB-sideband spectrum appears to be near-perfect, as seen in \Cref{fig:sideband_eb_nozmva}.
Indeed, even considering the \Mbc distribution for $\EB\in(1.4,1.6)~\gev$ in \Cref{fig:sideband_mbc_nozmva}, one clearly observes a better overall-agreement.
These observations strongly suppor the previous hypothesis that a hadronic cluster is not well-modelled in simulation.
The component cannot be common to \epem\ra\qqbar events, because the effect was not seen in off-resonance data in \Cref{sec:continuum_spectrum_validation}.
As a result, this component is suppressed in Belle~II simulation by a selection on the \texttt{zernikeMVA} observable, but this does happen in Belle~II data.
\begin{figure}[htbp!]
    \subcaptionbox{\label{fig:sideband_eb_nozmva}}{
        \includegraphics[width=0.31\textwidth]{figures/data_validation/Bboth_sidebands_eb_nozmva.pdf}
    }
    \subcaptionbox{\label{fig:sideband_mbc_nozmva}}{
        \includegraphics[width=0.31\textwidth]{figures/data_validation/sidebands_mbc_nozmva_1.pdf}
    }
    \subcaptionbox{\label{fig:zmva_discrepancy}}{
        \includegraphics[width=0.31\textwidth]{figures/data_validation/Bboth_sidebands_zmva.pdf}
    }
    \caption{\label{fig:nozmva_test}   The \EB distributions in the \EB sideband region (\Cref{fig:sideband_eb_nozmva})
    and the \Mbc distribution in $1.4<\EB<1.6~\gev$ regions in both cases the optimised requirement on $\mathtt{zernikeMVA}$.
    \Cref{fig:zmva_discrepancy} is the full-range of the \ZMVA distribution.
    The \Cref{fig:sideband_eb_nozmva,fig:sideband_mbc_nozmva} can be directly compared with \Cref{fig:Bboth_sidebands_eb,fig:sidebands_mbc3}, respectively.
    Without the $\mathtt{zernikeMVA}$ the agreement between data and simulation appears to be improved indicating that a component in data is not simulated adequatly.
    The presence of a mismodelled component in Belle~II simulation is apparent when inspecting the \ZMVA distribution.
    }
\end{figure}

These considerations are supported by inspecting the \ZMVA distribution in the $\mbox{\EB\in(1.4,1.8)~\gev}$ region, as seen in \Cref{fig:zmva_discrepancy}.
While simulation contains a sharp peak near 0, this is not evident in Belle~II data, which on the other hand has an excess at high-\ZMVA.
Hence, the background-suppression efficiency is not well-represented in simulation.
These results are also conflated with the differences in \Mbc endpoint, making an exact evaluation of the effect difficult at this stage.
Independent studies of the \ZMVA distributions, performed similarly to the photon detection efficiency study described in \Cref{sec:photon_efficiency}, did not observe the presence of such peak.
This may imply that this type of selection is particular for photon candidates from \BB events misidentified as high-energy photons.
Therefore, it was concluded that additional studies of \ZMVA in the context of radiative and inclusive analyses will be necessary for future iterations of this analysis.

While these observations are alarming, so far all the results have shown that the \Mbc fitter and background-subtraction procedure were robust against the
continuum \Mbc distribution shape differences, as seen in \Cref{sec:continuum_mbc_validation,sec:bb_background_validation}.
Therefore, the further analysis does not replace or remove the \ZMVA requirement.

The fit is performed on the full sample of data, however, the results from the signal region, $\EB\in(1.8,2.7)$ are hidden and not investigated at this stage.
The individual \Mbc fits on the \EB sideband regions are shown for data in \Cref{fig:sideband_data_fit}, and for simulation in \Cref{fig:sideband_mc_fit}.
\begin{figure}[htbp!]
    \centering
    \subcaptionbox{\label{fig:sideband_data_fit_1}}{
        \includegraphics[width=0.31\textwidth]{figures/data_validation/sidebands_fit/DATA_MbcFit_1p4to1p6ppdf.pdf}
    }
    \subcaptionbox{\label{fig:sideband_data_fit_2}}{
        \includegraphics[width=0.31\textwidth]{figures/data_validation/sidebands_fit/DATA_MbcFit_1p6to1p8ppdf.pdf}
            }
    \subcaptionbox{\label{fig:sideband_data_fit_11}}{
        \includegraphics[width=0.31\textwidth]{figures/data_validation/sidebands_fit/DATA_MbcFit_2p7to5p0ppdf.pdf}
    }
    \caption{\label{fig:sideband_data_fit}    
    The \Mbc fits on Belle II data corresponding to 189~\invfb of data in \EB sideband regions,
    as discussed in \Cref{sec:sidebands_validation}.
    The fitting model from \Cref{tab:fitting_init_params_updated} is used,
    which is defined on the `corrected'-\Mbc to account for variations $\sqrt{s}$ in Belle II data.
    Good description of the \Mbc distributions can be seen throughout the \EB bins.
    }
\end{figure}
\begin{figure}[htbp!]
    \centering
    \subcaptionbox{\label{fig:sideband_mc_fit_1}}{
        \includegraphics[width=0.31\textwidth]{figures/data_validation/sidebands_fit/NOSIGNALMC_MbcFit_1p4to1p6ppdf.pdf}
    }
    \subcaptionbox{\label{fig:sideband_mc_fit_2}}{
        \includegraphics[width=0.31\textwidth]{figures/data_validation/sidebands_fit/NOSIGNALMC_MbcFit_1p6to1p8ppdf.pdf}
            }
    \subcaptionbox{\label{fig:sideband_mc_fit_3}}{
        \includegraphics[width=0.31\textwidth]{figures/data_validation/sidebands_fit/NOSIGNALMC_MbcFit_2p7to5p0ppdf.pdf}
    }
    \caption{\label{fig:sideband_mc_fit}    
    The \Mbc fits on Belle II simulation corresponding to 189~\invfb of data in \EB sideband regions,
    as discussed in \Cref{sec:sidebands_validation}.
    These distributions contain no \BtoXsgamma events, although even without the requirement the number of radiative transitions in these energy ranges are negligible.
    The fitting model from \Cref{tab:fitting_init_params_updated} is used,
    which is defined on the `corrected'-\Mbc to account for variations $\sqrt{s}$ in Belle II data.
    Good description of the \Mbc distributions can be seen throughout the \EB bins.
    }
\end{figure}

The summarised results of the good tag-\B meson yields estimated in the \Mbc fits are shown in \Cref{fig:sidebands_background_versus_data}.
In the high-\EB sideband no peaking tag-\B mesons are observed in data or simulation, which is exactly consistent with the naive expectations.
In the low-\EB region a large number of events is observed.
Clearly, the data points are compatible with the background expectation, although the estimates in both intervals are higher than the expected background.
Subtracting the background expectation from the good tag-\B meson yield in data results in \Cref{fig:sidebands_subtracted}.
A similar observation follows: although both values are (nearly) compatible with zero,
the central values are slightly positive.

\begin{figure}[htbp!]
    \centering
    \includegraphics[width=0.45\textwidth]{figures/data_validation/sidebands_background_vs_data.pdf}
    \caption{\label{fig:sidebands_background_versus_data} The results of fitting the \Mbc on the sideband region in data (see \Cref{sec:sidebands_validation}).
    The values corresponding to data fits are estimated through an \Mbc fit shown in \Cref{fig:sideband_data_fit}.
    The remaining-\BB background expectations are estimated through \Mbc fits in \Cref{fig:sideband_mc_fit} and \Cref{fig:nosignal_fits_signal}.
    The signal region in this figure is blinded, therefore only simulation results are shown.
    The extracted results from the data fit and simulated background expectations 
    are compatible within their full uncertainty, but both points are higher than the background estimation.
    }
\end{figure}

\begin{figure}[htbp!]
    \centering
    \subcaptionbox{\label{fig:sidebands_subtracted}}{
        \includegraphics[width=0.45\textwidth]{figures/data_validation/sideband_event_counts.pdf}
    }
    \subcaptionbox{\label{fig:sidebands_subtracted_corrected}}{
        \includegraphics[width=0.45\textwidth]{figures/data_validation/sideband_event_counts_corrected.pdf}
    }
    \caption{\label{fig:sidebands_subtracted_figures} The results of subtraction of the remaining-\BB background after the fit on Belle~II data.
    \Cref{fig:sidebands_subtracted} show the results with no correction factor applied.
    \Cref{fig:sidebands_subtracted_corrected} includes an 8.7\% scaling factor for the simulated background values.
    The signal region, denoted by the shaded area, is blinded at this stage.
    In this analysis it is chosen to scale the background (i.e. scenario shown in \Cref{fig:sidebands_subtracted_corrected}),
    as that region is expected to contain a number of \BtoXsgamma events consistent with 0.
    This figure only includes systematic uncertainties related to simulation corrections described in \Cref{sec:corrections}.
    }
\end{figure}

The total number of events in the low-\EB sideband are $2698\pm139$ (expected to be predominantly background, see \Cref{sec:binning}).
The background expectation from simulation in the same region is $2483\pm130$.
Taking the ratio of these values yields $1.087\pm0.080$.
As this value is not compatible with unity within $1\sigma$, a background scaling of 8.7\% is adopted.
100\% of the scaling will also be adopted as a systematic uncertainty later.
The scaled-background-subtracted data fit yields are shown in \Cref{fig:sidebands_subtracted_corrected}.
By construction, they are fully compatible with zero.

% \subsection{Summary of study of validation samples}\label{sec:summary_of_validation}

% As there was quite a few important points presented in \Cref{sec:validation}, 
% a quick summary is produced here to condense the most important findings.
% The main observations are as follows:
% \begin{itemize}
%     \item Good \EB spectrum description in \epem\ra\qqbar events, is observed in \Cref{sec:continuum_spectrum_validation,sec:continuum_mbc_validation}.
%     \item Problems with \Mbc distribution endpoint modelling, related to experimental condition variations in Belle~II data, which are not captured in run-period-independent simulation.
%     This is observed in \Cref{sec:continuum_mbc_validation}.
%     \item Related to the last point, differences with the shape of \epem\ra\qqbar events in simulation and data.
%     \item A new, corrected \Mbc variable introduced to partially compensate the \Mbc endpoint difference in data and simulation.
%     \item The fitter is able to accurately describe the number of good tag-\B meson events in data and simulation.
%     \item A scaling factor is introduced, which will also be adopted as a systematic uncertainty related to normalisation.
% \end{itemize}






\section{Signal modelling and efficiency studies}\label{sec:signal_modelling}
In \Cref{sec:validation} it was shown that the background distributions are adequatly represented in simulation,
which proves that the analysis setup on simulation is valid for data.
It was also seen that the \Mbc fitter is able to extract values consistent with zero, where no \BtoXsgamma signal was expected.
The analysis strategy in \Cref{sec:final_optimisation,sec:fitting_mbc,sec:background_subtraction} does not strongly depend on the signal model.
In fact, no strong assumptions are made about the signal shape at any point in the analysis so far.
Therefore, following the fitting and background subtraction procedures, the number of \BtoXsgamma events as a function of \EB in the analysed Belle~II data sample can be evaluated. 

In order to transform the measured numbers of \BtoXsgamma events to partial branching fractions of \BtoXsgamma decays, efficiency corrections and unfolding is necessary.
In this section, the expected signal efficiency and \EB resolution of \BtoXsgamma events will be investigated.
The hybrid-signal model will then be used to derive unfolding correction factors.

\subsection{Efficiency of \texorpdfstring{\BtoXsgamma}{B->Xs gamma} decays}\label{sec:signal_efficiency}

The \BtoXsgamma selection efficiency is evaluated using Belle~II simulation and corrected
based on the studies that have been discussed in \Cref{sec:corrections}.
The signal efficiency is firstly assumed to be factorisable:
\begin{equation}\label{eq:factorisable_signal_efficiency}
    \varepsilon_{\BtoXsgamma} = \varepsilon_{\mathrm{FEI}} \times \varepsilon_{\mathrm{selection}},
\end{equation}
where $\varepsilon_{\mathrm{FEI}}$ is the \FEI tagging efficiency and 
$\varepsilon_{\mathrm{selection}}$ is the selection efficiency related to requirements shown in \Cref{sec:selection_summary}.
The factorisation assumption is a valid one as $\varepsilon_{\mathrm{FEI}}$  is related to the reconstruction of the tag-\B meson,
whereas $\varepsilon_{\mathrm{selection}}$ is fully a signal-\B meson quantity.

The \FEI tagging efficiency is evaluated as:
\begin{equation}\label{eq:fei_tagging_efficiency}
    \varepsilon_{\mathrm{FEI}} = \frac{N(\BtoXsgamma)_{\mathrm{good~tags}}}{N(\BtoXsgamma)_{\mathrm{untagged}}}
\end{equation}
The numerator, $N(\BtoXsgamma)_{\mathrm{good~tags}}$, is equal to the number of \BtoXsgamma events associated with good tag-\B mesons after running \FEI. 
It is evaluated using the good-tag definition in \Cref{sec:good_tag_definition}.
The denominator, $N(\BtoXsgamma)_{\mathrm{untagged}}$, is equal to the number of \BtoXsgamma events on an equivalent sample, where \FEI is not run.
In both cases the hybrid signal-model is used.

The evaluated tagging efficiency is shown in \Cref{fig:epsilon_fei}.
The efficiency is evaluated as a function of $\tilde{\EB}$, which is adopted only here to denote that the \textit{true} photon energy is used, 
as opposed to the reconstructed value.
This is done, as the untagged inclusive sample cannot have a meaningful comparison in terms of reconstructed \EB.
It can be seen that the efficiency increases with $\tilde{\EB}$, but the overall increase is roughly 10\% throughout the considered range.
As direct connection between $\tilde{\EB}$ and $\EB$ is difficult to evaluate, the average efficiency value is chosen as the tagging efficiency:
\begin{equation}\label{eq:avg_efficiency_fei}
    \varepsilon_{\mathrm{FEI}} = 0.006659 \pm 0.000006,
\end{equation}
where the uncertainty is fully statistical.
The signal-modelling uncertainty is expected to be small, because any deviations would be suppressed in the rattio in \Cref{eq:fei_tagging_efficiency}.
\begin{figure}[htbp!]
    \centering
    \subcaptionbox{\label{fig:epsilon_fei}}{
        \includegraphics[width=0.41\textwidth]{figures/signal_validation/epsilon_fei.pdf}
    }
    \subcaptionbox{\label{fig:epsilon_selection}}{
        \includegraphics[width=0.41\textwidth]{figures/signal_validation/selection_efficiency.pdf}
    }
    \caption{\label{fig:epsilon} The efficiency evaluation of \BtoXsgamma events in simulated samples based on the two factorised components in
    \Cref{eq:factorisable_signal_efficiency}.
    $\epsilon_{\mathrm{FEI}}$, shown in \Cref{fig:epsilon_fei}, is seen to vary lightly, no more than 10\% accross the $\tilde{\EB}$ range.
    Note that $\tilde{\EB}$ is used, as opposed to \EB, to represenet `true' photon energy (generated for simulation), 
    as opposed to the reconstructed value.
    $\epsilon_{\mathrm{selection}}$, shown in \Cref{fig:epsilon_selection} for three different models,
    grows with \EB approximately linearly and starts to drop at $\EB\approx2.6~\gev$.
    The three models show consistent results, strengthening the argument of a signal-model independent analysis.
    }
\end{figure}

The \BtoXsgamma signal efficiency is evaluated using three different signal models as:
\begin{equation}\label{eq:signal_efficiency}
    \varepsilon_{\mathrm{selection}} = \frac{N(\BtoXsgamma)_{\mathrm{after~selection}}}{N(\BtoXsgamma)_{\mathrm{before~selection}}},
\end{equation}
here $N(\BtoXsgamma)_{\mathrm{after(before)~selection}}$ is the count of \BtoXsgamma events in the \FEI tagged sample with(without) 
the background suppression selections in this analysis, given in \Cref{tab:interative_optimisation}.
This is evaluated on three models: the Kagan-Neubert model, the Belle~II generic-\MC signal model, and the hybrid-signal model.
The results are shown in \Cref{fig:epsilon_selection}.
All three models show compatible results.
The $\varepsilon_{\mathrm{selection}}$ grows approximately linearly from 30\% at 1.4~\gev to 60\% at 2.6~\gev and then drops sharply.
In this analysis, the values from hybrid signal-model are used as central values of the efficiency.

These results, summarised in \Cref{fig:epsilon} allow to evaluate the simulated efficiency based on \Cref{eq:factorisable_signal_efficiency}.




