\chapter{Summary and outlook}\label{ch:summary}

The work presented in this thesis, radiative \BtoXsgamma decays have been studied using 189~\invfb of \epem collision data collected at the \FourS resonance energy by the Belle~II experiment.
These decays are important probes of the Standard Model and beyond, providing relevant information to deepen the understanding of nature.
Such information can be obtained through the measurements of \BtoXsgamma photon energy spectrum shape, its moments and the total branching fraction.

\Crefrange{ch:theory}{ch:analysis_techniques} provided an overview of the theoretical foundation and experimental status of \mbox{\BtoXsgamma} decays, 
introduced the state-of-the-art \B-factory experiment Belle~II and summarised the most relevant analysis techniques used in high energy physics.
\Cref{ch:analysis} described the analysis
of the \BtoXsgamma decays using a hadronic-tagging approach, where the partnering \B meson from the \FourS decay is fully reconstructed.
In an inclusive treatment, only the high energy photon is reconstructed, ensuring that all $X_s$ states are selected.
Hadronic-tagged \BtoXsgamma decays were measured for the first time with the Belle~II experiment, and for the second time, in general.
It is an important measurement technique which enables direct access to observables in the decaying \B meson rest frame.
\Cref{ch:overview} discussed the analysis results in the context of past measurements, future outlook and current impact.

To achieve the best result, this \BtoXsgamma analysis used the Full Event Interpretation algorithm, which is a series of boosted decision trees, to reconstruct the partner \B meson.
As a result, this analysis described the steps to suppress two types of background: those related to the signal-side and those related to the tag-side.
Signal-side backgrounds have been suppressed by employing multivariate algorithms and selections.
The tag-side backgrounds were suppressed by extracting the counts of correctly-reconstructed tag-\B mesons in different photon energy bins.
The remaining photon backgrounds were then removed by relying on simulation.
The analysis selections and the overall procedure were tested and validated with numerous independent processes, including other \B decays, \mbox{\epem\ra\qqbar}, and \mbox{\epem\ra\mumu}.
The results obtained in this thesis, also available as preprint in Ref.~\cite{Belle-II:2022hys}, show excellent agreement with the world averages and the Standard Model predictions.
They are compiled in the Table below.
\begin{table}[hbtp!]
    \resizebox{1\textwidth}{!}{
    \begin{tabular}{|l|ccc|}
        \hline
        Observable [$\EB>1.8~\gev$] & Results of the analysis & World average value & Standard Model prediction \\
        \hline
        $\mathcal{B}(\BtoXsgamma)$ [$10^{-4}$]& 
        $3.54 \pm 0.78$ (stat.) $\pm~0.83$ (syst.)  &
        $3.49\pm0.19$ &
        $3.40\pm0.17$ 
        \\

        $\expval**{\EB}$ [\gev] & 
        $2.284 \pm 0.065$ (stat.) $\pm 0.071$ (syst.) &
        $2.314 \pm 0.011$ & 
        model dependent
        \\

        $\expval**{\EB^2}-\expval**{\EB}^2$ [$\gev^2$] & 
        $0.0502 \pm 0.0157$ (stat.) $\pm 0.0176$ (syst.) &
        $0.0303 \pm 0.0025$ &
        model dependent        
        \\
        \hline
    \end{tabular}
    }
\end{table}

The current version of the analysis contains a large statistical uncertainty component. 
This will be reduced in the future, as more and more data are collected by the Belle~II experiment.
The systematic uncertainty strongly depends on the lower photon energy threshold which is employed to suppress background.
Indeed, with a threshold of 1.8~\gev that was used in this analysis, background modelling uncertainties were seen to be some of the largest.
In the future versions of this analysis, improved understanding of the background can reduce the systematic uncertainty down to $5-10\%$.
Such precision is highly anticipated in many global fits, such as the one performed by the SIMBA collaboration~\cite{Bernlochner:2020jlt}.
While the current results may not provide a significant impact on the theoretical averages, they serve as a stepping stone for future Belle~II radiative hadronic-tagged analyses that will be leading contenders in the field of flavour physics.






