\chapter{Data Analysis Techniques}

\section{Fitting}

The distributions of observables that are measured experimentally always contain, at the very least, a degree of statistical uncertainty.
In Physics, we usually model processes with smooth distributions, which can be parametrised with a set of numbers. 
\textit{Fitting} is a process of extraction of parameters from observed distributions. 
It is one of the key tasks of statistical analyses and, consequentially, particle physics measurements. 
Nearly every particle physics analysis features some kind of fitting for the result extraction. 
It is also present in particle trajectory reconstruction algorithms, calorimeter cluster shape parametrisation, and calibration procedures.
One of the most common examples is the fitting of a particle species invariant mass distribution reconstructed from its decay products.
It was, for example, famously used in the discovery of the Higgs boson \cite{ATLAS:2012yve,CMS:2012qbp}, in the  $\H\ra\Z\Zstar\ra 4\ell$ and $\H\ra\g\g$ decay channels.

Fitting consists of two main steps: point estimation and uncertainty (confidence-interval) estimation. 
In the former, a best estimate for a set of parameters is derived, which describes a given dataset.
The latter sets the confidence interval on each parameter estimate in the set. 

The overview here presents only a brief overview of the most relevant method that was used in the work presented in this thesis. 
For a more in-depth consideration of statistical methods in data science and physics, I refer the readers to Refs. \cite{Behnke:2013pga,Blobel_Lohrmann_1998} and for rigorous proofs of the underlying statistical concepts, Refs. \cite{Bohm:2014vmk,James_2006,Barlow:1990vc}. 
The material presented here only summarises the details in these books and articles.
\subsection{Maximum-likelihood method}\label{sec:mle}

One of the most common and popular methods for fitting is the maximum-likelihood method. It is also used here widely in Chapter \todo[inline]{give chapter name}

An observed dataset can be defined as $x = (x_1, x_2,...,x_N)$, with $x_i$ being the result of $N$ independent measurements following an unknown probability density $f(x)$, and $i\in[0,N]$.
The function $f(x)$ is, usually, not known and the shape is often parametrised as $f(x; a)$, where $a=(a_1,...,a_M)$ is a dimension-$M$ vector of unknown-valued parameters. 
For a fitting procedure, one has to construct an \textit{estimator}, which is a function of the observed data that can provide an estimated numerical value, $\hat{a}$, corresponding to the parameter vector $a$.
% Any estimator should satisfy four main criterion:

% \begin{itemize}
%     \item \textbf{Consistency}: as the number of measurements increases, the precision of the estimator increases, i.e.:
%      \begin{equation}
%         \lim_{N\ra\infty}\hat{a} = a,
%      \end{equation}
%     \item \textbf{Bias}: the expectation value of the estimator should be close to the true value of $a$. 
%     If the difference, $b$, is small, it is known as an unbiased estimator:
%     \begin{equation}
%         E[\hat{a}] = a + b,
%     \end{equation}
%     \item \textbf{Efficiency}: The variance of the estimator, $V[\hat{a}]$, should be small. 
%     More formally speaking it should be higher than the inverse of Fisher information, $I(a)$. This is also known as the minimum-variance bound.
%     \begin{equation}
%         V[\hat{a}] \geq \frac{1}{I(a)} \quad \mathrm{with} \quad I_{jk}(a) = -E\left[\sum^N_{i=1}\pdv{f(x_i;a)}{a_j}{a_k}\right],
%     \end{equation}
%     \item \textbf{Robustness}: against wrong inputs or incorrect assumptions.
% \end{itemize}
%Usually, it is difficult to choose an estimator that can satisfy all of these criterion simultaneously. One common estimator is the maximum-likelihood estimator.

A very common technique for an estimator is the so-called maximum-likelihood estimate, which uses the likelihood function.
The likelihood function is built from one- or multi-dimensional PDFs $f(x;a)$ of the measured values $x$:
\begin{equation}\label{eq:likelihood_equation}
    \mathcal{L}(x;a) = \prod_{i=1}^N f(x_i; a).
\end{equation}
% The PDFs used to construct the likelihood function have to be normalised, which implies that the integral of the likelihood function is independent of parameters $a$:
% \begin{equation}
%     \int f(x;a)dx = 1 \quad \ra \quad \int \mathcal{L}(x;a)dx_1dx_2...dx_N = 1.
% \end{equation}
In this case, the maximum-likelihood estimate of the parameters $a$ correspond to $\hat{a}$ for which $\mathcal{L}(x;a)$ is globally maximised.
Because the product of a large number of components can vary over many orders of magnitude, in real-life applications it is more convenient to work with sums.
Therefore, a log-likelihood function is used in practice:
\begin{equation}
    l(x;a) \equiv \ln\mathcal{L}(x;a) = \sum_{i=1}^N \ln f(x_i;a).
\end{equation}
It is worthwhile to note that a logarithm is a monotonic function, therefore the maximum of a function is the same as the maximum of its logarithm. 
The maximum of the log-likelihood function satisfies the standard requirement for an extremum point:

\begin{equation}\label{eq:likelihood_derivative}
    \pdv{l(x;a)}{a_j}=0 \quad \mathrm{for} \quad j=1,...,M.
\end{equation}
The roots of \Cref{eq:likelihood_derivative} are maximum-likelihood estimates of $\hat{a}$. Generally, it is impossible to find an extremum in a large parameter space using analytical methods.
In practice, numerical procedures and dedicated software packages for optimisation are usually used.
Many optimisers used in modern day computers tend to minimise functions, rather than maximise them, therefore a \textit{negative log-likelihood} function is often used.
Maximum-likelihood method is unbiased and consistent as the number of measurements grows, i.e. $N\ra\infty$.
However, it requires a good prior knowledge of the form of the PDF $f(x;a)$, as otherwise the result might be incorrect.

\subsection{Variance of the maximum-likelihood method}\label{sec:mle_variance}


The likelihood function generally can have any non-Gaussian shape, however it can be shown~\cite{James_2006} that in the asymptotic limit, $N\ra\infty$, 
any given $f(x;a)$ can have its likelihood function approximated with a multivariate Gaussian distribution:

\begin{equation}
    \mathcal{L} \propto \exp{-\frac{1}{2}(a-\hat{a})^T H(a-\hat{a})},
\end{equation}
where $H$ is the Hessian matrix of the log-likelihood function. Assuming a good minimum is found (\Cref{eq:likelihood_equation}), the log-likelihood function can be expanded at $a=\hat{a}$ and approximated as a parabola:
\begin{equation}
    l(x;a_1,a_2,...,a_N) = l(x;\hat{a}_1,\hat{a}_2,...,\hat{a}_N) + \frac{1}{2}\sum_{i,k} \pdv{l}{a_i}{a_k} (a_i-\hat{a}_i)(a_k-\hat{a}_k)+...
\end{equation}
In this case, the covariance matrix, $V(\hat{a})$, of the estimated parameter vector is approximated as the inverted Hessian matrix, taken at the maximum-likelihood estimate $\hat{a}$:
\begin{equation}
    V(\hat{a}) = \left[-\pdv[2]{l(x;a)}{a}\middle|^{}_{\ds a=\hat{a}}\right]^{-1} = H^{-1}.
\end{equation}
This gives a symmetric uncertainty for each estimated parameter $a_j$ as:
\begin{equation}
    \hat{\sigma}_{a_j} = \sqrt{\hat{V}_{jj}(\hat{a})}.
\end{equation}
This method is always an approximation of the true covariance matrix, because the likelihood function shape is approximated as a parabola.

A more rigorous approach is to profile the likelihood function in order to calculate a likelihood-based confidence interval.
A profile likelihood ratio for a single parameter, $a_k$, is defined as:
\begin{equation}
    \lambda_{\mu} = \frac{l(x;a_k,\hat{\hat{a}})}{l(x;\hat{a_k},\hat{a})},
\end{equation} 
where the numerator is the log-likelihood for a set of parameters $\hat{\hat{a}}$ estimated for some given value of $a_k$. 
The denominator is the likelihood evaluated at the global extremum of the likelihood. 
The method follows the likelihood ratio function from the minimum to find where it crosses the value corresponding to a desired confidence interval.
In general, this leads to asymmetric uncertainty bars and a different result than uncertainties estimated by the Hessian matrix inversion method.
It is interesting to note, that in the asymptotic limit, both the profiling and Hessian matrix inversion methods give the same results.
However, reevaluating the likelihood for every given value of $a_k$ requires a minimisation of all other parameters, therefore the profiling method can be computationally intensive. 
Therefore, it is often sufficient to apply the Hessian matrix inversion method, as long as additional checks are performed to ensure that the provided uncertainties are accurate.

In particle physics, a commonly used minimisation software is \texttt{Minuit} \cite{James:1975dr,James:2296388}.
It implements the Hessian matrix inversion as the \texttt{HESSE} method and the likelihood-based uncertainty estimation method as \texttt{MINOS}.
\subsection{Extended maximum-likelihood}

In particle physics, it is common to not only parametrise a shape of a distribution of an observable, but also measure the absolute rate (normalisation) of the distribution.
The standard setup for maximum-likelihood method does not allow to determine the absolute normalisation. 
An additional term to the likelihood is to be introduced.
In Nature, if a measurement is performed repeatedly, its rate will fluctuate according to Poissonian statisics.
Hence, the \cref{eq:likelihood_equation}, needs to be have a Poissonian term included:
\begin{equation}\label{eq:extended_likelihood}
    \mathcal{L}(x;a) = \frac{\nu^N}{N!}\exp(-\nu)\prod_{i=1}^N f(x;a),
\end{equation}
where $N$ is the observed number of events and $\nu$ is the expected, or `true', normalisation. 
Such a modified likelihoon is called \textit{extended} likelihood.
Taking the logarithm of \Cref{eq:extended_likelihood} gives:
\begin{equation}\label{eq:extended_log_likelihood}
    l(x;a) = -\nu + N\ln{\nu} + \sum_{i=1}^N \ln (f(x;a)) + C,
\end{equation}
where $C$ is independent of $a$ and $\nu$. 
The extended log-likelihood fitting otherwise follows the same procedure as \Cref{sec:mle,sec:mle_variance}.
As such, when optimising \Cref{eq:extended_log_likelihood} for $a$ and $\nu$, the constant parameter $C$ can be ignored.

\subsection{Unbinned maximum-likelihood fitting}

There are two ways that data can be arranged for a fit: 

\begin{itemize}
    \item Each event enters the likelihood function (\Cref{eq:likelihood_derivative} or \Cref{eq:extended_likelihood}) independently,
    \item Events are first grouped in intervals of the observable $x$ and the count of measurements falling into each interval are provided as inputs to the likelihood function.
\end{itemize}
The intervals are often referred to as \textit{bins}. 
Consequentially, the techniques are referred to as \textit{unbinned} and \textit{binned} fits, respectively.
In this thesis, unbinned maximum-likelihood fits have been used. 
They are computationally more intensive, but are always more statistically-optimal than binned fits. 

Throughout this thesis, fitting is implemented using the \texttt{zfit} framework \cite{ESCHLE2020100508}. 
It provides a Python-based library that is developed to fulfill particle physics fitting requirements.
The \texttt{zfit} framework implements minimisation using the \texttt{iminuit} minimiser \cite{iminuit} which is a Python-friendly implementation of \texttt{Minuit}.

\section{Classification}
\todo[inline]{This section needs to be called differently, maybe Decision Theory, let's see}
When collecting data in measurements, many independent observations (\textit{events}) are usually performed in order to get a statistically significant sample.
Resulting datasets generally contain a component of interest, often referred to as \textit{signal}, and many components which might show similar behaviour as the signal in certain distributions. 
The latter is commonly referred to as \textit{background}. Disentangling the signal and background contributions in a given dataset is an extremely important task in Big-Data fields, where there may be hundreds or thousands of sub-components contributing to background that may be misclassified as signal.

In particle physics, the signal component usually refers to a single or a group of decay channels, whose properties are being measured. 
The most straightforward approach is to separate signal and background events by imposing requirements on observables, that are typical or expected for signal.
A requirement on some observable is often referred to as a \textit{selection} or a \textit{cut}.
The downside to this is that there can be non-linear underlying correlations between different selections, which might make them less efficient at background process separation than they could be in conjunction with other observables. 
Therefore, a multidimensional observable space is often desired.
However, as the number of observables used in a selection grows, the tuning of such multi-dimensional selection becomes increasingly difficult.


This sections introduces some relevant techniques to combine the informational of all observables of an event. 
Particularly, it overviews multivariate classification algorithms (\MVA{s}) concepts and boosted decision trees (\BDT{s}). 
The material presented here is only a summary detailing techniques used in the analyses presented in the thesis.
For detailed overview of multivariate-classification techniques I refer the reader to Ref.\cite{Behnke:2013pga} and for their underlying statistical framework to \cite{Hastie_Tibshirani_Friedman_2001,bishop_2016}.

\subsection{Multivariate classification}

In this subsection, binary classification is implied, as this is the most relevant for the work presented in this thesis. 
In general, a multi-class \MVA classifier can be broken down into a series of binary \MVA classifiers.

Let us assume, an event is described by $N$ observables, $X=\{x_1,x_2,...,x_N\}$. This is also known as a \textit{feature~vecture}.
Formally, an \MVA classifier is a mapping function, $f$, that maps the $N$-dimenstional feature vector, $x$, to a single real number, $y$:
\begin{equation}
    \mathbb{R}^N \ra \mathbb{R}: y = f(x).
\end{equation}
This can then be interpreted as a hypersurface in an $N$-dimensional feature space.
Signal is usually classified as $y=1$, whereas background as $y=0$, although different labelling is possible and mostly conventional.
The classifier may therefore output any value between 0 and 1.

The goal of multivariate classification is to build a classifier based on a set of examples: $\{X, Y\} = \{(x_1,y_1),..., (x_M,y_M)\}$, where $M$ is the number of events, and $y_i\in{0,1}$ is an associated \textit{target}.
The building of such a classifier is called \textit{training}, and $\{X, Y\}$ is called a \textit{training sample}.
In general, it is also possible to perform a training where no targets, $Y$, are supplied. 
In this case, the training is called \textit{unsupervised} learning.
The methods used for data analysis in this thesis only use \textit{supervised} learning.

Mathematically, a \textit{loss function} is desired which returns the discrepancy between training targets, $Y$ and the prediction, $\hat{Y}$, corresponding to the training sample, $X$:
\begin{equation}
    \mathcal{L}(Y, \hat{Y} = f(X;a)),
\end{equation}
where $a$ is a set of internal degrees of freedom, describing the model. 
In this formalism, one can minimize the loss to extract a set of parameters that provide the smallest difference between $Y$ and $\hat{Y}$.
This problem is then equivalent as described in \Cref{sec:mle}, with a different likelihood function.

Classifiers with a smaller number of degrees of freedom tend to be more robust against statistical fluctuations of a sample (smaller variance).
However, this might make the model unable to learn the more intricate details in the training sample, and lead to bias when applied on a statistically-independent sample.
Balancing these two requirements is known as the \textit{bias-variance trade-off}. 
Usually, one uses a \textit{validation} sample, which is equivalent but statistically independent from the training sample.
An optimally trained classifier will perform the same on the validation and training samples. 
After the optimal training point of the classifier is reached, further training will only degrade the performance of the classifier on the validation sample.
This regime is called \textit{overtraining} and needs to be avoided.

\begin{figure}[htbp!]
    \missingfigure{ROC curve here, AUC + classifier separation}
\end{figure}

There are many different ways to test for overtraining, but all of them usually rely on comparing the classifier performance between the training sample and a third, independent \textit{testing sample}.
In particle physics, it is common to use the validation sample also as a testing sample, although formally they should be different.
To compare the performance on training sample versus testing sample a receiver operating characteristic (\ROC) is often used.
A \ROC curve represents the the true positive rate as a function of the false positive rate and is calculated by iterating over all possible \MVA classifier outputs.
Furthermore, an area-under-curve (\AUC) score can be defined which is computed by integrating the \ROC curve.
An $\mathrm{\AUC}=1$ would represent false positive rate of 0, and a true positive rate of 1. 
The least-optimal classifier would provide a random guess and have a false-positive and true-positive rate of 0.5, and $\mathrm{\AUC}=0.5$.

\subsection{(Boosted) decision trees}

\todo[inline]{I probably need to talk about non parametric vs parametric in previous chapter, list some examples and say that I'll only talk about BDT}

When combining many different features together, selecting an appropriately parametrised classifier model can nearly impossible.
Therefore, often non-parametric models are desired, and a common way to implement these are decision trees.

A decision tree is a classification technique where the $N$-dimensional feature space is divided based on a series of binary selections.
A simple model (often a constant) is used to make a split in each of the feature space regions, and the splitting process is then repeated on the two resulting regions.
The number of the selections is called the \textit{depth} of the classifier.
Each final region which has not been sub-splitted is called a \textit{leaf}. 
A set of selections leading to a leaf is known as a \textit{branch}.