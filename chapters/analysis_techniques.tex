\chapter{Data Analysis Techniques}

\section{Fitting}

The distributions of observables that are measured experimentally always contain, at the very least, a degree of statistical uncertainty.
In Physics, we usually model processes with smooth distributions, which can be parametrised with a set of numbers. 
\textit{Fitting} is a process of extraction of parameters from observed distributions. 
It is one of the key tasks of statistical analyses and, consequentially, particle physics measurements. 
One of the simplest examples is the fitting of a particle species invariant mass distribution reconstructed from its decay products.
Some of the examples, include the discovery of the Higgs boson \cite{ATLAS:2012yve}.

