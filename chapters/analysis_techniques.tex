\chapter{Data Analysis Techniques}

\section{Fitting}

The distributions of observables that are measured experimentally always contain, at the very least, a degree of statistical uncertainty.
In Physics, we usually model processes with smooth distributions, which can be parametrised with a set of numbers. 
\textit{Fitting} is a process of extraction of parameters from observed distributions. 
It is one of the key tasks of statistical analyses and, consequentially, particle physics measurements. 
Nearly every particle physics analysis features some kind of fitting for the result extraction. 
It is also present in particle trajectory reconstruction algorithms, calorimeter cluster shape parametrisation, and calibration procedures.
One of the most common examples is the fitting of a particle species invariant mass distribution reconstructed from its decay products.
It was famously used in the discovery of the Higgs boson \cite{ATLAS:2012yve,CMS:2012qbp}, in the  $\H\ra\Z\Zstar\ra 4\ell$ and $\H\ra\g\g$ decay channels.

Fitting consists of two main steps: point estimation and confidence-interval estimation. 
In the latter, a best guess for a set of parameters is derived, which describes a given dataset.
The former sets the confidence interval on each parameter in the set. 
In this section, the mathematical background for fitting will be presented.

\subsection{Loss}

