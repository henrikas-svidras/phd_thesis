\chapter{Introduction}\label{sec:introduction}

The Standard Model is a theory describing the fundamental particles and interactions that constitute the Universe.
The predictive power and precision of the Standard Model is arguably unmatched by any other theory in science.
The Higgs boson was proposed already in 1964 \cite{PhysRevLett.13.508}, more than 50 years before its discovery \cite{ATLAS:2012yve,CMS:2012qbp}.
The top quark was discovered \cite{PhysRevLett.74.2632,PhysRevLett.74.2626} two decades after the existence of the third generation of fermions was inferred \cite{HARARI1975265}.

However, some experiments do show, as of yet, unexplained tensions. 
A notable example is the measurement of the anomalous magnetic moment of the muon \cite{PhysRevLett.126.141801}.
Although the Standard Model is able to provide an astounding 8 significant digit agreement with experiment, there is a disagreement at higher precision.
Similarly, for the case of anomalous magnetic moment of electron, there is a tension between different experimental approaches \cite{PhysRevLett.100.120801,Morel:2020dww,Li:2021koa}.
The Standard Model has also not been able to provide a clear candidate for the excess matter that is observed in the Universe (dark matter).
Furthermore, the observed matter-antimatter asymmetry is larger than what is currently predicted in the Standard Model.
The mathematical framework of the Standard Model, while incredibly powerful, appears to be arbitrary, with a large number of free parameters in the Standard Model.
For example, the existence of exactly 3 generations of fermions with a mass hierarchy is not theoretically well-grounded, but seems too organised to be accidental.
There are many other experimental and theoretical inconsistencies (see e.g. Ref.\cite{Ellis:2002wba}) that cannot be addressed by the Standard Model in a satsifactory way.

These theoretical and experimental challenges point to the necessity of discoveries that could lead to new developments in one of the most successful theories of science.
While there are numerous extensions to the Standard Model available within the particle physics theory community, there is no clear evidence that any of them is the correct description of our Universe.
At the moment, it seems that hints about a clear direction can only come from more precise measurements that will test the Standard Model particle properties, interactions and process rates.


This thesis will describe the measurement of $B$ mesons decaying into a meson system originating from an $s$ quark, $X_s$, and an energetic photon \g.
I adopt the standard notation of decay processes and label it as \BtoXsgamma.
This decay channel is sensitive to particles not included within the Standard Model. 
Furthermore, the parameters of the photon energy spectrum are important inputs to other precision measurements of the Standard Model.
I will employ a technique called hadronic-tagging, which reconstructs a partnering $B$-meson in each event, yielding a purer final measured data-sample.
The data that is analysed was collected in 2019-2020 by the Belle II detector, which is located in Tsukuba, Japan at the KEK laboratory.
Belle II collects the data of electron-positron collisions, provided by the SuperKEKB collider.

These aforementioned points will be discussed in detail in this thesis.
The thesis is therefore split up in X chapters:
\todo[inline]{chapter later}
\todo[inline]{review and finish this chapter later}

