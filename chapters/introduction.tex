\chapter{Introduction}\label{sec:introduction}

The Standard Model is a theory describing the fundamental particles and their interactions that constitute the Universe.
The predictive power and precision of the Standard Model are arguably unmatched by any other theory in science.
The Higgs boson was proposed already in 1964 \cite{PhysRevLett.13.508}, more than 50 years before its discovery \cite{ATLAS:2012yve,CMS:2012qbp}.
The top quark was discovered \cite{PhysRevLett.74.2632,PhysRevLett.74.2626} two decades after the existence of the third generation of fermions was inferred \cite{HARARI1975265}.

However, some experiments do show so far unexplained tensions with the predictions of the Standard Model. 
A notable example is the measurement of the anomalous magnetic moment of the muon \cite{PhysRevLett.126.141801}.
Although the Standard Model can provide an astounding 8 significant digit agreement with the experiment, there is a disagreement at higher precision.
Similarly, in the case of the anomalous magnetic moment of the electron, there is a tension between different experimental approaches \cite{PhysRevLett.100.120801,Morel:2020dww,Li:2021koa}.
The Standard Model has also not been able to provide a clear candidate for the excess matter that is inferred in the Universe (dark matter).
Furthermore, the observed matter-antimatter asymmetry 
is larger than what is currently predicted in the Standard Model.
The mathematical framework of the Standard Model, while incredibly powerful, appears to be arbitrary, with numerous free parameters.
For example, the existence of exactly 3 generations of fermions with a mass hierarchy is not theoretically well-grounded but seems too organised to be accidental.
The strong force conserves the charge-parity symmetry, although it is not required by the Standard Model framework (strong CP problem).
There are many other experimental and theoretical inconsistencies (see Ref.~\cite{Ellis:2002wba} for a review) that cannot be adequately addressed by the Standard Model

These theoretical and experimental challenges motivate the search for discoveries that could lead to new developments in one of the most successful theories of science.
While numerous extensions to the Standard Model have been proposed by the particle physics theory community, there is no clear evidence that any of them is the correct description of our Universe.
At the moment, it seems that hints about a clear direction can only come from more precise measurements that will test the Standard Model particle properties, interactions and process rates.

This thesis contributes such a measurement in the radiative decay of the beauty quark, where it transitions to a strange quark and a photon.
The measurement will be performed by studying data of $B$ mesons decaying into a meson system originating from an $s$ quark, $X_s$, and an energetic photon \g.
The data that is analysed was collected in 2019-2020 by the Belle II detector, which is located in Tsukuba, Japan at the KEK laboratory.
Belle II collects data of electron-positron collisions, provided by the SuperKEKB collider.
The colliding electron and positron beams create \FourS resonances which decay into pairs of \B mesons.
The measurement presented here uses a technique called hadronic-tagging, 
which also reconstructs the partnering $B$-meson, yielding a purer final measured data sample.

The standard notation of the radiative $B$ meson decay processes is adopted to label it as \BtoXsgamma.
This decay channel is sensitive to particles not included within the Standard Model. 
Furthermore, the parameters of the photon energy spectrum are important inputs to other precision measurements of the Standard Model.


These aforementioned points will be discussed in detail in this thesis.
The thesis is therefore split up into 8 chapters, which introduce the theoretical background, experimental machinery and status of \BtoXsgamma, data analysis tools and discusses the measurement itself.
The current Chapter and \Cref{ch:summary} are dedicated introductory and summary Chapters, respectively.
The remaining Chapters are as follows:
\begin{itemize}
    \item \Cref{ch:theory} introduces the main concepts of the Standard Model that are necessary to understand the basis of \BtoXsgamma decays, their importance and possible future developments.
    It is not a thorough mathematical overview but only serves as a synthesis of important concepts and conclusions, with a wide array of literature referenced for a deeper dive into the theoretical aspects of the analysis.
    \item \Cref{ch:exp_overview} introduces the experimental status of radiative $B$ decays which is intended as an introductory chapter for experimental \BtoXsgamma analysis.
    The information provided in the Chapter is only a summary, and not part of the original work in this thesis.
    \item \Cref{ch:belle2} presents the experimental machinery (the SuperKEKB accelerator and the Belle~II detector) and software used to collect and process the data for this analysis.
    \item \Cref{ch:analysis_techniques} introduces import statistical and data science concepts relating to parameter estimation and multivariate analysis which are employed in the analysis of the Belle II data.
    \item \Cref{ch:analysis} presents the strategy and the analysis of the Belle II data leading to the measurement of \BtoXsgamma decay properties with the hadronic tagging technique.
    It presents a full overview of the evaluation of statistical and systematic uncertainties related to the analysis.
    \item \Cref{ch:overview} provides a comprehensive discussion of the results of \Cref{ch:analysis}, and compares them with experimental world averages and past measurements.
    It also overviews the prospects and the impact of the result presented in this thesis to the theoretical status of the Standard Model.
\end{itemize}
Additional information, which supports or further explains the information presented in this thesis is provided in the \Crefrange{sec:appendix_local_opperators}{sec:appendix_resolution_fits}.

