\chapter{Introduction}

The Standard Model, is a $SU(3)\times SU(2)\times U(1)$ quantum field theory \cite{Peskin:1995ev}.
Its theoretical framework describes the electroweak and strong force interactions between elementary particles that constitute the world as we know it.
The elementary particles included in the Standard Model can have a half-integer spin (fermions) or integer spin (bosons).
The spin-1 bosons are the mediators of the electromagnetic (photon), weak ($\Wpm$ and $\Z$) and strong interactions (gluon) in the SM.
The spin-0 Higgs boson couples to all massive particles of the Standard Model via the Higgs mechanism \cite{PhysRevLett.13.508}.
The fermions are subsplit into two additional groups, quarks and leptons, which, respectively, can and cannot interact through the strong force.

The predictive power and precision of the Standard Model is arguably unmatched by any other theory in science.
The Higgs boson has been proposed already in 1964 \cite{PhysRevLett.13.508}, more than 50 years before its discovery \cite{ATLAS:2012yve,CMS:2012qbp}.
The top quark was discovered \cite{PhysRevLett.74.2632,PhysRevLett.74.2626} two decades after the existence of the third generation of fermions was inferred \cite{HARARI1975265}.

However, some experiments do show, as of yet, unexplained tensions. 
A notable example is the measurement of anomalous magnetic moment of the muon \cite{PhysRevLett.126.141801}.
Although the Standard Model is able to provide and astounding 8 significant figure agreement with the experiment, there is a disagreement at higher precision.
Similarly, for the case of anomalous magnetic moment of electron, there is a tension between different experimental approaches \cite{PhysRevLett.100.120801,Morel:2020dww,Li:2021koa}.
The Standard Model has also not been able to provide a clear candidate for the excess matter that is observed in the Universe (dark matter).
Furthermore, the observed matter-antimatter asymmetry is larger than what is currently predicted in the Standard Model.
In general, the amount of free parameters in the Standard Model, as well as existence of exactly 3 generations of fermions with a mass hierarchy seem arbitrary and are not theoretically well-grounded.

These theoretical and experimental challenges point to the necessity of finding inconsistencies within the Standard Model, that could lead to new developments in one of the most successful theories in Science.
While there are numerous extensions of the Standard Model available within the particle physics theory community, there is no clear evidence that any of them is right.
Hints about a clear direction can only come from more precise measurements that will test the Standard Model particle properties, interactions and process rates.

This thesis will describe the measurement of the $b$-quark decaying into an $s$-quark and a photon. 
This decay manifests as a $B$ meson decaying into a meson-system originating from the $s$ quark, $X_s$, and a high-energy photon \g.
We adopt the notation \BtoXsgamma.
We analyse the data collected in 2019-2020 by the Belle II detector, which is located in Tsukuba, Japan at the KEK laboratory.
Belle II collects the data of electron-positron collisions, provided by the SuperKEKB accelerator.

\todo[inline]{All this needs citations}

\BtoKstargamma \cite{PhysRevLett.71.674}
\BtoXsgamma \cite{PhysRevLett.74.2885}