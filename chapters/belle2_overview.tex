\chapter{Experimental setup}\label{chap:belle2}

A common strategy to study \B meson decays is by dedicated colliders, known as $B$ factories.
$B$~factory experiments operate by producing large quantities of \BB pairs, through the creation of \FourS mesons that primarily decay to two \B meson pairs (\BB) \cite{Workman:2022ynf}.
Historically, two $B$ factory experiments have operated:
\begin{itemize}
    \item BaBar with the PEP-II accelerator at SLAC, USA;
    \item Belle with KEKB accelerator at KEK, Japan.
\end{itemize}
Although not a $B$ factory, another experiment, known as LHCb, with the LHC accelerator at CERN, collects data from $B$ mesons that are produced in proton-proton collisions.
It has been running since 2008 and is still in operation.

The two aforementioned $B$ factory experiments have since completed their operation.
However, an upgraded version of Belle, known as Belle~II, began collecting data in 2018.
Belle~II is a detector at the KEK laboratory in Tsukuba, Japan.
Its main purpose is the collection of electron-positron (\epem) collision data 
at the center of mass energies ($\sqrt{s}$) at or near the \FourS meson mass.
The colliding beams are provided by the SuperKEKB \epem collider.
This chapter provides an overview of Belle II and introduces the mains concepts of the SuperKEKB accelerator.

\todo[inline]{add quotes.}

\section{The SuperKEKB accelerator}\label{sec:superkekb}

The SuperKEKB accelerator, discussed in detail in Ref.\cite{Akai:2018mbz}, is a double-ring electron-positron collider.
It is an upgraded version of the KEKB collider \cite{Oide:2009zz} that operator with the Belle experiment, a predecessor to Belle~II.
The SuperKEKB accelerator complex is schematically shown in \Cref{fig:superkekb}.
A photo-cathode radio-frequency gun produces two electron beams.
The first is subsequently accelerated to 7~\gev by a linear accelerator into the electron ring.
On the other hand, protons are created by directing the second electron beam to a tungsten target.
%, producing bremmstrahlung photons which create electron and positron pairs.
The positrons are singled out using the magnetic field and accelerated to 1.1~\gev, injected them into the damping ring and, finally,
accelerated by the linear accelerator to 4~\gev before the injection to the positron ring.
The subsequent collision occurs inside the Belle~II detector, where the two electron and positron rings meet (see \Cref{sec:belle2}).
\begin{figure}[htbp!]
    \centering
    \includegraphics[width=0.6\textwidth]{figures/experimental_setup/super_kekb.png}
    \caption{\label{fig:superkekb}
        The schematic visualisation of the SuperKEKB accelerator complex.
        The main components that contribute to the acceleration of electrons and positrons are shown.
        The four straight sections are named after Japanese cities.
        Credit to \cite{Akai:2018mbz}.
    }
\end{figure}

It is important to emphasise that an \epem collision at $10.58~\gev$ produces more than just \FourS.
In truth, many other decay products including $\epem\ra\ell^+\ell^-$ or $\epem\ra\qqbar$ proceses may occur and the production 
cross-section  of all these processes depends on $\sqrt{s}$.
This is shown for $\sqrt{s}\approx10.58~\gev$ in \Cref{fig:cross_sections}, with more details about the exact values of the cross-sections provided in \Cref{sec:appendix_major_production_cross_sections}.
\begin{figure}[htbp!]
    \includegraphics[width=1\textwidth]{figures/experimental_setup/corss_sections.pdf}
    \caption{\label{fig:cross_sections} Relative comparison of the largest $\epem\ra\ X$ production cross sections at $B$~factories.
    The numbers composing the charts are listed in \Cref{sec:appendix_major_production_cross_sections} and taken from \cite{Belle-II:2018jsg}.
    }
\end{figure}

Although by far the largest cross-section are related to the so-called low-multiplicity processes 
(such as $\epem\ra\epem$ and $\epem\ra\mumu$, see \Cref{sec:appendix_major_production_cross_sections}),
they differ largely from typical \FourS\ra\BB events.
On the other hand, the continuum processes (\epem\ra\qqbar ($q\in\{u,d,c,st\}$) and \epem\ra\tautau) 
events are a significant background process for many analyses aiming to measure \B meson decays
\footnote[1]{Because of the large $\epem\ra\qqbar$ and $\epem\ra\tautau$ production cross-sections, 
$B$ factories are also used to study $\tau\pm$ or $D$ meson and similar decays.
In such cases the continuum events may, in fact, be events of interest.}.

The beam-energy asymmetry (7~\gev for \en and 4~\gev for \ep) is an important design characteristic of SuperKEKB, 
which allows to separate $B$ meson vertices, necessary for measurements such as time-dependent CP violation \cite{BaBar:2014omp}.
On the other hand, the exact collision energy is chosen in order to operate at $\sqrt{(7+4)^2-(7-4)^2}\approx10.58~\gev\approx m(\FourS)$, 
hence fulfilling the requirements of the $B$~factory experiment.
The machine may also run at a 60~\mev lower collision energy, in which case $\epem\ra\FourS$ events cannot be produced.
Such data, containing only low-multiplicity and continuum events is called \textit{off-resonance} data.
Conversely, the conventional aforementioned setup is referred to as \textit{on-resonance} data.

\section{The Belle II experiment}\label{sec:belle2}

The Belle II detector, discussed in detail in Ref. \cite{Belle-II:2010dht}, is designed to reconstruct the final states of electron-positron 
collisions at center-of-mass energies at or near the \FourS meson mass.
The colliding beams are supplied by the SuperKEKB accelerator as discussed in \Cref{sec:superkekb}.

The Belle~II operates since 2018 and as of start of 2023 has collected 364~\invfb of on-resonance and 42~\invfb of off-resonance data.
The final goal is to reach 50~\invab, which will be nearly 50 times that of the combined datasets of all $B$ factories so far.

Belle II consists of several detector subsystems arranged cylindrically around the beam pipe. 
\Cref{sec:pxd} introduce the Belle~II subsystems.
The visual representation of the detector is given in \Cref{fig:belle2_detector} and shows the components of the detector, as well as their acronyms.
\begin{figure}[htbp!]
    \includegraphics[width=1\textwidth]{figures/experimental_setup/belle2.png}
    \caption{\label{fig:belle2_detector} The schematic representation of the Belle~II detector.
    The Belle~II cylinder is approximately 7 meters in diameter and 7.5 meters long.
    The description of all sub systems is provided in \Cref{sec:pxd,sec:svd,sec:cdc,sec:pid,sec:magnet,sec:ecl,sec:klm}.
    Figure taken from \cite{belle_2_picture}.
    Credit to the Belle~II collaboration.
    }
\end{figure}

Belle II coordinate system is defined as follows: the $x$ axis is defined to be horizontal and points to the outside of the tunnel with respect to the accelerator's
main rings, the $y$ axis is vertically upward, and the $z$ axis is defined in the direction of the electron beam. 
The azimuthal angle, $\phi$, and the polar angle, $\theta$, are defined with respect to the $z$ axis. 
Three regions in the detector are defined based on $\theta$: 
\begin{itemize}
    \item forward endcap (\mbox{$12^{\circ}<\theta<31^{\circ}$}), 
    \item barrel (\mbox{$32^{\circ}<\theta<129^{\circ}$}),
    \item backward endcap (\mbox{$131^{\circ}<\theta<155^{\circ}$}).
\end{itemize}


\subsection{Pixel detector}\label{sec:pxd}

Closest to the collision point is the Belle~II Pixel Detector (\PXD) \cite{Belle-II:2010dht}.
Its main purpose is the high-precision detection of short-lived particle decay vertices, such as $B$ mesons.

\PXD contains two layers that use depleted $p$-channel field-effect transistors (\DEPFET) to detect charged particles that pass through it \cite{Kemmer:1986vh}.
The first layer is situated at a 14~\mm radius from the collision point, whereas the second at 22~\mm.
By design, each layer consists of 16 and 24 sensor modules, which which are glued in pairs to make 8 and 12 ladders
\footnote[1]{Until the scheduled 2023 Belle~II upgrade the outer layer contains only 2 ladders.}, respectively.
Each module contains $768\times250$ \DEPFET pixels.

The \PXD is designed to operate in large radiation conditions which it is subjected to due to the proximity to the collision point,
while maintaining high precision vertex reconstruction and a low material budget.
As a result, the sensors are able to withstand a $20~M\rad$ 
radiation dose and have a $\sim 0.2\%$ radiation length per layer \cite{Belle-II:2010dht},
while maintaining an average spatial resolution of approximately $15~\mu m$ and a hit efficiency of 99\% percent after 4 years of data taking \cite{Belle-IIDEPFET:2022wis} .
The detector is shown in \Cref{fig:pxd}, whereas its schematic location in the Belle~II detector are shown in \Cref{fig:vxd}.

\begin{figure}[htbp!]
    \centering
    \subcaptionbox{\label{fig:pxd}}{
        \includegraphics[width=0.45\textwidth]{figures/experimental_setup/pxd.png}
    }
    \subcaptionbox{\label{fig:svd}}{
        \includegraphics[width=0.45\textwidth]{figures/experimental_setup/svd.png}
    }
    \caption{\label{fig:pxd_svd} 
    The Belle II pixel detector \Cref{fig:pxd} and silicon vertex detector \Cref{fig:svd}.
    They are installed in the Belle~II detector as shown in \Cref{fig:vxd}.
    The \PXD is arranged into two layers that are designed to contain 16 modules in the first layer and 24 in the second.
    Each two layers are glued into a ladder.
    The \SVD is composed of four layers around the \PXD, with a total of 172 double-sided silicon strip sensors.
    Credit to Belle~II PXD and SVD groups.}
\end{figure}

\subsection{Silicon vertex detector}\label{sec:svd}

The silicon vertex detector (\SVD) \cite{Belle-IISVD:2023mxk} surrounds the \PXD and 
is the second subsystem responsible for charged particle detection.
Its main roles include the reconstruction of short-lived particle decay vertices in conjunction with PXD,
standalone charged particle trajectory reconstruction (for low momentum particles),
and particle identification information through specific ionisiation measurements.

The \SVD contains four layers. 
The first layer is composed of 7 ladders with 2 sensors,
the second -- of 10 ladders with 3 sensors,
the third -- of 12 ladders with 4 sensors,
and the final of 16 ladders with 5 sensors.
The 172 double-sided silicon strip sensors are composed, in total, of 224 thousand strips.
Each sensor is based on an $n$-type bulk implanted with $p$ and $n$-doped sensitive strips.
The $p$ and $n$ strips are aligned perpendicularly and on opposite sides of the sensor.
The schematic representation of \SVD is shown in \Cref{fig:svd}, whereas its schematic location in the Belle~II detector are shown in \Cref{fig:vxd}.

As charged particles pass through the \SVD sensors, the electrons and holes created through ionisation drift to $p$ and $n$ strips, respectively.
The perpendicularity of the strips ensures that the spatial coordinates of the passing particle can be inferred.

\begin{figure}[htbp!]
    \centering
    \includegraphics[width=0.6\textwidth]{figures/experimental_setup/vxd.png}
    \caption{\label{fig:vxd}
    The Belle II pixel detector and silicon vertex detector shown inside of the Belle~II detector.
    The size of the both vertex detection components, as well as the interaction point are noted.
    Credit to \cite{Belle-IISVD:2023mxk}.
    }
\end{figure}

\subsection{Central Drift Chamber}\label{sec:cdc}

The central drift chamber (\CDC) \cite{Taniguchi:2017not} is the central subsystem responsible for the reconstruction of charged particle trajectories inside the Belle~II experiment.
As such, its main objective is the detection of particle momenta and charge.
The \CDC also provides particle identification information through specific ionisation measurements
and participates in the decision to save the event information (\textit{triggering}).

It is a large volume drift chamber filled with a 50\% helium and 50\% ethane mixture.
The \CDC begins immediately after \SVD at $160~\mm$ and is contained within an outer cylinder radius of $1130~\mm$
It consists of 14336 readout wires distributed across 56 layers.
Each readout is surrounded by 8 field wires that create an electric field in the chamber.
As a charged particle passes through the chamber ionising the gas,
the electrons are accelerated in the electric field creating avalanches that are read out as signal by the wires.
In order to obtain three-dimensional information about the particle trajectory,
some layers in the \CDC are skewed.
The first 8 layers are axial, whereas the rest alternate between axial and skewed every 6 layers.
The grouping of layers is shown in \Cref{fig:cdc_quadrant}, whereas \Cref{fig:cdc_wires} illustrate the difference between axial ad skewed layers.

The \CDC covers $\theta\in(17\deg,150\deg)$ range, and provides a highly accurate measurement of charged particle trajectories with a $0.1-0.2~\cm$ spatial resolution 
and a $\sim0.5\%$ momentum resolution for the majority of particles resulting in \B meson decays \cite{Kandra:2019qlz}.

\begin{figure}[htbp!]
    \centering
    \subcaptionbox{\label{fig:cdc_quadrant}}{
        \includegraphics[width=0.45\textwidth]{figures/experimental_setup/cdc_quadrant.png}
    }
    \subcaptionbox{\label{fig:cdc_wires}}{
        \includegraphics[width=0.45\textwidth]{figures/experimental_setup/cdc_layers.png}
    }
    \caption{\label{fig:cdc}
    The Belle~II central drift chamber schematic representation.
    \Cref{fig:cdc_quadrant} shows a quadrant of the \CDC in the $r$-$\phi$ plane.
    Different axial and skewed layer groups are visible.
    \Cref{fig:cdc_wires} show the axial (upper) and (skewed) wires.
    The skew is exaggerated for illustrative purposes.
    Credit to \cite{BelleIITrackingGroup:2020hpx}.
    }
\end{figure}

\subsection{Particle identification systems}\label{sec:pid}

Belle~II particle identification is performed using two systems: aerogel ring imaging Cherenkov counter (\ARICH) in the forward endcap region
and a time-of-propagation (\TOP) chamber in the barrel region: both detectors being located outside of the \CDC.
Both of these detectors are tasked with the distinction of different charged particle species.

The \ARICH detector \cite{Yusa:2014tua} consists of an array of silica aerogel used as a radiator.
As charged particles pass through the material, it emits Cherenkov photons, which are detected by a subsequent photon sensor.
This working principle is depicted in \Cref{fig:arich}.
The angle of the emitted Cherenkov light, $\theta_C$ can be used to calculate its velocity, $\beta$, given the refractive index of the radiator material,$n$:
\begin{equation}
    \beta = \frac{1}{n\cdot\theta_C}.
\end{equation}
The velocity information, combined the with the knowledge of particle's momentum from the \CDC, \SVD and \PXD allows to identify the species of a particle (through its mass).
\ARICH is designed to provide separation information for $\pi^{\pm}$ and $K^{\pm}$ in $(0.4,4)~\gevc$ momentum range, and for $\pi^{\pm},\mu^{\pm},e^{\pm}$ below 1~\gevc.

\begin{figure}[htbp!]
    \centering
    \subcaptionbox{\label{fig:arich}}{
        \includegraphics[width=0.4\textwidth]{figures/experimental_setup/arich.png}
    }
    \subcaptionbox{\label{fig:top}}{
        \includegraphics[width=0.45\textwidth]{figures/experimental_setup/top.png}
    }
    \caption{\label{fig:pid}
    The schematic working principle of the Belle~II particle identification detectors: aerogel ring-imaging Cherenkov counter (\Cref{fig:arich}) and a time-of-propagation chamber.
    \ARICH covers the forward endcap region, whereas \TOP covers the barrel.
    Credit to \cite{Yusa:2014tua} and \cite{Fast:2017pff}, respectively.
    }
\end{figure}

The \TOP detector \cite{Fast:2017pff} consists of sixteen $270 \times 45 \times 2$~\cm quartz radiator bars.
The working principle is illustrated in \Cref{fig:top}.
One end of the quartz crystal contains a spherical mirror that reflects the light to the opposite end containing photomultiplier tube arrays.
Due to a high refractive index of quartz, the Cherenkov light emitted by passing particles undergoes a total internal reflection and propagates to the photo-multiplier tubes.
Given a precisely known angle of the incoming particle, the time-of-propagation inside the chamber is a function of $\theta_C$.
The time of arrival of the photons is compared to the expectated distributions for different particle hypotheses ($e^{\pm},\mu^{\pm},\pi^{\pm},K^{\pm},p^{\pm}$) and corresponding likelihood values are computed for each (see \cite{Yonenaga:2020eby} for details).

\TOP provides an 85\% identification efficiency for $K^{\pm}$ at a 10\% $\pi^{\pm}$ misidentification rate \cite{Kojima:2022qcl}, whereas
\ARICH has a 94\% efficiency for $K^{\pm}$ identification with an 11\% $\pi^{\pm}$ misidentification rate \cite{Yonenaga:2020eby}.

\subsection{Electromagnetic calorimeter}\label{sec:ecl}

Belle~II reuses the calorimeter of Belle, with an upgraded readout electronics \cite{Belle-II:2010dht}.
The electromagnetic calorimeter (\ECL) surrounds the aforementioned detector systems and covers the barrel region and both endcap regions.
It is the main subdetector for photon detection and energy measurements.
Furthermore, \ECL is able to provide information necessary to differentiate electrons from hadrons, participates in $K_L$ detection (together with \KLM),
supplies triggering information, and measures the luminosity collected by the detector.

The \ECL covers $\theta\in(12.4,155.1)~\deg$ region and is composed of 8736 cesium iodide crystals \cite{Miyabayashi:2020xzp}.
Each crystal is aproximately 16 radiation lengths long \cite{Aulchenko:2015nvy}.
The rear surface of the crystal contain glued photodiodes with preamplifiers.
As electromagnetically interacting particles pass through the calorimeter they induce cascades of particle through interaction with the dense detector material (electromagnetic showers).
The photodiodes read out the final products of the shower and convert it to a signal that is digitised.
The sketch of a single \ECL crystal is given in \Cref{fig:ecl}.

\ECL has an excellent performance: photon-energy resolution which varies from 4~\% at 100~\mev \cite{Miyabayashi:2020xzp} to 2~\% at 5~\gev \cite{Aulchenko:2015nvy},
a position resolution of $5-10~\mm$ \cite{Miyabayashi:2020xzp}, and a mass-resolution of 5~\mevcc(12~\mevcc) of for the composite \piz ($\eta$) meson \cite{Miyabayashi:2020xzp}.

\begin{figure}[htbp!]
    \centering
    \includegraphics[width=0.45\textwidth]{figures/experimental_setup/ecl.png}
    \caption{\label{fig:ecl}
    A schematic depiction of one of the crystals that comprise the Belle~II electromagnetic calorimeter.
    A signal resulting from the shower of an electromagnetically interacting particle reaches the photodiode and is amplified by the preamplifier.
    Credit to \cite{Miyabayashi:2020xzp}.
    }
\end{figure}

\subsection{Superconducting magnet}\label{sec:magnet}

A superconducting solenoid surrounds the \ECL \cite{Belle-II:2010dht}.
The coil is made from a niobium-titanium-copper alloy and is wound around an aluminium support cylinder.
The cooling is performed using a liquid helium cryogenic system.
It generates a $1.5~\mathrm{T}$ magnetic field, which is crucial to bend the charged particle trajectories, enabling a transverse momentum measurement and charge separation.
The magnetic field is directed along the $z$ direction, and is measured to be homogeneous and vary less at $\order(1\%)$ in the entire volume \cite{BelleIITrackingGroup:2020hpx}.

\subsection{\texorpdfstring{$K_L$}{KL} and \texorpdfstring{\mu}{mu} detector}\label{sec:klm}

The $K_L$ and $\mu$ detectors (\KLM) \cite{Aushev:2014spa} are the outermost subsystem of Belle~II.
\KLM is composed of alternating layers of up to $4.7~\cm$ thick iron plates and detector active parts.
The iron plates decelerate the traversing particles and also act as a return yoke for the magnet.

The barrel and endcap regions differ by design \cite{Krohn:317929}.
The barrel region contains 14 iron layers and 15 detector layers, which are aligned parallel to the $z$ axis.
Here, two innermost detector layers are instrumented with scintillators, whereas the remaining use resisitve plate chambers.
The endcap region contains 14 iron and detector layers each, which are aligned perpendicular to the $z$ axis.
Conversely, all 14 detectors layers use plastic silicon strips with silicon photomultipliers.

The $K_L^0$ interact with the nuclei in the iron plates and cause hadronic showers, 
which are read out by the silicon strip detectors or the resistive plate chambers.
This process may already occur in the \ECL, which is why it is also a part of $K_L^0$ detection.
Non-showering particles, such as $\mu$ with momentum larger than 0.6~\gevc, traverse the detector in a straight line, 
depositing only small amounts of energy in the system.

\begin{figure}[htbp!]
    \subcaptionbox{\label{fig:rpc_klm}}{
        \includegraphics[width=0.45\textwidth]{figures/experimental_setup/rpc_klm.png}
    }
    \subcaptionbox{\label{fig:scintilator_klm}}{
        \includegraphics[width=0.45\textwidth]{figures/experimental_setup/scintilator_klm.png}
    }
    \caption{\label{fig:klm}
        The Belle~II $K_L^0$ and $\mu$ detector active detector layers in the barrel (\Cref{fig:rpc_klm}) and the endcap regions (\Cref{fig:scintilator_klm}).
        Iron plates between are sandwiched between the detector layers.
        Credit to \cite{Krohn:317929} and \cite{Aushev:2014spa}, respectively.
    }
\end{figure}

\section{The Belle II software}\label{sec:belle2_software}

The Belle~II analysis software (\basftwo) \cite{Kuhr:2018lps} is an open-source collection of tools developed to utilise in collision event reconstruction, analysis 
and any other tasks necessary for the physical interpretation of the data recorded by the Belle~II detector.
It is primarily based on \texttt{Python} and \texttt{C++} programming languages.
This section briefly introduces the charged and neutral particle reconstruction strategies, which are managed by \basftwo.

\subsection{Charged particle reconstruction}\label{sec:tracking}

Tracking is used to refer to the charged particle trajectory reconstruction.
The naming is used widely in high-energy physics and stems from the \textit{tracks} that particles leave in the detectors.

The tracking process in Belle~II is discussed broadly in Ref.\cite{BelleIITrackingGroup:2020hpx}.
Each particle trajectory (henceforth -- track) is modelled as a helix with 5 parameters:
\begin{itemize}
    \item $d_0$: the distance of the point of closest approach to the $z$ axis;
    \item $\phi_0$: the angle between the transverse momentum and the $x$ axis at the point of closest approach;
    \item $\omega$: the track curvature signed with the particle charge;
    \item $z_0$: the $z$ coordinate at $d_0$;
    \item $\tan\lambda$: the tangent of the dip angle.
\end{itemize}

The tracks are reconstructed by combinining the information from \CDC and/or \SVD with information from \PXD, if present.

\subsection{Photon reconstruction}\label{sec:neutrals}

Each particle interacting in the \ECL deposits energy into the crystal it interacts with.
This deposited energy, and the time of each energy deposit, is recorded by the calorimeter.
Clusters are sets energy deposits in the \ECL from the interaction of a single particle that traversed the detector.
A graphical illustration of the reconstruction of a cluster in an experimental environment is shown in \Cref{fig:clustering}.
\begin{figure}[htbp!]
    \subcaptionbox{\label{fig:clustering1}}{
    \includegraphics[width=0.4\textwidth]{figures/experimental_setup/clustering1.png}
    }
    \subcaptionbox{\label{fig:clustering2}}{
        \includegraphics[width=0.4\textwidth]{figures/experimental_setup/clustering2.png}
    }
    \caption{\label{fig:clustering}
    The Belle~II calorimeter, shown in the $\theta-\phi$ plane, 
    showcasing energy deposits (each point corresponds to a single \ECL crystal) in a single simulated photon event in the center of the image.
    The low-energy deposits resulting from background radiation are included in \Cref{fig:clustering1}.
    The cluster reconstruction algorithm of \basftwo singles out the cluster from the single photon, rejecting background as seen in \Cref{fig:clustering2}.
    Credit to Belle~II neutrals group.
    }
\end{figure}

The identification of photons exploits the fact that the energy deposited in the cluster
by an incident photon has a cylindrical symmetry in the lateral direction with an exponentially decreasing energy deposition away from the incident axis.
On the other hand, neutral or charged hadrons interactions tend to produce more asymmetric shower shapes.
To distinguish photons from charged particles, particularly electrons, tracks are extrapolated to the \ECL and compared for consistency with reconstructed clusters.
