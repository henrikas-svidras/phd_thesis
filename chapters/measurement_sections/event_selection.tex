The previous section overviewed the samples that are reconstructed using basic requirements laid out in \Cref{sec:tag_reconstruction,sec:gamma_reconstruction}.
In this section concrete selections will be discussed that will lead to background suppression and a best photon candidate
and best tag-candidate selection.

\subsection{Primary photon candidate selection}\label{sec:primary_photon_candidate_selection}
Contrary to the tag side, a selection of the the best photon candidate in the range $\EB>1.4~\gev$ is effectively trivial based on the discussion in \Cref{sec:event_reconstruction}.
As for 99.7\% and 99.8\% of the signal \MC sample the highest \EB photon is the correct photon originating from \BtoXsgamma decay,
this is chosen as then best photon-candidate requirement with virtually no signal efficiency loss.
Judging from \Cref{fig:photon_reco_candidates}, this provides an approximately 3\% background suppression.

\subsection{Main photon backgroundes sources}\label{sec:main_background_sources}

Judging from \Cref{fig:spectrum_after_reco}, number of photon and tag candidates originating 
in non-\BB events is significantly larger than number of \B meson events.
The proportion of \qqbar to \BB event candidates is 92.5\% to 7.5\% for \FEI \Bp mode;
and 91.7\% to 8.3\% for \FEI \Bz.

The majority of background photon candidates originate in \piz\ra\g\g or \eta\ra\g\g decays.
This in total accounts for nearly 80\% of background photon candidates.

