\textit{Thrust axis} $\vec{T}$ is defined in terms of $N$ momenta $\vec{p}_i,(i\in{1,2,..,N})$.
It is the unit vectory, which maximises the projection of the $\sum_i\vec{p}_i$.
The scalar observable known simply as \textit{thrust} is then defined as \cite{BaBar:2014omp}:
\begin{equation}\label{eq:thrust}
    \mathbf{T} = \frac{\sum^N_{i=1}|\mathbf{T}\vec{p}_i|}{\sum^N_{i=1}|\vec{p}_i|}.
\end{equation}

The reason why `thrust'-like observables can be utilised to distinguish \epem\ra\qqbar and \BtoXsgamma events follows the same logic as the one sketched in \Cref{fig:continuum_schematic}.
The decay particles of a $B$ tend to be spherically distributed in the detector, with a uniformly distributed $T\in(0,1)$.
For \qqbar events, their decay particles tend to be directional, and $T$ tends to unity.

Based on these definitions more thrust-related distributions can be defined.
Six thrust-related variables are tested in this analysis:
\begin{itemize}
    \item $\cos\theta_{\mathrm{TB}\wedge\mathrm{TO}}$: the cossine of the angle between the thrust of the tag-candidate $B$ meson ($B$ meson decay particle momenta evaluated in the collision center-of-mass frame),
    and the thrust of all the other particles;
    \item $\cos\theta_{\mathrm{TB}\wedge\mathrm{z}}$: the cossine of the angle between the thrust of the tag-candidate $B$ meson,
    and the $z$-axis of the detector;
    \item $T_{\mathrm{B}}$: the thrust of the tag-candidate $B$ meson;
    \item $T_{\mathrm{O}}$: the thrust of all particles \textit{except} the tag-candidate $B$ meson;
    \item $T$: the thrust of all particles in the event;
    \item $\cos\theta_{\mathrm{T}}$: the polar angle component of $\vec{T}$.
\end{itemize}

