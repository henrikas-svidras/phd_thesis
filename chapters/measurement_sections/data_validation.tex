Up until now the disucssion of the analysis revolved around simulation studies.
The full analysis procedure on simulation was defined and shown to produce an unbiased, stable result in \Cref{sec:MC_validation}.
Then, \Cref{sec:corrections} looked at the corrections required to accurately correct differences expected in simulation to better represent data.
At this stage, all appropriate measures have been taken and the analysis can be fully applied on Belle II data.
The concepts of a blinded analysis, however, dictate that to ensure no biases are present the full analysis procedure must be performed in validation samples.
For the \BtoXsgamma analysis four validation samples are defined:
\begin{itemize}
    \item \epem\ra\qqbar sample, that contains collision data, collected 60~\mev below the \FourS center-of-mass energy.
    \item Sample with enhanced \epem\ra\qqbar, where the \texttt{BDT~output} score requirement is inverted: $\mathtt{BDT~output}<0.4$.
    \item Sample with enhanced \BB background, where the $\piz$ and $\eta$ veto requirements are inverted.
    \item $1.4<\EB<1.8$~\gev and $2.7<\EB$~\gev regions, where signal-to-background ratio is small, and signal is kinematically forbidden, respectively.
    Note that some signal events may still be present in $2.7<\EB$~\gev due to resolution effects, but they are not expected to be statistically significant.
\end{itemize}
In this Section, the analysis selections, \Mbc fitting and leftover-\BB background subtraction will be investigated thoroughly using these samples.

\subsection{Validation on the \texorpdfstring{\epem\ra\qqbar}{e+e- -> qqbar} off-resonance sample}\label{sec:continuum_spectrum_validation}

The validation on \epem\ra\qqbar events is performed using only \epem\ra\qqbar simulation.
The goal of this validation is to ensure that continuum backgrounds are described by the simulated samples correctly.
Although \Cref{sec:continuum_validation} partially ensures this, only the distribution shape requirements were tested there.
Furthermore, the best candidate selection, which was developped on \BB samples, may change the conclusions that were found earlier.

All the corrections for \piz and \eta veto, \FEI calibrations are applied as discussed in \Cref{sec:corrections}.
Full analysis selection procedure involving the most optimal selections and the best-tag side candidate selection are applied, as presented in \Cref{sec:selection_summary,sec:select_best_candidate}.
The results are shown in \Cref{fig:offresonance_validation}.
\begin{figure}[htbp!]
    \subcaptionbox{\label{fig:bplus_offresonance_eb}}{
        \includegraphics[width=0.31\textwidth]{figures/data_validation/Bplus_offresonance_EB.pdf}
    }
    \subcaptionbox{\label{fig:bzero_offresonance_eb}}{
        \includegraphics[width=0.31\textwidth]{figures/data_validation/Bzero_offresonance_EB.pdf}
    }
    \subcaptionbox{\label{fig:bboth_offresonance_eb}}{
        \includegraphics[width=0.31\textwidth]{figures/data_validation/Bboth_offresonance_EB.pdf}
    }
    \caption{\label{fig:offresonance_validation} Validation of the \EB distribution of \epem\ra\qqbar events.
    Excellent agreement is observed in \feiBp (\Cref{fig:bplus_offresonance_eb}, \feiBz (\Cref{fig:bzero_offresonance_eb}),
    and the combined sample (\Cref{fig:bboth_offresonance_eb}).
    The uncertainty for data contains only the statistical component.
    The simulation uncertainty contains statistical and systematic uncertainties corresponding to \Cref{tab:correction_table}.
    }
\end{figure}

Overall, the agreement of continuum data and simulation is excellent.
Due to rather low size of the off-resonance data sample and the strong continuum suppression in this analysis, the amount of statistical uncertainties are relatively large.
It is concluded, that the \EB distribution of \epem\ra\qqbar events is well-modelled in simulation and follows the expectations seen in earlier Sections.

Due to known issues with beam energy values in the off-resonance data, which affected the \Mbc calculation (but not overall validity of other values), 
the \epem\ra\qqbar off-resonance \Mbc distributions do not accurately represent the \Mbc values of continuum events.
Therefore, these samples are not used for \Mbc distribution and \Mbc fitting validation.

\subsection{Validation on \texorpdfstring{\epem\ra\qqbar}{e+e- -> qqbar} enhanced sample}\label{sec:continuum_mbc_validation}

As it was mentioned in \Cref{sec:continuum_spectrum_validation}, due to problems in \Mbc distribution, 
the dedicated \epem\ra\qqbar-only samples were not used to validate the continuum simulation.
However, an alternative validation sample is prepared, where the continuum component is enhanced.
This is achieved by inverting the \texttt{BDT~output} selection (see \Cref{sec:continuum_suppression}) thereby suppressing \BB events.
To ensure a minimal amount of \BtoXsgamma events in the sample (as the \texttt{BDT~output} is not perfectly efficient), here \piVeto and \etaVeto
are also inverted.
This creates a sample with mostly \epem\ra\qqbar events and small components of \BB events.
The inverted values are chosen as $\mathtt{BDT~output}<0.2$ and $\piVeto>0.4$ and $\etaVeto>0.4$.
The resulting \EB spectra for both \FEI modes are shown in \Cref{fig:qqbar_enhanced_eb_validation}.
\begin{figure}[htbp!]
    \subcaptionbox{\label{fig:bplus_qqbar_enhanced_eb}}{
        \includegraphics[width=0.31\textwidth]{figures/data_validation/Bplus_qqbar_enhanced_eb.pdf}
    }
    \subcaptionbox{\label{fig:bzero_qqbar_enhanced_eb}}{
        \includegraphics[width=0.31\textwidth]{figures/data_validation/Bzero_qqbar_enhanced_eb.pdf}
    }
    \subcaptionbox{\label{fig:bboth_qqbar_enhanced_eb}}{
        \includegraphics[width=0.31\textwidth]{figures/data_validation/Bboth_qqbar_enhanced_eb.pdf}
    }
    \caption{\label{fig:qqbar_enhanced_eb_validation} The \EB distribution of \qqbar enhanced samples (see \Cref{sec:continuum_spectrum_validation}).
    Adequate agreement is observed in \feiBp (\Cref{fig:bplus_offresonance_eb}, \feiBz (\Cref{fig:bzero_offresonance_eb}),
    and the combined sample (\Cref{fig:bboth_offresonance_eb}).
    The uncertainty for data contains only the statistical component.
    The simulation uncertainty contains statistical and systematic uncertainties corresponding to \Cref{tab:correction_table}.
    }
\end{figure}

Although the agreement is generally adequate, particularly in the signal region, although a small excess of events is observed in low-\EB region.
Similarly, the resulting \Mbc distributions are shown in \Cref{fig:qqbar_enhanced_mbc}.
A striking difference from the generally good agreement observed so far can be seen for \feiBp, \feiBz and the combined sample.
A larger amount of continuum events, particularly at low-\Mbc is present and,
moreover, the high-endpoint of the \Mbc is shifted.
The overall data-to-simulation discrepancy can be at around twenty percent.
\begin{figure}[htbp!]
    \centering
    \subcaptionbox{\label{fig:Bplus_qqbar_enhanced_mbc}}{
        \includegraphics[width=0.3\textwidth]{figures/data_validation/Bplus_qqbar_enhanced_mbc.pdf}
    }
    \subcaptionbox{\label{fig:Bzero_qqbar_enhanced_mbc}}{
        \includegraphics[width=0.3\textwidth]{figures/data_validation/Bzero_qqbar_enhanced_mbc.pdf}
    }
    \subcaptionbox{\label{fig:Bboth_qqbar_enhanced_mbc}}{
        \includegraphics[width=0.3\textwidth]{figures/data_validation/Bboth_qqbar_enhanced_mbc.pdf}
    }
    \caption{\label{fig:qqbar_enhanced_mbc} The \Mbc distribution of \qqbar enhanced samples (see \Cref{sec:continuum_spectrum_validation}).
    Some clear differences in \feiBp (\Cref{fig:bplus_offresonance_eb}), \feiBz (\Cref{fig:bzero_offresonance_eb}),
    and the combined sample (\Cref{fig:bboth_offresonance_eb}) can be observed.
    Particularly, it is evident that there are more continuum events at low-\Mbc, and the \Mbc high-endpoint is shifted to lower values.
    These results motivate the modification of the \Mbc fitting procedure, discussed in XXXX.
    \todo[inline]{discussed in XXX}
    }
\end{figure}

These differences are understood as a result of two reasons related to the fact that data-taking period independent simulation is used in this analysis.
In normal data collection conditions, the collision energy, $\sqrt{s}$ is not perfectly stable: minute variations or drifts can occur over time.
These can be related to many reasons including intentional, such as various tests, and unintentional, such as 
The data-taking period independent simulation does not account for these changes, with a collision energy simply set to a predetermined target value.
The \Mbc endpoint is directly affected by the set $\sqrt{s}$, as seen in \Cref{eq:mbc_exclusive}, with lower values of $\sqrt{s}$ making some values of \Mbc kinematically forbidden.
On the other hand, a larger overall amount of \qqbar events is understood as a consequence of the fact that collecting the data at lower collision energies (but now lower than \FourS energy) enhances the \epem\ra\qqbar process cross-section.
Altogether, this leads to more continuum events present in the data sample than predicted by the data-taking period independent simulation.

Note that for correctly reconstruced \B mesons (i.e. good tag-\B mesons) the shift would not occur.
This is a result of the fact that $p_B^*$ (as seen in \Cref{eq:mbc_exclusive}) is directly related to the total energy of the collision.
Lower collision energies will simply lower $p_B^*$, as the resonant-like behaviour in \Mbc is driven by the \B meson mass.
Such constraints are not present for misreconstructed events, therefore shifts are expected to happen there.

Although the most robust solution is the usage of data-taking period independent simulation, at the time of preparation of this analysis such simulation was noy yet fully available at Belle II.
While future studies will be able to rely on it, in this analysis additional steps were taken to account for this.

In particular, while \Mbc is strongly affected, the \EB spectrum is still well-described.
This is a consequence of the fact that \EB and \Mbc are not strongly correlated, or more plainly, \EB does not depend as strongly on $\sqrt{s}$ as \Mbc.
Therefore, a correction is only necessary for \Mbc distribution \textit{and} only for combinatorial-\BB and continuum events.
An \textit{ad hoc} approach is developed, where the \Mbc distribution for simulated events only is shifted manually. 
The procedure is as follows:
\begin{itemize}
    \item Count the frequencies of each $\sqrt{s}$ value occurring in the Belle II on-resonance dataset.
    \item Randomly remove half of the beam energies in simulated Belle II dataset of \textit{events where no good tag-\B mesons are present}.
    \item Replace the removed beam energies with the values of the first step, based on the frequencies they occur at in the Belle~II on-resonance dataset.
\end{itemize}
The reason why only $50\%$ of energies are replaced is to minimise any potential bias that such a procedure could introduce.
Note that the replacement of $\sqrt{s}$ only affects the \Mbc calculation and not other observables which may be rely on $\sqrt{s}$ in their definition.
The result of the correction on the \Mbc distribution of the sample of both \FEI modes combined is shown in \Cref{fig:qqbar_enhanced_mbccorrected}.
\begin{figure}[htbp!]
    \centering
    \includegraphics[width=0.45\textwidth]{figures/data_validation/Bboth_qqbar_enhanced_mbccorrected.pdf}
    \caption{\label{fig:qqbar_enhanced_mbccorrected} The \Mbc distribution of \qqbar enhanced samples (see \Cref{sec:continuum_spectrum_validation}),
    where an \Mbc correction has been applied to the simulated distribution.
    Although the correction does not perfectly correct the distributions, the signal region ($\Mbc\approx5.28~\gevcc$) is described correctly.}
\end{figure}
Although perfect correction is not achieved via this \textit{ad hoc} correction, but the peak-region is described correctly, which is evident when comparing \Cref{fig:qqbar_enhanced_mbccorrected} and \Cref{fig:qqbar_enhanced_mbc}.

A key point to discuss here is the effect the different \Mbc shape may have on the \Mbc fitter.
The differences in the tail, and the end-point are expected to not strongly affect the result, because the \Mbc fitter is prepared with shape differences and accounts for them
by estimating parameters $c$ and $m_0$ of the argus distribution, as well ass the overall $\frac{\mathcal{N}_{\mathrm{CHEB}}}{\mathcal{N}_{\mathrm{ARGUS}}}$ ratio.
However, the \Mbc fitter initial parameters in \Cref{tab:fitting_init_params} have to be updated to emphasise that a different `corrected'-\Mbc is the new fitting observable.
The \Mbc fitter is therefore updated, following the exact same procedures as \Cref{sec:fitting_setup} and the new values are given in \Cref{tab:fitting_init_params_updated}.
\begin{table}[htbp!]
    \centering
    \caption{\label{tab:fitting_init_params_updated} The summary of the fitting model used in this analysis for the \Mbc fit after updating the initial values to correspond for the correction in \Mbc distributions of background, as discussed in \Cref{sec:continuum_spectrum_validation}.
    The paramaters are initialised at the values that are listed, corresponding to the ones determined in the primary fitting steps, explained in \Cref{sec:crystal_ball_prefit,sec:chebyshev_prefit,sec:argus_prefit}, with \Mbc replaced by a `corrected'-\Mbc value.
    The values that are bolded in the table are not estimated from the final \Mbc fit, but are kept at their initialised values.
    On the other hand, all non-bolded values are estimated from the final fitter.
    The uncertainties are those estimated using the \texttt{HESSE} method.
    }
\resizebox{1\textwidth}{!}{
    {\def\arraystretch{1.5}\tabcolsep=5pt
        \begin{tabular}{|c|c|c|c|c|c|c|c|c|c|c|c|c|c|c|}

            \hline
            \multirow{2}{*}{\EB bin} & \multicolumn{5}{|c|}{Crystal Ball} & \multicolumn{6}{c|}{Chebyshev}    & \multicolumn{3}{c|}{Argus} \\
            \cline{2-15}
                                       & $\mathcal{N}_{\mathrm{CB}}$ &$\boldsymbol{\mu}$ & $\boldsymbol{\sigma}$ & $\boldsymbol{\alpha}$ & $\mathbf{n}$ & $\mathcal{N}_{\mathrm{cheb}}$  & $\mathbf{k_1}$ & $\mathbf{k_2}$ & $\mathbf{k_3}$ & $\mathbf{k_4}$ & $\mathbf{k_5}$ & $\mathcal{N}_{\mathrm{ARGUS}}$ & $c$ & $m_0$                 \\
            \hline
            1.4 -- 1.6               & $17294\pm131$ &\multirow{11}{*}{$\mathbf{5.279}$} & \multirow{11}{*}{$\mathbf{0.003}$} & \multirow{11}{*}{$\mathbf{1.573\pm0.035}$} & \multirow{11}{*}{$\mathbf{3.561\pm0.22}$} & $70507\pm 266$& $\mathbf{-0.150\pm0.007}$ & $\mathbf{-0.382\pm0.007}$ & $\mathbf{-0.272\pm0.006}$ & $\mathbf{-0.132\pm0.006}$  & $\mathbf{-0.003\pm0.006}$  & $76798\pm277$ &  $-26.35 \pm    0.81$ & \multirow{11}{*}{$5.2897$} \\
            \cline{1-2}\cline{7-14}
            1.6 -- 1.8               & $10218\pm101$&                                      &                                        &                                   &                                  & $33666\pm183$ & $\mathbf{-0.084 \pm 0.010}$ & $\mathbf{-0.411\pm0.010}$ & $\mathbf{-0.300 \pm  0.009}$ & $\mathbf{-0.140\pm0.009}$ & $\mathbf{-0.003\pm0.009}$ & $50658\pm225$ & $-21.08 \pm 0.99$ & \\               
            \cline{1-2}\cline{7-14}
            1.8 -- 2.0               & $5947\pm77$&                                      &                                        &                                   &                                  &$16192\pm127$ & \multirow{9}{*}{$\mathbf{0.030\pm0.011}$} & \multirow{9}{*}{$\mathbf{-0.438\pm0.011}$} & \multirow{9}{*}{$\mathbf{-0.377\pm 0.009}$} & \multirow{9}{*}{$\mathbf{-0.175\pm0.010}$} & \multirow{9}{*}{$\mathbf{-0.007 \pm 0.010}$} & $31228\pm176$ & \multirow{9}{*}{$-18.64 \pm 0.93$} & \\                         
            \cline{1-2}\cline{7-7}\cline{13-13}
            2.0 -- 2.1               & $1938\pm44$&                                      &                                        &                                   &                                  & $4279\pm65$ &                 &                  &                   &                  &                    & $9983\pm100$ &                   & \\                                                                  
            \cline{1-2}\cline{7-7}\cline{13-13}
            2.1 -- 2.2               & $1246\pm35$&                                      &                                        &                                   &                                  & $2589\pm51$&                 &                  &                   &                  &                    & $6951\pm83$&                   & \\   
            \cline{1-2}\cline{7-7}\cline{13-13}
            2.2 -- 2.3               & $909\pm30$&                                      &                                        &                                   &                                  & $1581\pm40$&                &                  &                   &                  &                    &  $4470\pm67$&                  & \\   
            \cline{1-2}\cline{7-7}\cline{13-13}
            2.3 -- 2.4               & $985\pm31$&                                      &                                        &                                   &                                  & $1197\pm35$&                &                  &                   &                  &                    & $2663\pm52$&                   & \\   
            \cline{1-2}\cline{7-7}\cline{13-13}
            2.4 -- 2.5               & $1213\pm35$&                                     &                                        &                                   &                                  &$779\pm28$  &                &                  &                   &                  &                    & $1482\pm39$ &                  & \\   
            \cline{1-2}\cline{7-7}\cline{13-13}
            2.5 -- 2.6               & $626\pm25$&                                     &                                        &                                   &                                  &$310\pm18$ &                 &                  &                   &                  &                    &  $662\pm26$&                  & \\   
            \cline{1-2}\cline{7-7}\cline{13-13}
            2.6 -- 2.7               & $62\pm8$&                                     &                                        &                                   &                                  & $52\pm7$ &                &                  &                   &                  &                    & $211\pm15$ &                   & \\   
            \cline{1-2}\cline{7-7}\cline{13-13}
            2.7 -- 5.0               & $1\pm1$&                                     &                                        &                                   &                                  & $6\pm2$&                &                  &                   &                  &                    & $73\pm9$&                  & \\   
            \hline
        \end{tabular}
    }
}

\end{table}

This fitter is applied on the continuum-enhanced samples of Belle~II data used in this analysis,
and the  fitted \Mbc distributions are shown in \Cref{tab:fitting_init_params_updated}.
The figures also contain the number of good tag-\B mesons that are estimated by the fitter.
It can be seen that no statistically significant peak in \Mbc is extracted.
Therefore it can be concluded that the validation of the \Mbc fitter is successful: enough flexibility is seen
to describe the slight difference in \Mbc shapes between simulation and real data, without introducing a bias towards the extracted number of good tag-\B mesons.


\begin{figure}[htbp!]
    \centering
    \subcaptionbox{\label{fig:mbc_qqbar_ehnhanced_data_1p4}}{
        \includegraphics[width=0.3\textwidth]{figures/data_validation/qqbar_ehnaced_fits/full_MbcFit_1p4to1p6ppdf.pdf}
    }
    \subcaptionbox{\label{fig:mbc_qqbar_ehnhanced_data_1p6}}{
        \includegraphics[width=0.3\textwidth]{figures/data_validation/qqbar_ehnaced_fits/full_MbcFit_1p6to1p8ppdf.pdf}
    }
    \subcaptionbox{\label{fig:mbc_qqbar_ehnhanced_data_1p8}}{
        \includegraphics[width=0.3\textwidth]{figures/data_validation/qqbar_ehnaced_fits/full_MbcFit_1p8to2p0ppdf.pdf}
    }
    \subcaptionbox{\label{fig:mbc_qqbar_ehnhanced_data_2p0}}{
        \includegraphics[width=0.3\textwidth]{figures/data_validation/qqbar_ehnaced_fits/full_MbcFit_2p0to2p1ppdf.pdf}
    }
    \subcaptionbox{\label{fig:mbc_qqbar_ehnhanced_data_2p1}}{
        \includegraphics[width=0.3\textwidth]{figures/data_validation/qqbar_ehnaced_fits/full_MbcFit_2p1to2p2ppdf.pdf}
    }
    \subcaptionbox{\label{fig:mbc_qqbar_ehnhanced_data_2p2}}{
        \includegraphics[width=0.3\textwidth]{figures/data_validation/qqbar_ehnaced_fits/full_MbcFit_2p2to2p3ppdf.pdf}
    }
    \subcaptionbox{\label{fig:mbc_qqbar_ehnhanced_data_2p3}}{
        \includegraphics[width=0.3\textwidth]{figures/data_validation/qqbar_ehnaced_fits/full_MbcFit_2p3to2p4ppdf.pdf}
    }
    \subcaptionbox{\label{fig:mbc_qqbar_ehnhanced_data_2p4}}{
        \includegraphics[width=0.3\textwidth]{figures/data_validation/qqbar_ehnaced_fits/full_MbcFit_2p4to2p5ppdf.pdf}
    }
    \subcaptionbox{\label{fig:mbc_qqbar_ehnhanced_data_2p5}}{
        \includegraphics[width=0.3\textwidth]{figures/data_validation/qqbar_ehnaced_fits/full_MbcFit_2p5to2p6ppdf.pdf}
    }
    \subcaptionbox{\label{fig:mbc_qqbar_ehnhanced_data_2p6}}{
        \includegraphics[width=0.3\textwidth]{figures/data_validation/qqbar_ehnaced_fits/full_MbcFit_2p6to2p7ppdf.pdf}
    }
    \subcaptionbox{\label{fig:mbc_qqbar_ehnhanced_data_2p7}}{
        \includegraphics[width=0.3\textwidth]{figures/data_validation/qqbar_ehnaced_fits/full_MbcFit_2p7to5p0ppdf.pdf}
    }
    \caption{\label{fig:mbc_qqbar_ehnhanced_fits}The fits on Belle II dataset corresponding to 189~\invfb with selection that enhances \epem\ra\qqbar events, as discussed in \Cref{sec:continuum_spectrum_validation}.
    The fitting model from \Cref{tab:fitting_init_params_updated} is used, which is defined on the `corrected'-\Mbc to account for variations $\sqrt{s}$ in Belle II data.
    Good description of the \Mbc distributions can be seen throughout the \EB bins.
    As the continuum component in each \EB bin is enhanced, not good tag-\B mesons are expected and indeed - no statistically significant peak is extracted from the results.
    }
\end{figure}

\subsection{Validation on \texorpdfstring{\BB}{BB}-background enhanced sample}\label{sec:bb_background_validation}

In the last Section, the validation on a sample with \epem\ra\qqbar events was performed.
An alternative, and perhaps more important validation given the background subtraction step (\Cref{sec:background_subtraction}),
is the \BB background decription in simulated data.
