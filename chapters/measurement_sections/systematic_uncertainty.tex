Most systematic uncertainties, or the grounds for introducing them, have already been introduced in this analysis.
The main systematic uncertainties arise from selection, background modelling and efficiency corrections.
Additional, subleading uncertainties that arise from unfolding and \BtoXdgamma component subtraction are included.
At the moment, the systematic uncertainties (except for the low-\EB) are sub-dominant compared to the much larger statistical component.
Many systematic uncertainties are set to their conservative estimates, 
which highlights that the analysis can be further optimised for future versions, as the Belle~II data set increases.

In this section the already presented information about systematic uncertainties will be condensed and finalised.

\todo[inline]{I need to make a table with all corrections here}

\subsection{Uncertainties due to background normalisation}\label{sec:background_normalisation_systematic}

\Cref{sec:sidebands_validation} introduced that the average number of 
good tag-\B mesons in $\EB\in(1.4,1.8)$ in data and simulation differs by $8.7\%$.
This difference is illustrated in \Cref{fig:sidebands_subtracted_figures}.
This difference of 8.7\% may be partially correlated to the differences in the \Mbc endpoint (discussed broadly in \Cref{sec:bb_background_validation}).
It could also partially be related to \ZMVA discrepancy, that is illustrated in \Cref{fig:zmva_discrepancy}.
As these effects are difficult to disentangle without additional, independent studies of the, the full 8.7\% correction is assigned as a systematic uncertainty.
Therefore the background expectations are varied by $\pm 8.7\%$, then subtracted from data, and the full extent of the variation is assigned as a systematic uncertainty to the measured \BtoXsgamma yields.
The uncertainties are not considered correlated to account for the fact that the background shape can vary from that of simulation as a function of \EB.

\subsection{Uncertainties due to \texorpdfstring{\Mbc}{Mbc} endpoint variations}\label{sec:fit_endpoint_systematic}

As discussed in \Cref{sec:bb_background_validation}, the fit model is modified to account for $\sqrt{s}$ in Belle~II data manifesting as a shift in \Mbc.
To account for an imperfect correction to the fit model, a systematic uncertainty is calculated here.
The fit on data is performed with and without the \Mbc correction introduced.
The full variation $\Delta\mathcal{N}_{\mathrm{CB}}$ between the estimated $\mathcal{N}_{\mathrm{CB}}$ in the two scenarios is assigned as a systematic uncertainty.
The uncertainties are not considered correlated to account for possible changed in the model variations as a function of \EB.

The evaluation of theses uncertainties is performed in a blinded way (that is only $\Delta\mathcal{N}_{\mathrm{CB}}$  is evaluated).
This also serves as a test ensuring that the fit converges on the full data set.

\subsection{Uncertainties due to selection modelling}\label{sec:correction_systematic}

The selection modelling affects the final result through background modelling (when it is subtracted from the \Mbc fit on Belle~II data),
and through \BtoXsgamma efficiency calculations.
The available corrections based on results from independent studies are summarised in \Cref{sec:corrections}.
The corresponding uncertainties from the corrections are propagated as systematic uncertainties.
The background yields attain uncertainties from \FEI calibration factors, \piVeto and \etaVeto modelling, photon-detection efficiency, background modelling. 
The \BtoXsgamma efficiency obtains uncertainties from the same sources, with the exception of background modelling, which is not applicable.

All the different ucnertainties are considered as correlated between different \EB intervals.
The background modelling ucnertainties follow the correlations depicted in \Cref{fig:bbar_correlation_matrix}.
The uncertainties related to other corrections are considered fully correlated across different \EB intervals.

\subsection{Uncertainties due to \texorpdfstring{\BtoXdgamma}{B->Xd gamma}}\label{sec:xdgamma_systematic}

The uncertainty due to \BtoXdgamma component modelling has the strategy largely discussed in \Cref{sec:xdgamma_modelling}.
The measured number of $\BtoXsdgamma$ events is corrected based on the theoretical \BtoXdgamma expectation, which amounts to roughly 4\% of the value.
The full difference $N(\BtoXdgamma) \equiv N(\BtoXsdgamma)-N(\BtoXsgamma)$ is assigned as a systematic uncertainty.
The systematic uncertainty is considered as uncorrelated between different \EB bins to account for possible spectrum shape variations.

\subsection{\texorpdfstring{\EB}{EB} spectrum unfolding uncertainties}\label{sec:unfolding_systematic}

The unfolding uncertainties are calculated for every unfolding factor calculated in \Cref{sec:signal_unfolding}.
Each correction factor obtains an uncertainty based on the hybrid signal-model, shown in \Cref{fig:hybrid_uncertainty}.
The unfolding factor uncertainties are propagated appropriatiely for the number of $\BtoXsdgamma$ events measured in the appropriate \EB interval.
The unfolding uncertainties are considered as correlated based on the correlation matrix evaluates when building the hybrid signal-model.

\subsection{Uncertainty on the number of \texorpdfstring{\B}{B} mesons in the sample}\label{sec:b_meson_uncertainty}

The number of \B mesons 