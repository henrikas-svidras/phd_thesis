After the fitting procedure in \Cref{sec:fitting_mbc} the good tag-\B counts are estimated, and in particular:
\begin{itemize}
    \item all \epem\ra\qqbar contributions have now been removed;
    \item events where the tag-side \B mesons are not properly reconstructed have been removed.
\end{itemize}

However, it may seem odd that after performing the \Mbc fit (see \Cref{fig:mc_fit_yield_comparisons}), the result is still not quite comparable to an \EB spectrum,
resembling that of the background seen in, e.g., \Cref{fig:spectrum_after_optimisation}.
Since in the inclusive treatment the $X_s$ is not constrained, a component of \textit{irreducible} background will always be present.
In the case of this analysis, the good tag-\B counts after the optimal selection in \Cref{tab:cutflow} and the fitting procedure presented here contain:
\begin{itemize}
    \item correctly tagged \BtoXsgamma events,
    \item correctly tagged-\BB events other than \BtoXsgamma.
\end{itemize}
Although in the future versions of this analysis it may be possible to further diminish the second component through improved selection and fitting procedures (see discussion in \Cref{sec:future_prospects}), 
some irreducible background component will always remain.

The final step before the full \BtoXsgamma spectrum extraction in \MC is evaluating the size of the remaining \BB background.
Two strategies are considered:
\begin{itemize}
    \item Count good tag-\B mesons in each \EB interval that correspond to non-\BtoXsgamma in \MC and use these values to subtract remaining \BB events from the fitted values.
    This is a simple method which is independent of the fitting procedure.
    \item Perform the fit again on the simulated sample with \BtoXsgamma events removed.
    This way, good tag-\B meson counts are estimated with fitting effects included.
    This effectively corresponds to the data points in \Cref{fig:no_bxsgamma_mc_fit_yield_comparisons}.
\end{itemize}
Although both choices are valid and have certain advantages, in this analysis, the second method is chosen.
The main reason for this choice is the fact that biases or systematic effects in the fit result, if such exist, could be suppressed when subtracting.
It also requires a similar validation procedure (e.g. to be an unbiased estimator) as the total fit.
Therefore, after the \Mbc fit, the number of correctly-tagged \BtoXsgamma events in data will be measured as follows:
\begin{equation}\label{eq:background_subtraction}
    N_{\mathrm{DATA}}^{\BtoXsgamma} = \mathcal{N}_{\mathrm{DATA}}^{\mathrm{good~tags}}  - \frac{189~\invfb}{1600~\invfb}\cdot \mathcal{N}_{\mathrm{MC}}^{\mathrm{good~tags~with~\BtoXsgamma~removed}}.
\end{equation}