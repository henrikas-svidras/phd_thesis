After the fitting procedure introduced in \Cref{sec:fitting_mbc} the good tag-\B counts are estimated, and in particular:
\begin{itemize}
    \item all \epem\ra\qqbar contribution has now been removed;
    \item events where the tag-side \B mesons are not properly reconstructed have been removed.
\end{itemize}

However, it may seem odd that after performing the \Mbc fit (see \Cref{fig:mc_fit_yield_comparisons}), the result is still not quite comparable to an \EB spectrum,
and resembles that of the background seen in, e.g., \Cref{fig:spectrum_after_optimisation}.
Since in the `inclusive' treatment the $X_s$ is not constrained, a component of \textit{irreducible} background will always be present.
In the case of this analysis, the good tag-\B counts after the optimal selection in \Cref{tab:cutflow} and the fitting procedure presented here contain:
\begin{itemize}
    \item correctly tagged \BtoXsgamma events,
    \item correctly tagged-\BB events other than \BtoXsgamma.
\end{itemize}
Although in the later iterations of this analysis it may be possible to further diminish the second background component through improved selection and fitting procedures (see discussion in \Cref{sec:future_prospects}), 
some irreducible background component will always remain.

The final step before the full \BtoXsgamma spectrum extraction on simulation is evaluating the size of the remaining \BB background.
Two strategies are considered:
\begin{itemize}
    \item Count good tag-\B mesons in each event that correspond to non-\BtoXsgamma in simulation and use this value to subtract remaining \BB events from data.
    This effectively corresponds to the red histograms in \Cref{fig:mc_fit_yield_comparisons}.
    \item Perform the fit again on the simulated sample with \BtoXsgamma events removed.
    This way, good tag-\B meson counts are estimated with fitting effects included.
    This effectively corresponds to the data points in \Cref{fig:no_bxsgamma_mc_fit_yield_comparisons}.
\end{itemize}
Although both choices are valid and have certain advantages, in this analysis, the second method is chosen.
The main reason for this choice is the fact that biases or systematic effects, if such exist, could be suppressed in the fitter when subtracting.
It also requires a similar validation procedure (e.g. to be an unbiased estimator) as the total fit.
Therefore, after the \Mbc fit, the number of good-tagged \BtoXsgamma events in data will be measured as follows:
\begin{equation}\label{eq:background_subtraction}
    \mathcal{N}_{\mathrm{DATA}}^{\BtoXsgamma} = \mathcal{N}_{\mathrm{DATA}}^{\mathrm{good~tags}}  - \frac{189~\invfb}{1600~\invfb} \mathcal{N}_{\mathrm{MC}}^{\mathrm{good~tags~with~\BtoXsgamma~removed}}.
\end{equation}