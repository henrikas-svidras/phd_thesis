\Cref{sec:fitting_mbc,sec:background_subtraction} introduced the fitting procedure and \BB background subtraction.
Together with the optimal selections from \Cref{sec:final_optimisation}, this fully defines the analysis strategy from the Belle~II simulated datasets to the \BtoXsgamma spectrum.
However, the defined fitter has to be validated in simulation to give an unbiased estimation of \BtoXsgamma events, with a good resoltion and signal efficiency.
The studies in this Section will show such results.

\subsection{Validation of \texorpdfstring{\Mbc}{Mbc} fit on reduced sambple size}

The results that were shown in \Cref{fig:mc_fit_yield_comparisons} only provided results for fitting 1.6~\invab -- a dataset about an order of magnitude larger than is expected in the case of this analysis.
Therefore, the generic \MC dataset is pseudorandomly split into 10 smaller subsets, corresponding to 160~\invfb, and each of them are fitted independently.
The choice of 160~\invfb, rather than 190~\invfb which is the sample size of the Belle~II data used in the analysis, is due to anticipated data-simulation differences.
\todo[inline]{see XXX}.
Indeed, a 190~\invfb dataset should correspond to roughly 160~\invfb due to differences in tag-\B reconstruction efficiency.

