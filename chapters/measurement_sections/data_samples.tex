
\subsection{Experimental data sets}\label{sec:data}

This measurement uses data sets of \epem collisions produced by the SuperKEKB accelerator and collected by the Belle II detector in 2019-2021.
There are two data-collection modes in Belle II:
\begin{itemize}
    \item \textit{on-resonance data}: data sets collected at the collision energy $\sqrt{s}\approx=10.58~\gev$, corresponding to the mass of \FourS meson;
    \item \textit{off-resonance data}: data sets collected 60~\mev below the on-resonance threshold. 
    Such data, by definition, does not contain $\BB$ events and is an excellent testing and validation sample to understand continuum processes.
\end{itemize}
The integrated luminosity, henceforth denoted as $\int~\hspace{-7pt}~\mathcal{L}$, corresponding to the on(off)-resonance sample is 189(18)~\invfb.
The on-resonance data set contains approximately $198$ million \BB pairs.

\subsection{Simulated data sets}\label{sec:MC}

In order to prepare the analysis procedure, calculate the signal-selection efficiencies and perform a validation adhering to the principles of a blinded-analysis, 
large simulated samples are utilised.
These samples are significantly larger than the data sets that were anticipated to be analysed by this analysis to ensure that uncertainties due to limited simulation samples are minimal.
The overview of the samples is discussed in this subsection, but a quick overview is provided in \Cref{tab:simulated_samples}

\begin{table}[htbp!]
    \centering
\caption{\label{tab:simulated_samples}The overview of simulated samples used in the measurement described by this thesis.
More in-depth discussion for each sample is present in the text.}
\resizebox{0.7\textwidth}{!}{
\begin{tabular}{lcc}

Simulated sample & Size  & Generators used \\ 
\hline
Generic-\Bz & \multirow{6}{*}{1+0.6~\invab}& \multirow{2}{*}{\texttt{EvtGen} \cite{Ryd:2005zz} }\\
Generic-\Bp &                          &\\
\cline{3-3}
continuum \uubar &                     & \multirow{4}{*}{\texttt{KKMC} \cite{Ward:2002qq} interfaced to Pythia 8 \cite{Sjostrand:2007gs}}\\
continuum \ddbar &                     &\\
continuum \ccbar &                     &\\
continuum \ssbar &                     &\\
\cline{2-3}
\BptoXsgamma      & \multirow{2}{*}{100 million}  & \multirow{2}{*}{\texttt{EvtGen}, \texttt{BTOXSGAMMA} model \cite{Ryd:2005zz}}\\
\BztoXsgamma      & & \\
\cline{2-3}
\Bp\to\Kstarp(782)\g    & \multirow{2}{*}{10 million} & \multirow{2}{*}{\texttt{EvtGen}, \texttt{SVP\_HELAMP} model \cite{Ryd:2005zz}}\\
\Bz\to\Kstarz(782)\g    &  & \\
\end{tabular}
}
\end{table}

All the simulated samples correspond to the 14th official Belle II simulation production campaign, and are based on Monte-Carlo methods.
Therefore, I will henceforth refer to the simulated samples as \MC.
For the analysis I use:
\begin{itemize}
    \item four \epem\ra\qqbar ($\q\in\{\u,\d,\s,\c\}$) simulated sample categories, referred to as continuum \MC,
    \item two \FourS\ra\BB categories for charged and neutral \B modes, referred to as generic-$B$ \MC.
\end{itemize}
Altogether, the above to categories are referred to as \textit{generic} \MC.
The analysis will be performed on 1~\invab of simulation, which is more than 5 times larger than the on-resonance data set for this analysis.
Furthermore, for the background subtraction step described in XXXX, which has the strongest dependance on limited-\MC sample size, I use an even larger 1.7~\invab dataset, utilising the total available Belle~II \MC dataset.
\todo[inline]{described where?}

The generic-\B \MC includes \BtoXsgamma decays, however the number of such events is expected to be small, further diminished by the efficiency of the hadronic-tagged analysis procedure, as discussed in \Cref{sec:had_tagged_overview}.
For this reason we utilise additional samples:
\begin{itemize}
    \item 100 million \BpBm sample, where at least one $B$ is guaranteed to decay as \BptoXsgamma based on the Kagan-Neubert model \cite{Kagan:1998ym} (see \Cref{sec:btosgamma_spectrum_theory});
    \item 100 million \BzBzb sample, where at least one $B$ is guaranteed to decay as \BztoXsgamma based on the Kagan-Neubert model \cite{Kagan:1998ym} (see \Cref{sec:btosgamma_spectrum_theory});
    \item 10 million \BpBm sample, where at least one $B$ is guaranteed to decay as \Bpm\ra\Kstarpm\gamma~based on the \texttt{EvtGen SVP\_HELAMP} model \cite{Ryd:2005zz};
    \item 10 million \BzBzb sample, where at least one $B$ is guaranteed to decay as \Bz\ra\Kstarz\gamma~based on the \texttt{EvtGen HELAMP} model \cite{Ryd:2005zz}.
\end{itemize}
The first two are used for general analysis setup in sections XXXX.
The two latter will be combined in with \BtoXsgamma in a hybrid-model approach \cite{Ramirez:1989yk} discussed in XXX.
\todo[inline]{discussed where? I need to decide myself where this goes. Maybe even here, right now.}

In all cases, the detector response and readout is simulated using Geant 4 \cite{GEANT4:2002zbu}.

\epem\ra\tautau events are negligible.