This chapter serves as an overview of the analysis
and is intended to guide the reader through different analysis steps in extracting \BtoXsgamma photon-energy spectrum.
It also introduces key terminology that will be used throughout the entire text, without further clarification.
In this entire chapter, superscripts are used to denote observables in a particular frame of reference.
Given an observable $\mathcal{X}$ (laboratory frame), the same value in the frame of colliding electrons (center-of-mass) will be denoted as $\mathcal{X}^*$.
Observables in the rest-frame of the decaying $B$ meson adopt the notation $\mathcal{X}^B$.

First, the analysed data sets are introduced in XXXX.
We use a simulated data set, based on the expectations on the Standard Model to prepare the analysis procedure.
After the full analysis procedure is set up, I unblind (see \Cref{sec:blinding}) experimental data, collected by the Belle II experiment.

XXXX introduces the reconstruction of Belle II experimental and simulated \BtoXsgamma samples.
It also discusses the main processes which mimic \BtoXsgamma signal (backgrounds) and the strategies to suppress them.
In general, background for \BtoXsgamma can be divided as:
\begin{itemize}
    \item \textit{Signal-side background}: where a photon candidate is originating in a non-\BtoXsgamma decay.
    In particular, either \epem\ra\qqbar ($\q\in\{\u,\d,\s,\c\}$), henceforth -- \textit{continuum}, or from a different \B decay, e.g. where a $\piz$ is created in the decay chain that decays into 2 photons, henceforth -- \textit{generic} $B$ background.
    \item \textit{Tag-side background}: where the $B$, recoiling against the candidate \BtoXsgamma, is reconstructed incorrectly, or \epem\ra\qqbar collision products are combined to resemble a \B decay.
    I will refer to such decays as \textit{combinatorial} $B$ or continuum background, respectively.
    \item \BtoXdgamma component: which is an irreducible background at the current analysis setup. 
\end{itemize}

The efficiencies and optimisation of selections for signal and tag-side backgrounds is described in XXX.
It will rely on selections suppress the signal-side background to and a dedicated \BDT training, which aims to suppress tag-side backgrounds.

After the full background suppression procedure, a thorough setup and validation of the fitting model will be described.
The fitting procedure is aimed at removing the combinatorial tag-side backgrounds, and hence the fit of \Mbc variable is performed (\Cref{eq:mbc_exclusive}).

Lastly, an irreducible signal-side background component still remains, particularly from generic signal-side $B$ decays.
The \EB spectrum is extracted by the simulation-dependant subtraction of remaining background processes.

XXX will describe the in-depth validation procedure of the analysis technique on data samples.
The two data samples used for validation are the continuum samples and $B$ decay datasets, where selections from section XX are inverted.

The final extraction of results, unfolding is presented in XXXX.
\todo[inline]{fix all the XXXX}

