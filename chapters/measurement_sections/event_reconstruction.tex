\subsection{Tag-side $B$-meson candidate reconstruction}

The analysis begins with a reconstruction of a $B$ meson candidate in each event using the Belle II Full Event Intepretation algorithm.
\todo[inline]{Do I describe FEI here or in software section? not sure yet}
\todo[inline]{add cuts here tomorrow from note}


\subsection{Candidate photon reconstruction}

In the inclusive \BtoXsgamma analysis, the aim is to reconstruct an inclusive sample of all possible \Xs states, 
as described before (e.g. \Cref{sec:exp_overview}).
This means that explicit requirements on the momentum, number of tracks, angles etc. of the \Xs system may introduce a direct bias on the `inclusiveness' of the measurement.
Assessing this in a model-independent way is difficult.
Therefore, the \Xs system is treated in a completely `missing-momentum' approach, such that no direct requirements on it are imposed.
Furthermore, any indirect requirements, such as photon background-suppression requirements, are always checked for correlation with parameters of photon energy (which is equivalent $m_{\Xs}$ (as per \Cref{eq:mx_egamma_relation}) because it is effectively a two-body decay).
Therefore, whenever possible, only requirements on the properties of the photon that do not correlate with its energy are applied.

