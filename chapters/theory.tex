\chapter{Theoretical overview}

In this Chapter, the theoretical overview of the Standard Model, flavour physics and \BtoXsgamma will be presented.
It will introduce the main theoretical concepts that are featured in this thesis and provide a theoretical grounding for the experimental analyses performed.
\todo[inline]{section listing here once they are finished}

\section{The Standard Model}

a $SU(3)\times SU(2)\times U(1)$ quantum field theory \cite{Peskin:1995ev}.
Its theoretical framework describes the electroweak and strong force interactions between elementary particles that constitute the world as we know it.
The elementary particles included in the Standard Model can have a half-integer spin (fermions) or integer spin (bosons).
The spin-1 bosons are the mediators of the electromagnetic (photon), weak ($\Wpm$ and $\Z$) and strong interactions (gluon) in the SM.
The spin-0 Higgs boson couples to all massive particles of the Standard Model via the Higgs mechanism \cite{PhysRevLett.13.508}.
The fermions are subsplit into two additional groups, quarks and leptons, which, respectively, can and cannot interact through the strong force.

\section{Flavour physics}

Ongoing...

\section{Radiative decays \texorpdfstring{\BtoXsdgamma}{B->Xsg}}\label{eq:btosgamma_theory}

In the Standard Model, $b\ra s$ transitions are mediated by so-called flavour-changing neutral currents.
These processes can only occur through loops mediated by heavy particles.
One of the decay processes used to study these transitions are the so-called rare radiative $b\ra s\g$ and $b\ra d\g$ decays.
Due to colour confinement of the quarks, in reality they manifest as \BtoXsgamma or \BtoXdgamma, where $X_s$ and $X_d$ denote any meson state originating from the $s$ quark hadronisation.
The leading Standard Model diagrams for \BtoXsdgamma (used to collectively identify $X_s$ and $X_d$ states) processes are shown in \Cref{fig:sm_diagrams}.

\todo[inline]{citations here are needed, but I don't know how I want to proceed yet. Maybe this is common knowledge}

\begin{figure}[htbp!]
\resizebox{0.66\textwidth}{!}{
    \subcaptionbox{\label{fig:sm_diagrams}}{
        \begin{tikzpicture}
\begin{feynman}
\vertex (i1){b};
\vertex[right =1.2cm of i1] (a) ;
\vertex[right=0.85cm of a] (b);
\vertex[right=0.85cm of b] (c);
\vertex[right=1.2cm of c] (o1) {s,d};
\vertex[below=2em of c] (g1);
\vertex[right=2em of g1] (o2) {$\gamma$};

\diagram* {
(i1) -- [fermion] (a) -- [boson, edge label =\(W^{\pm}\)] (c) -- [fermion] (o1),
(a) -- [fermion,half left, edge label = \(uct\)] (c),
(b) -- [photon] (o2),
};
\end{feynman}
\end{tikzpicture}

        \begin{tikzpicture}
\begin{feynman}
\vertex (i1){b};
\vertex[right =1.2cm of i1] (a);
\vertex[right=0.85cm of a] (b);
\vertex[right=0.85cm of b] (c);
\vertex[right=1.2cm of c] (o1) {s,d};
\vertex[below=2em of c] (g1);
\vertex[right=2em of g1] (o2) {$\gamma$};

\diagram* {
(i1) -- [fermion] (a) -- [fermion, edge label =\(uct\)] (c) -- [fermion] (o1),
(a) -- [boson,half left, edge label = \(W^{\pm}\)] (c),
(b) -- [photon] (o2),
};
\end{feynman}
\end{tikzpicture}



    }
}
\resizebox{0.33\textwidth}{!}{
\subcaptionbox{\label{fig:bsm_diagrams}}{
    \begin{tikzpicture}
\begin{feynman}
\vertex (i1){b};
\vertex[right =1.2cm of i1] (a) ;
\vertex[right=0.85cm of a] (b);
\vertex[right=0.85cm of b] (c);
\vertex[right=1.2cm of c] (o1) {s,d};
\vertex[below=2em of c] (g1);
\vertex[right=2em of g1] (o2) {$\gamma$};

\diagram* {
(i1) -- [fermion] (a) -- [scalar, edge label =\(H^{\pm}\)] (c) -- [fermion] (o1),
(a) -- [fermion,half left, edge label = \(uct\)] (c),
(b) -- [photon] (o2),
};
\end{feynman}
\end{tikzpicture}

}
}
\caption{\label{fig:b_to_s_gamma_diagrams}
The Feynman diagrams for radiative $b\ra s$ transitions. 
\Cref{fig:sm_diagrams} shows the leading-order Standard Model diagrams, where the $b\ra s$ transition occurs via electroweak loops.
A beyond-Standard Model scenario where this transition is mediated by a charged-Higgs boson particle is shown in \Cref{fig:bsm_diagrams}.
}
\end{figure}

Since \BtoXsdgamma decays proceed via $b\ra s\g$ transitions, they are interesting probes for beyond-Standard-Model (\BSM) particles. 
The leading contributions to these transitions only happen via one-loop diagrams, as seen in \Cref{fig:b_to_s_gamma_diagrams}.
The electroweak loop can get contributions from all the quarks, but is dominated by the much heavier top quark \cite{Mannel:2001vn} as followed by the Glashow-Iliopoulos-Maiani mechanism~\cite{Glashow:1970gm}. 
Furthermore, the masses of possible \BSM weakly-interacting particles that can appear in the loops may be as high as \order{(100~\tev)} \cite{Misiak:2020vlo} which makes studying these decays incredibly appealing.

The decay rate of \BtoXsdgamma is affected by the contributions from weak and strong interactions.
Weak contributions can be calculated perturbatively, however the strong contributions are usually described by effective point-like theories.
In weak decays, including \BtoXsgamma, \textit{Operator Product Expansion} \cite{Peskin:1995ev,Buras:1998raa} is used to derive the effective theory.
It replaces the hadron-binding strong quark interactions and the weak quark interactions mediating $\b\ra\s$ by a single effective vertex.
\todo[inline]{consider adding here sketches}
An effective Lagrangian describing the $\b\to\s\g$ can be written as~\cite{Kaminski:2012eb,Misiak:2015xwa}:
\begin{equation}\label{eq:effective_lagrangian}
    \mathcal{L}_{\mathrm{eff}} = \frac{4G_F}{\sqrt{2}}V_{ts}^*V_{tb}\left[\sum^{8}_{i=1}\mathcal{C}_i(\mu)\mathcal{O}_i(\mu)
                                                + \frac{V^*_{uq}V_{ub}}{V^*_{tq}V_{tb}}\sum^{2}_{i=1}\mathcal{C}_i(\mu)(\mathcal{O}_i(\mu)-\mathcal{O}_i^u(\mu))\right],
\end{equation}
where $G_F$ is the Fermi constant.
The factors $\mathcal{C}_i$, known as \textit{Wilson coefficients}, encode the weak interactions and can be calculated perturbatively.
The other set of operators, $\mathcal{O}_i$ contain the contributions from the strong interactions and are calculated using non-perturbative methods.
The renormalisation scale, $\mu$, is chosen such that strong and weak contributions are separated.
Conventionally, it is set to be of order of the $b$ quark mass $\mu\sim m_b$.

The exact expressions of operators $\mathcal{O}_i$ are given in \Cref{sec:appendix_local_opperators}. 
Coefficients $\mathcal{C}_{3-6}$ have been calculated and shown to be small, therefore the most important contributions arise from $\mathcal{O}_{1,2,7,8}$ \cite{Chetyrkin:1996vx,Misiak:2020vlo}.
Furthermore, the ratio $V^*_{uq}V_{ub}/V^*_{tq}V_{tb}$ is small for the case of $q=s$ \cite{Charles:2015gya}, and the terms including $\mathcal{O}^u_i$ are irrelevant at leading-order calculation of the total decay rate \cite{Misiak:2015xwa}.
Interestingly, the latter point does not hold for $\b\to\d$ case, where additional effects need to be considered due to the contributions from these terms.
In particular, additional non-perturbative uncertainties need to be considered for the Standard Model calculation of the decay rate \cite{Misiak:2015xwa}.
However, for the rest of \Cref{eq:btosgamma_theory}, $\b\to\s\g$ decays will be implied unless explicitly stated, as this is the main analysis of the thesis.

The decay rate of the inclusive \BtoXsgamma is modelled as the rate of its parton decay, and expressed as a sum of a perturbative and non-perturbative contribution as \cite{Misiak:2015xwa}:
\begin{equation}\label{eq:btosgammarate_nonp}
    \Gamma(\bar{B}\ra\X_s\g) = \Gamma(b\ra s\g) + \delta\Gamma_{\mathrm{non-p.}},
\end{equation}
The $\Gamma(b\ra s\g)$ is the perturbatively calculable rate of $b$ quarks decaying into charmless partons, and $\delta\Gamma_{non-p.}$ is the non-perturbative contribution arising outside of local operator product expansion.
The non-perturbative effects arise from the so-called resolved photon contributions.
These contributions are a result of the photon coupling directly to partons instead of the effective weak interaction vertex.
As long as an appropriate photon energy threshold (in the decaying $B$ meson frame), \EB, is chosen, the non-perturbative effects in \Cref{eq:btosgammarate_nonp} can be considered much smaller.
Until recently, a 5\% uncertainty was attached to this assumption \cite{Benzke:2010js}. 
However, recent developments in understanding the non-perturbative effects \cite{Gunawardana:2019gep} led to an improved treatment of the associated uncertainties \cite{Misiak:2020vlo}.
As will be seen later, the threshold for \EB is motivated also from the experimental side, as large background-process contamination is present in the low-\EB region.

Using operator product expansion, one can calculate $|<\s\g|\mathcal{L}_{\mathrm{eff}}|b>|^2$ and then the perturbative decay rate can be written as \cite{Misiak:2020vlo}:
\begin{equation}\label{eq:theory_decay_rate}
    \Gamma(b\ra\s\g) = \frac{G^2_F m_{\mathrm{b}}^5 \alpha_{\mathrm{em}}}{32\pi^4}|V_{ts}^*V_{tb}|^2\sum_{i,j}^8C^{\mathrm{eff}}_{i}(\mu_b)C^{\mathrm{eff}}_{j}(\mu_b)\hat{G}_{ij}(E_0,\mu_b),
\end{equation}
where $m_{\mathrm{b}}$ is the $b$-quark pole mass, $\alpha_{\mathrm{em}}$ is the fine-structure constant.
The $C_{i}^{\mathrm{eff}}(\mu_b)$ coefficients represent effective Wilson coefficients \cite{Buras:1993xp} that are renormalisation scheme independent and defined through linear combinations of the Wilson coefficients of \Cref{eq:effective_lagrangian}.
Finally, the functions $\hat{G}_{ij}$ encapsulate the terms describing the interference between operators $\mathcal{O}_{i,j}$.
As already mentioned before, the dominant terms arise from a handful of local operators and the combined efforts to calculate this have brought \BtoXsgamma theoretical estimates to the next-to-next-to-leading order precision.
The dominant functions $\hat{G}_{77}$ \cite{Asatrian:2006rq}, $\hat{G}_{78}$ \cite{Asatrian:2010rq}, $\hat{G}_{(1,2)7}$ \cite{Boughezal:2007ny,Misiak:2020vlo} have been evaluated.
Contributions from $\hat{G_{(1,2,8)8}}$ \cite{Ferroglia:2010xe,Misiak:2010tk} have also been calculated.
In fact, as the calculations have already reached next-to-next-to-leading order precision, even contributions involving $\mathcal{O}_{i}^u$ are accounted \cite{Huber:2014nna}.
The mixing of $\mathcal{O}_{1-6}\to\mathcal{O}_8$ has been also described \cite{Czakon:2006ss}. 

In order to minimise the uncertainties arising from the CKM matrix element determination and the factor of $m_b^5$ in \Cref{eq:theory_decay_rate}, the calculation of \BtoXsgamma decay rate is usually normalised to the semi-leptonic decay rate of $\B\to\X_c\ell\nu_{\ell}$.
In doing so, the decay rate is expressed as \cite{Misiak:2020vlo}:
\begin{equation}\label{eq:normalised_br}
    \frac{\Gamma(\BtoXsgamma)_{\Egamma>E_0}}{\Gamma(\B\to X_c\ell\nu_\ell)} = \frac{\mathcal{B}(\BtoXsgamma)}{\mathcal{B}(\B\to X_c\ell\nu_\ell)} = \left|\frac{V_{ts}^*V_{tb}}{V_{cb}}\right|^2\frac{6\alpha_{\mathrm{em}}}{\pi C}[P(E_0)+N(E0)].
\end{equation}
Here, $P(E_0)$ denotes the perturbatively calculable contribution, and $N(E_0)$ the non-perturbative correction, which was already briefly discussed.
The additional semileptonic phase-space factor:
\begin{equation}
    C = \left|\frac{V_{ub}}{V_{cb}}\right|^2 \frac{\Gamma(\B\to X_ce\bar{\nu})}{\Gamma(\B\to X_ue\bar{\nu})},
\end{equation}  
accounts for the fact that the charmless semileptonic decay rate is used in $P(E_0)$ determination:
\begin{equation}
    \frac{\Gamma(\b\to\s\g)_{\Egamma>E_0}}{|V_{cb}/V_{ub}|^2 \Gamma(\b\to\u e\nub)} = \left|\frac{V_{ts}^*V_{tb}}{V_{cb}}\right|^2 \frac{6\alpha_{\mathrm{em}}}{\pi}P(E_0).
\end{equation}
The factor $C$ is determined using the experimental value of $\B\to X_c \ell\bar{\nu}$ \cite{Alberti:2014yda}.
The decay channel $\B\to X_c \ell\bar{\nu}$ has a large branching fraction and is known precisely experimentally \cite{Workman:2022ynf}, therefore this requirement does not pose a problem.

At leading order, where non-perturbative effects are disregarded, effects from $\mathcal{C}_{2,7,8}$ are most important and this can be expressed in a compact form \cite{Buras:1993xp}:
\begin{equation}
    \frac{\mathcal{B}(\BtoXsgamma)}{\mathcal{B}(\B\to X_c\ell\nu_\ell)} = \left|\frac{V_{ts}^*V_{tb}}{V_{cb}}\right|^2\frac{6\alpha_{\mathrm{em}}}{\pi C}|C_{7}^{(0)\mathrm{eff}}(\mu)|^2.
\end{equation}
Here $C_{7}^{(0)\mathrm{eff}}(\mu)$ is the effective Wilson coefficient which was already encountered before, defined here as:
\begin{equation}
    C_{7}^{(0)\mathrm{eff}}(\mu) = \eta^{16/23}\mathcal{C}^{(0)}_7(M_W) + \frac{8}{3} \left(\eta^{14/23}-\eta^{16/23}\right)\mathcal{C}^{(0)}_8(M_W) + \mathcal{C}^{(0)}_2(M_W) \sum_i^8h_i\eta^{a_i}
\end{equation}

Here $\eta=\alpha_s(M_W)/\alpha_s(\mu)$, whereas $h_i$ and $a_i$ are the eigenvalues of the scheme-independent matrix $\gamma^{(0)\mathrm{eff}}$, which can be found defined in the Appendix A of Ref. \cite{Buras:1993xp}.
The matrix $\gamma^{(0)\mathrm{eff}}$ governs the leading order QCD corrections to $\b\ra\s\g$ and its elements appear in the renormalisation group equation of effective Wilson coefficients.

\todo[inline]{I could change $M_W$ to $\mu_0$ because it doesnt matter what scale is there I think idk}



At lowest order, $b\ra\s(\d) \g$ is a two-body decay, which means that the decay rate peaks very close to the kinematic limit equal to half the \B-meson mass, $\lsim E_{\g}$.
