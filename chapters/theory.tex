\chapter{Theoretical overview}

In this Chapter, the theoretical overview of the Standard Model, flavour physics and \BtoXsgamma will be presented.
It will introduce the main theoretical concepts that are featured in this thesis and provide a theoretical grounding for the experimental analyses performed.
\todo[inline]{section listing here once they are finished}

\section{The Standard Model}

a $SU(3)\times SU(2)\times U(1)$ quantum field theory \cite{Peskin:1995ev}.
Its theoretical framework describes the electroweak and strong force interactions between elementary particles that constitute the world as we know it.
The elementary particles included in the Standard Model can have a half-integer spin (fermions) or integer spin (bosons).
The spin-1 bosons are the mediators of the electromagnetic (photon), weak ($\Wpm$ and $\Z$) and strong interactions (gluon) in the SM.
The spin-0 Higgs boson couples to all massive particles of the Standard Model via the Higgs mechanism \cite{PhysRevLett.13.508}.
The fermions are subsplit into two additional groups, quarks and leptons, which, respectively, can and cannot interact through the strong force.

\section{Flavour physics}

Ongoing...

\section{The decay rate of \texorpdfstring{\BtoXsdgamma}{B->Xsg}}\label{eq:btosgamma_theory}

In the Standard Model, $b\ra s$ transitions are mediated by so-called flavour-changing neutral currents.
These processes can only occur through loops mediated by heavy particles.
One of the decay processes used to study these transitions are the so-called rare radiative $b\ra s\g$ and $b\ra d\g$ decays.
Due to colour confinement of the quarks, in reality they manifest as \BtoXsgamma or \BtoXdgamma, where $X_s$ and $X_d$ denote any meson state originating from the $s$ quark hadronisation.
The leading Standard Model diagrams for \BtoXsdgamma (used to collectively identify $X_s$ and $X_d$ states) processes are shown in \Cref{fig:sm_diagrams}.

\todo[inline]{citations here are needed, but I don't know how I want to proceed yet. Maybe this is common knowledge}

\begin{figure}[htbp!]
\resizebox{0.66\textwidth}{!}{
    \subcaptionbox{\label{fig:sm_diagrams}}{
        \input{tikz_diagrams/b_to_s_gamma_W.tex}
        \input{tikz_diagrams/b_to_s_gamma_Q.tex}
    }
}
\resizebox{0.33\textwidth}{!}{
\subcaptionbox{\label{fig:bsm_diagrams}}{
    \input{tikz_diagrams/b_to_s_gamma_H.tex}
}
}
\caption{\label{fig:b_to_s_gamma_diagrams}
The Feynman diagrams for radiative $b\ra s$ transitions. 
\Cref{fig:sm_diagrams} shows the leading-order Standard Model diagrams, where the $b\ra s$ transition occurs via electroweak loops.
A beyond-Standard Model scenario where this transition is mediated by a charged-Higgs boson particle is shown in \Cref{fig:bsm_diagrams}.
}
\end{figure}

Since \BtoXsdgamma decays proceed via $b\ra s\g$ transitions, they are interesting probes for beyond-Standard-Model (\BSM) particles. 
The leading contributions to these transitions only happen via one-loop diagrams, as seen in \Cref{fig:b_to_s_gamma_diagrams}.
The electroweak loop can get contributions from all the quarks, but is dominated by the much heavier top quark \cite{Mannel:2001vn} as followed by the Glashow-Iliopoulos-Maiani mechanism~\cite{Glashow:1970gm}. 
Furthermore, the masses of possible \BSM weakly-interacting particles that can appear in the loops may be as high as \order{(100~\tev)} \cite{Misiak:2020vlo} which makes studying these decays incredibly appealing.

The decay rate of \BtoXsdgamma is affected by the contributions from weak and strong interactions.
Weak contributions can be calculated perturbatively, however the strong contributions are usually described by effective point-like theories.
In weak decays, including \BtoXsgamma, \textit{Operator Product Expansion} \cite{Peskin:1995ev,Buras:1998raa} is used to derive the effective theory.
It replaces the hadron-binding strong quark interactions and the weak quark interactions mediating $\b\ra\s$ by a single effective vertex.
\todo[inline]{consider adding here sketches}
An effective Lagrangian describing the $\b\to\s\g$ can be written as~\cite{Kaminski:2012eb,Misiak:2015xwa}:
\begin{equation}\label{eq:effective_lagrangian}
    \mathcal{L}_{\mathrm{eff}} = \frac{4G_F}{\sqrt{2}}V_{ts}^*V_{tb}\left[\sum^{8}_{i=1}\mathcal{C}_i(\mu)\mathcal{O}_i(\mu)
                                                + \frac{V^*_{uq}V_{ub}}{V^*_{tq}V_{tb}}\sum^{2}_{i=1}\mathcal{C}_i(\mu)(\mathcal{O}_i(\mu)-\mathcal{O}_i^u(\mu))\right],
\end{equation}
where $G_F$ is the Fermi constant.
The factors $\mathcal{C}_i$, known as \textit{Wilson coefficients}, encode the weak interactions and can be calculated perturbatively.
The other set of operators, $\mathcal{O}_i$ contain the contributions from the strong interactions and are calculated using non-perturbative methods.
The renormalisation scale, $\mu$, is chosen such that strong and weak contributions are separated.
Conventionally, it is set to be of order of the $b$ quark mass $\mu\sim m_b$.

The exact expressions of operators $\mathcal{O}_i$ are given in \Cref{sec:appendix_local_opperators}. 
Coefficients $\mathcal{C}_{3-6}$ have been calculated and shown to be small, therefore the most important contributions arise from $\mathcal{O}_{1,2,7,8}$ \cite{Chetyrkin:1996vx,Misiak:2020vlo}.
Furthermore, the ratio $V^*_{uq}V_{ub}/V^*_{tq}V_{tb}$ is small for the case of $q=s$ \cite{Charles:2015gya}, and the terms including $\mathcal{O}^u_i$ are irrelevant at leading-order calculation of the total decay rate \cite{Misiak:2015xwa}.
Interestingly, the latter point does not hold for $\b\to\d$ case, where additional effects need to be considered due to the contributions from these terms.
In particular, additional non-perturbative uncertainties need to be considered for the Standard Model calculation of the decay rate \cite{Misiak:2015xwa}.
However, for the rest of \Cref{eq:btosgamma_theory}, $\b\to\s\g$ decays will be implied unless explicitly stated, as this is the main analysis of the thesis.

The decay rate of the inclusive \BtoXsgamma is modelled as the rate of its parton decay, and expressed as a sum of a perturbative and non-perturbative contribution as \cite{Misiak:2015xwa}:
\begin{equation}\label{eq:btosgammarate_nonp}
    \Gamma(\bar{B}\ra\X_s\g) = \Gamma(b\ra s\g) + \delta\Gamma_{\mathrm{non-p.}},
\end{equation}
The $\Gamma(b\ra s\g)$ is the perturbatively calculable rate of $b$ quarks decaying into charmless partons, and $\delta\Gamma_{non-p.}$ is the non-perturbative contribution arising outside of local operator product expansion.
The non-perturbative effects arise from the so-called resolved photon contributions.
These contributions are a result of the photon coupling directly to partons instead of the effective weak interaction vertex.
As long as an appropriate photon energy threshold (in the decaying $B$ meson frame), \EB, is chosen, the non-perturbative effects in \Cref{eq:btosgammarate_nonp} can be considered much smaller.
Until recently, a 5\% uncertainty was attached to this assumption \cite{Benzke:2010js}. 
However, recent developments in understanding the non-perturbative effects \cite{Gunawardana:2019gep} led to an improved treatment of the associated uncertainties \cite{Misiak:2020vlo}.
As will be seen later, the threshold for \EB is motivated also from the experimental side, as large background-process contamination is present in the low-\EB region.

Using operator product expansion, one can calculate $|<\s\g|\mathcal{L}_{\mathrm{eff}}|b>|^2$ and then the perturbative decay rate can be written as \cite{Misiak:2020vlo}:
\begin{equation}\label{eq:theory_decay_rate}
    \Gamma(b\ra\s\g) = \frac{G^2_F m_{\mathrm{b}}^5 \alpha_{\mathrm{em}}}{32\pi^4}|V_{ts}^*V_{tb}|^2\sum_{i,j}^8C^{\mathrm{eff}}_{i}(\mu_b)C^{\mathrm{eff}}_{j}(\mu_b)\hat{G}_{ij}(E_0,\mu_b),
\end{equation}
where $m_{\mathrm{b}}$ is the $b$-quark pole mass, $\alpha_{\mathrm{em}}$ is the fine-structure constant.
The $C_{i}^{\mathrm{eff}}(\mu_b)$ coefficients represent effective Wilson coefficients \cite{Buras:1993xp} that are renormalisation-scheme independent and defined through linear combinations of the Wilson coefficients of \Cref{eq:effective_lagrangian}.
Finally, the functions $\hat{G}_{ij}$ encapsulate the terms describing the interference between operators $\mathcal{O}_{i,j}$.
As already mentioned before, the dominant terms arise from a handful of local operators and the combined efforts to calculate this have brought \BtoXsgamma theoretical estimates to the next-to-next-to-leading order precision.
The dominant functions $\hat{G}_{77}$ \cite{Asatrian:2006rq}, $\hat{G}_{78}$ \cite{Asatrian:2010rq}, $\hat{G}_{(1,2)7}$ \cite{Boughezal:2007ny,Misiak:2020vlo} have been evaluated.
Contributions from $\hat{G}_{(1,2,8)8}$ \cite{Ferroglia:2010xe,Misiak:2010tk} have also been calculated.
In fact, as the calculations have already reached next-to-next-to-leading order precision, even contributions involving $\mathcal{O}_{i}^u$ are accounted \cite{Huber:2014nna}.
The mixing of $\mathcal{O}_{1-6}\to\mathcal{O}_8$ has been also described \cite{Czakon:2006ss}. 
Some of the next-to-next-to-leading order corrections depend on the mass of the charm quark.
However, as the evaluation of such corrections at the physical mass of the charm quark is complicated, the mass is interpolated between $m_c=0$ and $m_c\gg m_b$ \cite{Misiak:2019ccp}.

In order to minimise the uncertainties arising from the CKM matrix element determination and 5th power dependance on the ill-defined pole-mass of the $b$ quark in \Cref{eq:theory_decay_rate}, the calculation of \BtoXsgamma decay rate is usually normalised to the semi-leptonic decay rate of $\B\to\X_c\ell\nu_{\ell}$.
In doing so, the decay rate is expressed as \cite{Misiak:2020vlo}:
\begin{equation}\label{eq:normalised_br}
    \frac{\Gamma(\BtoXsgamma)_{\Egamma>E_0}}{\Gamma(\B\to X_c\ell\nu_\ell)} = \frac{\mathcal{B}(\BtoXsgamma)}{\mathcal{B}(\B\to X_c\ell\nu_\ell)} = \left|\frac{V_{ts}^*V_{tb}}{V_{cb}}\right|^2\frac{6\alpha_{\mathrm{em}}}{\pi C}[P(E_0)+N(E_0)].
\end{equation}
Here, $P(E_0)$ denotes the perturbatively calculable contribution, and $N(E_0)$ the non-perturbative correction, both of which have been briefly discussed earlier.
The additional semileptonic phase-space factor:
\begin{equation}
    C = \left|\frac{V_{ub}}{V_{cb}}\right|^2 \frac{\Gamma(\B\to X_ce\bar{\nu})}{\Gamma(\B\to X_ue\bar{\nu})},
\end{equation}  
accounts for the fact that the charmless semileptonic decay rate is used in $P(E_0)$ determination:
\begin{equation}
    \frac{\Gamma(\b\to\s\g)_{\Egamma>E_0}}{|V_{cb}/V_{ub}|^2 \Gamma(\b\to\u e\nub)} = \left|\frac{V_{ts}^*V_{tb}}{V_{cb}}\right|^2 \frac{6\alpha_{\mathrm{em}}}{\pi}P(E_0).
\end{equation}
The factor $C$ is determined using the experimental value of $\B\to X_c \ell\bar{\nu}$ \cite{Alberti:2014yda}.
The decay channel $\B\to X_c \ell\bar{\nu}$ has a large branching fraction and is known precisely experimentally \cite{Workman:2022ynf}, therefore this requirement does not pose a problem.

At leading order, where non-perturbative effects are disregarded, effects from $\mathcal{C}_{2,7,8}$ are most important and this can be expressed in a compact form \cite{Buras:1993xp}:
\begin{equation}
    \frac{\mathcal{B}(\BtoXsgamma)}{\mathcal{B}(\B\to X_c\ell\nu_\ell)} = \left|\frac{V_{ts}^*V_{tb}}{V_{cb}}\right|^2\frac{6\alpha_{\mathrm{em}}}{\pi C}|C_{7}^{(0)\mathrm{eff}}(\mu)|^2.
\end{equation}
Here $C_{7}^{(0)\mathrm{eff}}(\mu)$ is the effective leading-order Wilson coefficient which was briefly introduced before.
At leading-order calculation it takes the form:
\begin{equation}\label{eq:effective_c7}
    C_{7}^{(0)\mathrm{eff}}(\mu) = \eta^{16/23}\mathcal{C}^{(0)}_7(\mu) + \frac{8}{3} \left(\eta^{14/23}-\eta^{16/23}\right)\mathcal{C}^{(0)}_8(\mu) + \mathcal{C}^{(0)}_2(\mu \sum_i^8h_i\eta^{a_i}.
\end{equation}
The Standard Model coefficients $C_{2,7,8}^{(0)}$ are expressed in terms of $x=(m_t/M_W)^2$:
\begin{align}
    \begin{split}
    C_2^{(0)}(M_W) &= 1\\
    C_7^{(0)}(M_W) &= \frac{3x^3-2x^2}{4(x-1)^4}\ln x + \frac{-8x^3-5x^2+7x}{24(x-1)^3}\\
    C_8^{(0)}(M_W) &= \frac{-3x^2}{4(x-1)^4}\ln x + \frac{-x^3+5x^2+2x}{8(x-1)^3}\\
    \end{split}
\end{align}

The coefficient $\eta$ is expressed as $\eta=\alpha_s(M_W)/\alpha_s(\mu)$, whereas $h_i$ and $a_i$ are the eigenvalues of the scheme-independent anomalous dimension matrix $\gamma^{(0)\mathrm{eff}}$, which can be found defined in the Appendix A of Ref. \cite{Buras:1993xp}.
The matrix $\gamma^{(0)\mathrm{eff}}$ governs the leading order QCD corrections to $\b\ra\s\g$ and its elements appear in the renormalisation group equation of the effective Wilson coefficients.

Here it is interesting to look at the \Cref{eq:effective_c7} in more detail. 
In the absence of QCD effects one has $\eta=1$, therefore the \BtoXsgamma decay would be governed solely by the photonic dipole exchange.
However, in the Standard Model, the additional terms lead to a QCD-enhancment of the decay.
To illustrate, when \Cref{eq:effective_c7} is evaluated explicitly:

\todo[inline]{I have to correct these coefficients, probably best according to page 186 of \cite{Buras:1998raa}}

\begin{align}
    \begin{split}
        C_7^{(0)\mathrm{eff}}(M_W) &= 0.689 \cdot C_7^{(0)}(M_W) + 0.087 \cdot C_8^{(0)}(M_W) - 0.161 \cdot C_2^{(0)}(M_W)  \\ 
                          &= 0.689 \cdot (-0.161) + 0.087 \cdot (-0.086) - 0.161\\
                          &= -0.280, \\
    \end{split}
\end{align}
it is clear that the additive QCD two-loop contributions enhance the rate significantly.
Also notable is the fact that the multiplicative QCD correction in the first term suppresses the decay rate, but at a smaller scale than the additive corrections.


\todo[inline]{I could change $M_W$ to $\mu_0$ because it doesnt matter what scale is there I think idk}

To go beyond the leading-order approximation, one has to take into account the full form of \Cref{eq:normalised_br}.
Marking the top contributions as $K_t$, as well as other contributions (dominated by charm) $K_c$, and the electroweak corections $K_{\mathrm{ew}}$ one arrives at \cite{Gambino:2001ew}:
\begin{equation}\label{eq:perturbative_entry}
    P(E_0) = \left| K_c + \left(1+\frac{\alpha_s(M_W)}{\pi} \ln\frac{M_W^2}{m_t^2}r(\mu_0)K_t+K_{\mathrm{ew}}\right)\right| +B(E_0).
\end{equation}
The additional term $B(E_0)$ is the bremmstrahlung function encodes the effects of $b\ra s\g g$ and $b\ra s\g \qqbar$ transitions.
It is the only part of \Cref{eq:perturbative_entry} that depends on the chosen \Egamma threshold.
The ratio $r(\mu_0)$ is introduced to account for differences in the choice of the renormalisation scale for $m_b$-related kinematic parameters.

Finally, estimations of the non-perturbative corrections are always done separately.
\todo[inline]{add here about N(E0) if/when Ivan replies}

A detailed look of the higher-order calculation of the total \BtoXsgamma decay rate is beyond the scope of this thesis, but such calculations can be followed up in the references already introduced in this chapter, e.g. \cite{Misiak:2020vlo} and references therein.

The above mentioned reference also includes the most up-to-date estimation of the total decay rate of \BtoXsgamma in the Standard Model:
\begin{equation}
    \mathcal{B}(\BtoXsgamma) = (3.40\pm 0.17) \times 10^{-4}.
\end{equation}
The uncertainties going into the value arise from unevaluated higher-order effects, the necessity to perform an interpolation in $m_c$, and a parametric uncertainty that also encodes the non-perturbative effects.
The first two amount to $3\%$ each, whereas the last is considered at $2.5\%$. 
This amounts to a total uncertainty of, approximately, 4.9\%.
For significant accuracy improvements in the future, higher-order calculations will not be sufficient alone.
It is necessary to remove the dependance on $m_c$ interpolation and improve the treatment of the parametric uncertainties (non-perturbative effects) to go below the $\sqrt{3^2+2.5^2}\approx3.9\%$ uncertainty.

While this value is sufficient to compare the current experimental and theoretical predictions (see Chapter XX), the input from Belle II will reduce the experimental uncertainty below that of the theoretical prediction.
\todo[inline]{experimental status add here or maybe just snowmass citation}

\section{The photon energy spectrum of \texorpdfstring{\BtoXsdgamma}{B->Xsg}}\label{eq:btosgamma_spectrum_theory}

It was already briefly discussed when introducing \Cref{eq:btosgammarate_nonp} that theoretical and experimental evaluations of the \BtoXsgamma decay rate employ a photon-energy threshold.
At lowest order, $b\ra\s(\d) \g$ is a two-body decay, which means that the decay rate peaks near to the kinematic limit equal to half the \B-meson mass, $m_B/2\gsim E_{\g}$.
Higher-order effects, such as gluonstrahlung, and the Fermi motion of the $b$ quark within the $B$ meson smear the distribution.
The necessity to account for all of these effects occuring at different energy-scales motivates the use of soft-collinear effective-theory to describe \BtoXsgamma photon-energy spectrum \cite{Neubert:2004qw,Ligeti:2008ac}.
It allows to factorise different contributions to the decay rate, $d\Gamma$, into terms originating from effective hard-interaction vertices, collinear particles and soft particle:
\begin{equation}\label{eq:differential_decay_rate_SCET}
    d\Gamma \propto \mathcal{H} \times \mathcal{J} \otimes \mathcal{S}.
\end{equation}
The $\mathcal{H}$, $\mathcal{J}$, $\mathcal{S}$ represent the hard, jet and hadronic-soft functions, respectively, with $\otimes$ symbolising convolution between the two terms.
Interestingly, the $\mathcal{S}$ can be further factorised into a partonic soft-function, $\mathcal{S}_{\mathrm{partonic}}$, and a non-perturbative \textit{shape-function}, $\mathcal{F}$:

\begin{equation}\label{eq:factorisation}
    \mathcal{S} = \mathcal{S}_{\mathrm{partonic}} \otimes \mathcal{F},
\end{equation}
which means that the application of SCET is fully capable to separate perturbative and non-perturbative contributions in the differential decay-rate of \BtoXsgamma.

Due to the larger \BtoXsgamma decay rate and overall smaller background-process rate, experimental values have the highest precision around the peak region.
However, from the theory side, the peak part is mostly governed by the non-perturbative effects, encoded within the shape function \cite{Ligeti:2008ac}.
The shape function encodes the $b$-quark residual-momentum distribution within the $B$-meson. 
Therefore, reliable theoretical description of the shape-function is critical to make sensible experimental and theoretical comparisons of the photon-energy spectrum. 

An additional motivation to study the shape function is the fact that such function is a universal property of the $B$ meson at leading order in $1/m_b$ \cite{Neubert:1993um,Bigi:1993ex}.
This allows to extract the functional form by a precise experimental determination of the \BtoXsgamma spectrum and use it to improve the precision of other measurements.
For example, in the measurement of the $|V_{ub}|$ using $B\rightarrow X_u \ell \bar{\nu_{\ell}}$ decays suffers from orders-of-magnitude larger backgrounds from $B\rightarrow X_c \ell \bar{\nu_{\ell}}$ in most phase-space regions.
The regions where this background is kinematically-forbidden, the theoretical predictions are dependant on these non-perturbative shape functions.
The extracted inputs from \BtoXsgamma could then be used to predict the $B\rightarrow X_u \ell \bar{\nu_{\ell}}$ spectra.
This is an important relation which could lead to a model-independent evaluation of the $|V_{ub}|$ element \cite{Neubert:1993um}.

Let's consider \Cref{eq:differential_decay_rate_SCET,eq:factorisation} in a bit more detail.
Following the treatment of Ref.~\cite{Ligeti:2008ac}, the differential decay rate takes the form:

\begin{equation}\label{eq:differential_decay_simba}
    \frac{d\Gamma}{dE_{\gamma}} = 2\frac{G_F^2\alpha_{\mathrm{em}} m_b^5}{32\pi^4}|V_{ts}V_{tb}^*|^2 \mathcal{H}(x, \mu) \times \int dk \mathcal{P}(m_b, x - k, \mu_i) \mathcal{F}(k) + \mathrm{corrections},
\end{equation}
where $x=m_B-2\Egamma$, $\mathcal{P}$ is used to denote the perturbatively calculable $\mathcal{J}\times\mathcal{S_{\mathrm{partonic}}}$ and the symbolic convolution has now been replaced by an integration over a dummy-momentum $k$.
The higher order corrections introduce additional shape-functions, but are suppressed by a factor of $1/m_b$ \cite{Ligeti:2008ac,Bernlochner:2020jlt}.
Generally, $\mathcal{H}$ and $\mathcal{P}$ is calculable in perturbation theory (see Appendix A of \cite{Ligeti:2008ac}).
$\mathcal{H}$ is directly expressed in terms of the effecive Wilson coefficient $\mathcal{C}_7^{\mathrm{eff}}$. 
Moreover, at lowest-order in perturbation theory $\mathcal{P}$ can be expressed as a delta function:

\begin{equation}
    \mathcal{P}(m_B,k,\mu) = \delta(k) + \mathcal{O}(\alpha_s),
\end{equation}
and therefore integrating out the momentum $k$:
\begin{equation}\label{eq:integrated_egamma}
    \frac{d\Gamma}{d\Egamma} \propto |C_7^{\mathrm{eff}}|^2 \mathcal{F}(x).
\end{equation}
It is clear that the peak region (close to $m_B/2$) is governed strongly by the shape-function.

Conversely to the Wilson coefficients, the shape function is not well-known and non-perturbative.
Therefore, theoretical and experimental comparisons always lead to modelling-related uncertainties.
There are a number of different works which propose ways to describe it \cite{Benson:2004sg,Lange:2005yw,Andersen:2005mj,Gambino:2007rp,Aglietti:2007ik,Bernlochner:2020jlt}.
For example, Ref. \cite{Andersen:2005mj} describe it based on a technique called dressed gluon exponentiation.
The result of Ref. \cite{Lange:2005yw} provides several functional forms of the shape function, based on the leading-order shape-function that can be extracted from \BtoXsgamma decays.
An interesting analysis by the SIMBA collaboration \cite{Bernlochner:2020jlt} describes a model-independent treatment of the shape function, embodying the idea sketched out here.
In particular, any chosen shape function is expanded in a complete set of orthonormal basis functions, and can therefore be extracted, together with the normalisation, from a global-fit of available experimental results.
The strength of this approach is a consistent method to combine several measurements for the extraction of the shape function and its moments.

One of the most commonly chosen inclusive \BtoXsgamma models for experimental analyses is known as the Kagan-Neubert model \cite{Kagan:1998ym}.
The idea of the model is to model the shape function using a simple form:
\begin{equation}
    \mathcal{F}(x) = N\left(1-\frac{x}{m_B-m_b}\right)^a\exp{(1+a)\frac{x}{m_B-m_b}}
\end{equation}
One of the main advantages of the Kagan-Neubert model is the fact that it is readily available in \texttt{EvtGen} Monte Carlo generator, commonly used by $B$-factories \cite{Ryd:2005zz}.
The \BtoXsgamma inclusive decay is implemented as the \texttt{BTOXSGAMMA} model within the generator. 
This model has also been used in the analysis described in this thesis.

It should be noted that Kagan-Neubert shouldn't be considered `state-of-the-art' model for \BtoXsgamma. 
Different, and more recent, approaches exist for the description of the shape-function and, consequentially, \BtoXsgamma spectrum \cite{Benson:2004sg,Lange:2005yw,Andersen:2005mj,Gambino:2007rp,Aglietti:2007ik}.
For example, Ref. \cite{Andersen:2005mj} describe it based on a technique called Dressed Gluon Exponentiation.
The result of Ref. \cite{Lange:2005yw} provides several functional forms of the shape function, based on the leading-order shape-function that can be extracted from \BtoXsgamma decays.
A detailed discussion of these and other models is beyond the scope of this thesis, but overall they all depend on a specific choice of assumptions or representations of the shape-function.





