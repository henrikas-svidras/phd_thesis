\chapter{Theoretical overview}

In this Chapter, the theoretical overview of the Standard Model, flavour physics and \BtoXsgamma will be presented.
It will introduce the main theoretical concepts that are featured in this thesis and provide a theoretical grounding for the experimental analyses performed.
\todo[inline]{section listing here once they are finished}

\section{The Standard Model}

a $SU(3)\times SU(2)\times U(1)$ quantum field theory \cite{Peskin:1995ev}.
Its theoretical framework describes the electroweak and strong force interactions between elementary particles that constitute the world as we know it.
The elementary particles included in the Standard Model can have a half-integer spin (fermions) or integer spin (bosons).
The spin-1 bosons are the mediators of the electromagnetic (photon), weak ($\Wpm$ and $\Z$) and strong interactions (gluon) in the SM.
The spin-0 Higgs boson couples to all massive particles of the Standard Model via the Higgs mechanism \cite{PhysRevLett.13.508}.
The fermions are subsplit into two additional groups, quarks and leptons, which, respectively, can and cannot interact through the strong force.

\section{Flavour physics}

Ongoing...

\section{Radiative decays \texorpdfstring{\BtoXsdgamma}{B->Xsg}}

In the Standard Model, $b\ra s$ transitions are mediated by so-called flavour-changing neutral currents.
These processes can only occur through loops mediated by heavy particles.
One of the decay processes used to study these transitions are the so-called rare radiative $b\ra s\g$ and $b\ra d\g$ decays.
Due to colour confinement of the quarks, in reality they manifest as \BtoXsgamma or \BtoXdgamma, where $X_s$ and $X_d$ denote any meson state originating from the $s$ quark hadronisation.
The underlying Standard Model diagrams for \BtoXsdgamma processes are shown in \Cref{fig:sm_diagrams}.

\todo[inline]{citations here are needed, but I don't know how I want to proceed yet. Maybe this is common knowledge}

\begin{figure}[htbp!]
\resizebox{0.66\textwidth}{!}{
    \subcaptionbox{\label{fig:sm_diagrams}}{
        \begin{tikzpicture}
\begin{feynman}
\vertex (i1){b};
\vertex[right =1.2cm of i1] (a) ;
\vertex[right=0.85cm of a] (b);
\vertex[right=0.85cm of b] (c);
\vertex[right=1.2cm of c] (o1) {s,d};
\vertex[below=2em of c] (g1);
\vertex[right=2em of g1] (o2) {$\gamma$};

\diagram* {
(i1) -- [fermion] (a) -- [boson, edge label =\(W^{\pm}\)] (c) -- [fermion] (o1),
(a) -- [fermion,half left, edge label = \(uct\)] (c),
(b) -- [photon] (o2),
};
\end{feynman}
\end{tikzpicture}

        \begin{tikzpicture}
\begin{feynman}
\vertex (i1){b};
\vertex[right =1.2cm of i1] (a);
\vertex[right=0.85cm of a] (b);
\vertex[right=0.85cm of b] (c);
\vertex[right=1.2cm of c] (o1) {s,d};
\vertex[below=2em of c] (g1);
\vertex[right=2em of g1] (o2) {$\gamma$};

\diagram* {
(i1) -- [fermion] (a) -- [fermion, edge label =\(uct\)] (c) -- [fermion] (o1),
(a) -- [boson,half left, edge label = \(W^{\pm}\)] (c),
(b) -- [photon] (o2),
};
\end{feynman}
\end{tikzpicture}



    }
}
\resizebox{0.33\textwidth}{!}{
\subcaptionbox{\label{fig:bsm_diagrams}}{
    \begin{tikzpicture}
\begin{feynman}
\vertex (i1){b};
\vertex[right =1.2cm of i1] (a) ;
\vertex[right=0.85cm of a] (b);
\vertex[right=0.85cm of b] (c);
\vertex[right=1.2cm of c] (o1) {s,d};
\vertex[below=2em of c] (g1);
\vertex[right=2em of g1] (o2) {$\gamma$};

\diagram* {
(i1) -- [fermion] (a) -- [scalar, edge label =\(H^{\pm}\)] (c) -- [fermion] (o1),
(a) -- [fermion,half left, edge label = \(uct\)] (c),
(b) -- [photon] (o2),
};
\end{feynman}
\end{tikzpicture}

}
}
\caption{\label{fig:b_to_s_gamma_diagrams}
The Feynman diagrams for radiative $b\ra s$ transitions. 
\Cref{fig:sm_diagrams} shows the Standard Model diagrams, where the $b\ra s$ transition occurs via electroweak loops.
A beyond-Standard Model scenario where this transition is mediated by a charged-Higgs boson particle is shown in \Cref{fig:bsm_diagrams}.
}
\end{figure}