%====================================================================================================
\begin{abstractpage}{Abstract}
    This doctoral thesis documents an inclusive measurement of the photon energy spectrum in radiative \B meson decays 
    into final states involving a strange hadron and a photon, denoted \BtoXsgamma. 
    A partner $B$ meson from \FourS meson decays is reconstructed in hadronic channels.
    This is the first such measurement with the Belle~II experiment.
    The analysed dataset corresponds to $189~\invfb$ of integrated luminosity of electron-positron collisions provided by the SuperKEKB accelerator.
    Only the high energy photon from the signal \B meson is reconstructed in order to achieve an unbiased sample of final states involving strange hadrons, $X_s$.
    The partner \B meson is fully reconstructed in a large number of hadronic decay channels using a series of classifiers,
    allowing to infer the momentum and energy of the signal $B$ meson.
    The partial branching fraction of \BtoXsgamma is measured  in eight intervals of photon energy 
    in the signal $B$ meson rest frame within $1.8-2.7~\gev$.
    The obtained signal yield for this photon energy region is $343 \pm 122$ events. 
    In addition, integrated branching fractions, and the first and second moments of the photon energy spectrum are calculated for several photon energy thresholds.
    For the lowest energy threshold they are determined to be, 
    $(3.54 \pm 0.78 \pm 0.83)\cdot10^{-4}$,
    $(2.284 \pm 0.065 \pm 0.071)~\gev$,
    and $(0.0502 \pm 0.0157 \pm 0.0176)~\gev^2$,
    respectively, where the uncertainties are statistical and systematic.
    The results show excellent agreement with the Standard Model predictions and set the stage for future hadronic-tagged radiative measurements at Belle~II.
    \end{abstractpage}
%============================================================================a========================
    \begin{abstractpage}{Zusammenfassung}
        Diese Doktorarbeit dokumentiert eine umfassende Messung des Photonenenergiespektrums bei strahlenden \B-Meson-Zerfällen 
        in Endzustände mit einem fremden Hadron und einem Photon, bezeichnet als \BtoXsgamma. 
        Ein $B$-Partnermeson aus \FourS-Meson-Zerfällen wird in hadronischen Kanälen rekonstruiert.
        Dies ist die erste derartige Messung mit dem Belle~II Experiment.
        Der analysierte Datensatz entspricht $189~\invfb$ integrierter Luminosität von Elektron-Positron-Kollisionen, die vom SuperKEKB-Beschleuniger bereitgestellt wurden.
        Nur das hochenergetische Photon des Signalmesons \B wird rekonstruiert, um eine unverzerrte Stichprobe von Endzuständen mit seltsamen Hadronen $X_s$ zu erhalten.
        Das Partner-Meson \B wird in einer gro{\ss}en Anzahl von Hadronenzerfallskanälen mit Hilfe einer Reihe von Klassifikatoren vollständig rekonstruiert,
        die es erlauben, den Impuls und die Energie des Signalmesons $B$ zu bestimmen.
        Der partielle Verzweigungsverh\"altnis von \BtoXsgamma wird in acht Intervallen der Photonenenergie gemessen 
        im Ruhezustand des Signalmesons $B$ innerhalb von $1.8-2.7~\gev$ gemessen.
        Die erhaltene Signalausbeute für diesen Photonenenergiebereich beträgt $343 \pm 122$ Ereignisse. 
        Darüber hinaus werden integrierte Verzweigungsfraktionen und die ersten und zweiten Momente des Photonenenergiespektrums für verschiedene Photonenenergieschwellenwerte berechnet.
        Für die niedrigste Energieschwelle werden sie wie folgt bestimmt, 
        $(3.54 \pm 0.78 \pm 0.83)\cdot10^{-4}$,
        $(2.284 \pm 0.065 \pm 0.071)~\gev$,
        und $(0.0502 \pm 0.0157 \pm 0.0176)~\gev^2$,
        wobei die Unsicherheiten statistisch und systematisch sind.
        Die Ergebnisse zeigen eine hervorragende Übereinstimmung mit den Vorhersagen des Standardmodells und bilden die Grundlage für künftige Strahlungsmessungen an Belle~II mit hadronischen Markierungen.
    \end{abstractpage}
%====================================================================================================
    