%====================================================================================================
\begin{abstractpage}{Abstract}
    This doctoral thesis documents the measurement of the photon energy spectrum in radiative bottom meson (\B) decays 
    into inclusive final states involving a strange hadron and a photon, denoted \BtoXsgamma. 
    A partner $B$ meson from a \FourS mesons decaying into a pair of \B mesons decay is reconstructed decaying hadronically.
    This is the first such measurement with the Belle~II experiment.
    The analysed dataset corresponds to $189~\invfb$ of integrated luminosity of electron-positron collisions provided by the SuperKEKB accelerator and detected by the Belle~II experiment.
    Only the high energy photon from the signal \B meson is reconstructed in order to achieve an unbiased inclusive sample of final states involving strangeness, $X_s$.
    The partner \B mesons are fully reconstructed in a large number of hadronic decay channels using a series of classifiers,
    which allows to infer the momentum and energy of the signal $B$ meson.
    The \BtoXsgamma partial branching fractions are measured as a function of photon energy 
    in the signal $B$ meson rest frame in eight bins above $1.8~\gev$.
    The background-subtracted signal yield for this photon energy region is $343 \pm 122$ events. 
    Integrated branching fractions, first and the second moment of the photon energy spectrum are calculated for several photon energy thresholds.
    For the lowest energy threshold they are determined to be, respectively, 
    $(3.54 \pm 0.78 \pm 0.83)\cdot10^{-4}$,
    $(2.284 \pm 0.065 \pm 0.071)~\gev$,
    $(0.0502 \pm 0.0157 \pm 0.0176)~\gev^2$,
    where the uncertainties are statisical and systematic.
    The results show excellent agreement with the Standard Model predictions and set the stage for future hadronic-tagged radiative measurements at Belle~II.
    \end{abstractpage}
%============================================================================a========================
    \begin{abstractpage}{Zusammenfassung}
    Diese Doktorarbeit dokumentiert die Messung des Photonen-Energiespektrums in radiativen Bottom-Meson (\B)-Zerfällen 
    in inklusive Endzustände mit einem Strange-Hadron und einem Photon, bezeichnet als \BtoXsgamma. 
    Ein $B$-Partnermeson aus einem \FourS-Meson, das in ein Paar \B-Mesonen zerfällt, wird durch hadronischen Zerfall rekonstruiert.
    Dies ist die erste derartige Messung mit dem Belle~II Experiment.
    Der analysierte Datensatz entspricht $189~\invfb$ der integrierten Luminosität von Elektron-Positron-Kollisionen, die vom SuperKEKB-Beschleuniger geliefert und vom Belle~II-Experiment nachgewiesen wurden.
    Nur das hochenergetische Photon des Signalmesons \B wird rekonstruiert, um eine unverzerrte, umfassende Stichprobe von Endzuständen zu erhalten, die Strangeness, $X_s$, beinhalten.
    Die \B-Partnermesonen werden in einer gro{\ss}en Anzahl von hadronischen Zerfallskanälen mithilfe einer Reihe von Klassifikatoren vollständig rekonstruiert,
    die es erlauben, den Impuls und die Energie des Signalmesons $B$ zu bestimmen.
    Die partiellen Verzweigungsbrüche von \B zu Xsgamma werden als Funktion der Photonenenergie gemessen 
    im Ruhezustand des Signal-$B$-Mesons in acht Bins oberhalb von $1,8~\gev$ gemessen.
    Die untergrundsubtrahierte Signalausbeute für diesen Photonenenergiebereich beträgt $343 \pm 122$ Ereignisse. 
    Integrierte Verzweigungsbrüche, das erste und das zweite Moment des Photonenenergiespektrums werden für verschiedene Photonenenergieschwellenwerte berechnet.
    Für die niedrigste Energieschwelle werden sie wie folgt bestimmt 
    $(3.54 \pm 0.78 \pm 0.83)\cdot10^{-4}$,
    $(2.284 \pm 0.065 \pm 0.071)~\gev$,
    $(0.0502 \pm 0.0157 \pm 0.0176)~\gev^2$,
    wobei die Unsicherheiten statischer und systematischer Natur sind.
    Die Ergebnisse zeigen eine hervorragende Übereinstimmung mit den Vorhersagen des Standardmodells und bilden die Grundlage für künftige Strahlungsmessungen mit hadronischen Markierungen an Belle~II.
    \end{abstractpage}
%====================================================================================================
    