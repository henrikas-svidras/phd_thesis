%====================================================================================================
\begin{abstractpage}{Abstract}
    % This doctoral thesis documents an inclusive measurement of the photon energy spectrum in radiative \B meson decays 
    % into final states involving a strange hadron and a photon, denoted \BtoXsgamma. 
    % A data set of \FourS mesons is used that decay primarily in pairs of \B mesons.
    % The partner \B meson is fully reconstructed in a large number of hadronic decay channels using a series of multivariate classifiers.
    % The full reconstruction of the second \B meson allows to infer the momentum and energy of the signal $B$ meson in a technique called hadronic-tagging.
    % The analysed data set corresponds to an integrated luminosity of $189~\invfb$ of electron-positron collisions at the \FourS resonance energy provided by the SuperKEKB accelerator.
    % This is the first such measurement of the \BtoXsgamma decay with the Belle~II experiment.
    % Only the high energy photon from the signal \B meson decay is reconstructed in order to achieve an unbiased sample of final states involving strange hadrons.
    % The kinematic constraint from hadronic-tagging provides direct access to the signal \B meson rest frame, which provides better sensitivity to the photon energy spectrum without the additional smearing observed in the \FourS rest frame.
    % The partial branching fractions of the \BtoXsgamma decay are measured in eight intervals of photon energy 
    % in the signal \B meson rest frame between $1.8-2.7~\gev$.
    % The obtained signal yield for this photon energy range is $343 \pm 122$ events. 
    % The integrated branching fraction in this region is $(3.54 \pm 0.78~(\mathrm{stat.}) \pm 0.83~(\mathrm{syst.}))\cdot10^{-4}$.
    % Additionally, the first and second moments of the photon energy spectrum are calculated for several photon energy thresholds.
    % For photon energies above 1.8~\gev, they are determined to be, 
    % $(2.284 \pm 0.065~(\mathrm{stat.})\pm 0.071~()\mathrm{syst.}))~\gev$,
    % and $(0.0502 \pm 0.0157~(\mathrm{stat.}) \pm 0.0176~(\mathrm{syst.}))~\gev^2$,
    % respectively.
    % The results show excellent agreement with the Standard Model predictions and set the stage for future hadronic-tagged radiative measurements at Belle~II.
    This doctoral thesis presents an inclusive analysis of the photon energy spectrum in radiative \B meson decays (\BtoXsgamma) that involve strange hadrons and a photon.
    The data set consists of $\Upsilon(4S)$ mesons that primarily decay into pairs of \B mesons.
    The partner \B meson is fully reconstructed in a large number of hadronic decay channels using multivariate classifiers. 
    This full reconstruction allows for the inference of the momentum and energy of the signal \B meson through a technique referred to as hadronic-tagging.
    The data set analysed corresponds to an integrated luminosity of $189~\mathrm{fb}^{-1}$ of electron-positron collisions at the $\Upsilon(4S)$ resonance energy provided by the SuperKEKB accelerator. 
    This is the first measurement of the \BtoXsgamma decay with the Belle~II experiment.
    Only the high energy photon from the signal \B meson decay is reconstructed in order to achieve an unbiased inclusive sample of final states involving strange hadrons. 
    The hadronic-tagging provides direct access to the signal \B meson rest frame, leading to the photon energy spectrum without the additional kinematic smearing observed in the $\Upsilon(4S)$ rest frame.
    The partial branching fractions of the \BtoXsgamma decay are measured in eight intervals of photon energy in the signal \B meson rest frame between $1.8-2.7~\gev$. 
    The obtained signal yield for this photon energy range is $343 \pm 122$ events. 
    The integrated branching fraction in this region is found to be $(3.54 \pm 0.78~(\mathrm{stat.}) \pm 0.83~(\mathrm{sys.}))\cdot10^{-4}$.
    Additionally, the first and second moments of the photon energy spectrum are calculated for several photon energy thresholds. 
    For photon energies above $1.8~\gev$, they are determined to be $(2.284 \pm 0.065~(\mathrm{stat.})\pm 0.071~(\mathrm{syst.}))~\gev$ and $(0.0502 \pm 0.0157~(\mathrm{stat.}) \pm 0.0176~(\mathrm{syst.}))~\gev^2$, respectively.
    The results show excellent agreement with the Standard Model predictions and serve as a proof-of-concept for future hadronic-tagged radiative measurements at Belle~II.
        
    \end{abstractpage}
%============================================================================a========================
    \begin{abstractpage}{Zusammenfassung}
        \sloppy
        Diese Doktorarbeit beschreibt eine inklusive Analyse des Photonen-Energiespektrums bei radiativen \B-Meson-Zerfällen (\BtoXsgamma), an denen Strange-Hadronen und ein Photon beteiligt sind.
        Der Datensatz besteht aus $\Upsilon(4S)$-Mesonen, die hauptsächlich in Paare von \B-Mesonen zerfallen.
        Das Partner-\B-Meson wird in einer gro{\ss}en Anzahl von hadronischen Zerfallskanälen mithilfe von multivariaten Klassifikatoren vollständig rekonstruiert. 
        Diese vollständige Rekonstruktion ermöglicht den Rückschluss auf den Impuls und die Energie des Signal-\B-Mesons durch eine Technik, die als hadronisches Tagging bezeichnet wird.
        Der analysierte Datensatz entspricht einer integrierten Luminosität von $189~\mathrm{fb}^{-1}$ von Elektron-Positron-Kollisionen bei der $\Upsilon(4S)$-Resonanzenergie des SuperKEKB-Beschleunigers. 
        Dies ist die erste Messung des \BtoXsgamma-Zerfalls am Belle~II Experiment.
        Nur das hochenergetische Photon aus dem Zerfall des Signal-\B-Mesons wird rekonstruiert, um ein unverzerrtes, umfassendes Sample von Endzuständen zu erhalten, an denen Strange-Hadronen beteiligt sind. 
        Die kinematische Einschränkung durch das hadronische Tagging ermöglicht einen direkten Zugang zum Signal-\B-Meson-Ruhesystem, was zu einem Photonenenergiespektrum führt, ohne die zusätzliche kinematische Verschmierung, die im $\Upsilon(4S)$-Ruhesystem beobachtet wird.
        Das partielle Verzweigungsverh\"altnis des \BtoXsgamma-Zerfalls wird im Ruhesystem des Signal-\B-Mesons in acht Photonenergieintervallen innerhalb von $1.8 - 2.7~\gev$ gemessen. 
        Die erhaltene Anzahl von Signalkandidaten für diesen Photonenenergiebereich beträgt $343 \pm 122$. 
        Das integrierte Verzweigungsverhältnis in diesem Bereich beträgt \mbox{$(3.54 \pm 0.78~(\mathrm{stat.}) \pm 0.83~(\mathrm{sys.}))\cdot10^{-4}$}.
        Zusätzlich werden die ersten und zweiten Momente des Photonenenergiespektrums für mehrere Schwellenwerte der Photonenenergie berechnet. 
        Für Photonenenergien oberhalb von $1.8~\gev$ werden sie zu \mbox{$(2.284 \pm 0.065~(\mathrm{stat.})\pm 0.071~(\mathrm{syst.}))~\gev$} bzw. \mbox{$(0.0502 \pm 0.0157~(\mathrm{stat.}) \pm 0.0176~(\mathrm{syst.}))~\gev^2$} bestimmt.
        Die Ergebnisse zeigen eine hervorragende Übereinstimmung mit den Vorhersagen des Standardmodells und dienen als Proof-of-Concept für zukünftige Messungen von radiativen \BtoXsgamma Zerfällen mit hadronischem Tagging am Belle~II Experiment.
    \end{abstractpage}
%====================================================================================================
    