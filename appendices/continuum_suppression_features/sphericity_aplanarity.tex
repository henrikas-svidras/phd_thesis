Sphericity matrix is defined using a collection of momenta $\vec{p}_i$, as \cite{BaBar:2014omp}:
\begin{equation}
    S^{\alpha,\beta} = \frac{\sum^N_{i=1}p_i^{\alpha}p_i^{\beta}}{\sum^N_{i=1}|\vec{p}_i|^2},
\end{equation}
where $\alpha,\beta\in\{x,y,z\}$.
For an isotropic distribution its three eigenvalues, $\lambda_{1-3}$, are expected to be of similar size and matrix itself -- diagonal.
On the other hand, collimated distributions tend to have one of the values significantly smaller.
Therefore two values are tested in this analysis as continuum suppression features:
\begin{itemize}
    \item $\mathtt{sphericity}\equiv\frac{3}{2}(\lambda_2+\lambda_3)\in(0,1)$ (\Cref{fig:sphericity});
    \item $\mathtt{aplanarity}\equiv\frac{3}{2}\lambda_3\in(0,1)$ (\Cref{fig:aplanarity}).
\end{itemize}
In these definitions, $\lambda_{3(2)}$ is the (second-)smallest eigen value of the sphericity matrix.
A spherical event will have a \texttt{sphericity} close to $1$, and \texttt{aplanarity} close to $1/2$.
As the definitions of the sphericity matrix include momentum, these variables turn out to introduce a significant bias to the photon-energy spectrum, and are therefore not used.

\begin{figure}[htbp!]
    \subcaptionbox{\label{fig:sphericity}}{
        \includegraphics[width=1\textwidth]{figures/appendices/continuum_suppression_features/sphericity_aplanarity/sphericity_bias_tested.pdf}
    }
    \subcaptionbox{\label{fig:aplanarity}}{
        \includegraphics[width=1\textwidth]{figures/appendices/continuum_suppression_features/sphericity_aplanarity/aplanarity_bias_tested.pdf}

    }
    \caption{\label{fig:sphericity_aplanarity} The bias-test on \EB, \Estar and \Mbc for sphericity and aplanarity.
    The test is performed based on \textbf{Test~1} strategy, defined in \Cref{sec:continuum_features}.
    Variable definitions are given in text.}
\end{figure}