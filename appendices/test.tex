\documentclass{article}
\usepackage{amsmath}
\usepackage{cleveref}
\begin{document}
Consider an $\Upsilon(4S)$ event, produced at rest, in an $e^+e^-$ collision with $\sqrt{s}$.
The $\Upsilon(4S)$ subsequently decays into two $B$ mesons, $B$ and $B'$.
Let's assume the former undergoes a decay $B\to X_s \gamma$, where $X_s$ is some system of particles that cannot be measured.
In such case, the four-momentum of the $B$ meson is denoted as:
\begin{equation}
	p_{B} = (E_B^*,\vec{p}_{B}^*),
\end{equation}
and that of the photon:
\begin{equation}
	p_{\gamma} = (E_{\gamma}^*,\vec{p}_{\gamma}^*).
\end{equation}
Boosting into the $B$ rest frame from the $\Upsilon(4S)$ rest frame can be done
with the factors $\vec{\beta}=\vec{p}_{B}^*/E_{B}^*$ and $\gamma = 1/\sqrt{1-\beta^2}$:
\begin{equation}\label{eq:egamma}
	E_{\gamma}^B = \gamma(E_{\gamma}^* + \vec{\beta}\vec{p}_{\gamma}^*). 
\end{equation}
However, the $X_s$ system, as mentioned before cannot be measured and as a result, the value of $\beta$ cannot be evaluated.
Using the fact that $\Upsilon(4S)$ is stationary in its own rest frame, and the relation between $B$ and $B'$,
$p_{\Upsilon(4S)}=p_B^*+p'^*_B$, as they are produced in a two body decay of $\Upsilon(4S)$,
one arrives at
\begin{align}
	\begin{split}
		\sqrt{s} &= E^*_{B}+E'^*_{B},\\
		0 &= \vec{p}_B^* + \vec{p}'^*_{B}.\\
	\end{split}
\end{align}
Knowing the precise value of collision energy and the momentum of $B'$ is therefore sufficient to evaluate:
\begin{equation}
	\vec{\beta}= \vec{p}'^*_B/(\sqrt{s}-E'^*_{B})
\end{equation}
which can be substituted into \Cref{eq:egamma}.
\end{document}
