\chapter{Transformation to the \texorpdfstring{$B$}{B} meson rest frame} \label{sec:appendix_boosting_to_b_frame}

Consider an $\Upsilon(4S)$ event, produced at rest, in an $e^+e^-$ collision with a collision energy of $\sqrt{s}$.
The $\Upsilon(4S)$ subsequently decays into two $B$ mesons, $B$ and $B'$.
Let's assume the former undergoes a decay $B\to X Z$, where $X$ is some system of particles that cannot be measured, and $Z$ is a particle whose energy is known in the \FourS frame.
In such case, the four-momentum of the $B$ meson in the \FourS rest frame is denoted as:
\begin{equation}
	p_{B} = (E_B^*,\vec{p}_{B}^*),
\end{equation}
and that of the $Z$ particle:
\begin{equation}
	p_{Z} = (E_{Z}^*,\vec{p}_{Z}^*).
\end{equation}
Performing a Lorentz boost into the $B$ rest frame from the $\Upsilon(4S)$ rest frame can be done
with the factors $\vec{\beta}=-\vec{p}_{B}^*/E_{B}^*$ (the negative sign as we are boosting into a frame of reference where $B$ is stationary)
and $\gamma = 1/\sqrt{1-\beta^2}$:
\begin{equation}\label{eq:egamma}
	E_{Z}^B = \gamma(E_{Z}^* + \vec{\beta}\vec{p}_{Z}^*). 
\end{equation}
However, the $X$ system, as stated before, cannot be measured and, as a result, the value of $\beta$ cannot be evaluated from reconstructed decay products.
On the other hand, let's assume that the $B'$ is reconstructed, with its total momentum and energy known.
Using the fact that $\Upsilon(4S)$ is stationary in its rest frame and the four-momentum relation between $B$ and $B'$,
$p_{\Upsilon(4S)}=p_B^*+p^*_{B'}$, one arrives at
\begin{align}
	\begin{split}
		\sqrt{s} &= E^*_{B}+E^*_{B'},\\
		0 &= \vec{p}_B^* + \vec{p}^*_{B'}.\\
	\end{split}
\end{align}
Knowing the precise value of collision energy and the momentum of $B'$ is therefore sufficient to evaluate:
\begin{equation}
	\vec{\beta}= \vec{p}^*_{B'}/(\sqrt{s}-E^*_{B'})
\end{equation}
which can be substituted into \Cref{eq:egamma}.

